\subsubsection{Interfaccia per riga di comando}
Il comando che è necessario invocare per eseguire il compilatore Basalt è \texttt{basalt}. Esso 
espone una interfaccia di concezione moderna basata su sotto-comandi.

Tra i vari sotto-comandi disponibili, il più importante è \texttt{compile}, il quale
si occupa di tradurre il codice sorgente Basalt nel formato richiesto: \\

\noindent\hfill \texttt{basalt compile -i <source-files> -o <output-files> -t <target-triple>} \hfill\null \\

È possibile fornire più di un output file a patto che essi abbiano diverse estensioni, e Basalt si occuperà di 
compilare per il corretto formato. Le estensioni possibili sono: \\

\vspace{0.5cm}
\begin{table}[h]
    \centering
        \begin{tabularx}{\textwidth}{|b|b|} \hline
            \cheader{Estensione}                    & \cheader{Descrizione}                      \\ \hline
            \texttt{.ll / .llvm}:                   & file di testo in formato LLVM-IR           \\ \hline
            \texttt{.s  / .asm}:                    & file di testo in formato GNU-assembly      \\ \hline
            \texttt{.o  / .obj}:                    & file oggetto (in formato binario)          \\ \hline
        \end{tabularx}
    \caption{Estensioni supportate per output di compilazione}
\end{table}
\vspace{0.5cm}

I possibili valori per il campo \texttt{<target-triple>} sono stringhe della seguente forma: \\

\noindent\hfill \texttt{<arch><sub\_arch>-<vendor>-<sys>-<env>} \hfill\null \\

\begin{table}[h]
    \centering
        \begin{tabularx}{\textwidth}{|b|b|} \hline
            \cheader{Target-Triple}                 & \cheader{Descrizione} \\ \hline
            \texttt{x86\_64-pc-linux-gnu}           & Linux (x86)    \\ \hline
            \texttt{x86\_64-pc-winodws-msvc}        & Windows (x86)  \\ \hline
            \texttt{x86\_64-apple-darwin}           & Darwin (x86)   \\ \hline
        \end{tabularx}
    \caption{Esempi di target-triple per macchine x86}
\end{table}
\vspace{0.5cm}

\newpage
