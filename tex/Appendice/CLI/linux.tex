\subsubsection{Compilazione in ambienti Linux x86}
Al fine di compilare codice Basalt su sistemi Linux x86, 
è consigliabile avvalersi di \texttt{make} e \texttt{gcc}, due strumenti per 
riga di comando facilmente installabili sia manualmente che tramite i più comuni 
package manager (e.g: apt, pacman, dnf, ...). \texttt{gcc} è un compilatore per C/C++ che 
però è utilizzabile anche come linker: \\


\vspace{0.5cm}
\begin{lstlisting}[frame=single]
# build_x86_linux.mk

all: test
	./$^

test.o: src/*.bt
	basalt compile -i $^ -o $@ -t x86_64-pc-linux-gnu

test: test.o
	gcc -no-pie $^ -o $@ -lc -lm

\end{lstlisting}
\vspace{0.5cm}

Avendo cura di posizionarsi in una shell aperta nella directory in cui si 
trovano sia il file \texttt{build\_x86\_linux.mk} che la 
directory \texttt{src} contenente i file sorgente Basalt, allora 
lanciando il seguente comando si compileranno tutti i file sorgente in un 
eseguibile di nome \texttt{test}, il quale viene immediatamente eseguito: \\

\noindent\hfill \texttt{make -f build\_x86\_linux.mk} \hfill\null \\

L'insieme delle flag passate al linker sono necessarie per
linkare correttamente le librerie di sistema di Linux, necessarie al corretto 
funzionamento di qualsiasi programma. Viene anche linkata la libreria standard 
del C per poter consentire una facile integrazione mediante l'utilizzo di 
funzioni \textit{extern} e viene disabilitata la modalità \textit{PIE} (Position-Indipendent-Executable)
la quale è attualmente non supportata da Basalt. \\

Se si vuole, è possibile utilizzare altri linker come \texttt{ld}, \texttt{mold} o \texttt{clang} a patto
di modificare opportunamente il file \texttt{build\_x86\_linux.mk}. La scelta consigliata è \texttt{gcc} 
solo in quanto esso è tipicamente installato in qualsiasi distribuzione Linux. \\

Se lo si desidera, è possibile linkare asseieme ad object files di librerie esterne 
ed è possibile utilizzarle mediante l'opportuna dichiarazione di funzioni \textit{extern}. \\

\newpage