\section{Architettura}
In questo capitolo, sarà documentata l'architettura interna del compilatore. Per 
analizzare in dettaglio la suddetta architettura, saranno utilizzati svariate rappresentazioni grafiche 
in veste di diagramma, quali UML-Class-Diagram, UML-Sequence-Diagram, UML-Component-Diagram. Si tenga presente che tali
diagrammi saranno redatti considerando solo gli aspetti centrali del sistema, ignorando alcuni dettagli tecnici
dipendenti ad esempio dal linguaggio \texttt{C++} \\

Per brevità, tali diagrammi documenteranno solo ed esclusivamente metodi pubblici, tralasciando i meotdi
a visibilità ristretta e gli attributi di stato interno delle varie classi. Per facilitare la fruizione del documento
da parte di coloro non padroneggiano \texttt{C++}, le firme dei metodi saranno semplificate (utilizzando dei nomi più 
User-friendly per i tipi).\\