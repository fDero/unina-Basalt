Per quanto riguarda in particolare il sistema di generazione degli overload tramite CFA, esso modella gli overload auto-generati 
grazie alla classe \texttt{CommonFeatureAdoptionPlanDescriptor}. Essa è implementata tramite un variant, un'approccio tipico in C++
il quale non è facilmente rappresentabile in UML. Per rappresentare tale classe quindi, si è scelto di diagrammare una gerarchia 
di classi funzionalmente analoga a quanto implementato. Si tenga a mente dunque che il seguente diagramma è solo a scopo illustrativo,
ma non riflette fedelmente la struttura interna.

\begin{figure}[h]
\vspace{0.5cm}
\begin{adjustwidth}{-2cm}{-2cm} % Adjust left and right margins
\begin{center}
\begin{tikzpicture}[shift={(current page.west)}]
\begin{umlpackage}{model/typesystem}

\umlclass[x=4,y=0, scale=0.6]{CommonFeatureAdoptionPlanDescriptor}{
    ...
}{
    +store\_file\_representation(FileRepresentation) : void \\
    +get\_package\_name\_by\_file\_name(String) : String \\
    +get\_files\_by\_package\_name(String) : List<FileRepresentation> \\
    +get\_imports\_by\_file\_name(String) : List<String>
}

\umlclass[x=0,y=-4,scale=0.6]{RecursiveAdoptionPlan}{
    -argument\_index : Int \\
    -alternatives : List<TypeSignature> \\
    -nested\_plans : List<CommonFeatureAdoptionPlanDescriptor>
}{
    ... 
}

\umlclass[x=9,y=-4,scale=0.6]{DirectAdoptionPlan}{
   -function\_defintion : FunctionDefinition
}{
    ...
}

\umlinherit[geometry=|-]
{DirectAdoptionPlan}
{CommonFeatureAdoptionPlanDescriptor}


\umlinherit[geometry=|-]
{RecursiveAdoptionPlan}
{CommonFeatureAdoptionPlanDescriptor}


\umlaggreg[geometry=-|]
{RecursiveAdoptionPlan}
{CommonFeatureAdoptionPlanDescriptor}

\end{umlpackage}
\end{tikzpicture}
\end{center}
\end{adjustwidth}
\caption{UML-Class-Diagram CommonFeatureAdoptionPlanDescriptor}
\end{figure}
\vspace{0.5cm}

Tale struttura è atta a rappresentare un albero in memoria, dove ogni nodo foglia è una definizione di funzione, 
e ogni nodo non-foglia è un dispatch per un certo argomento di una funzione basato sul suo tipo. Per costruzione tale albero è N-ario, con N corrispondente
al numero di casi coperti dalla union (anonima o non) che rappresenta il tipo dell'argomento in questione. \\