\subsubsection{ANTLR: Another Tool for Language Recognition}
ANTLR è un progetto open-source che si occupa di offrire uno strumento per generare codice in vari linguaggi,
a partire da grammatiche e vocabolari in formato g4, che rappresentano la sintassi di un linguaggio di programmazione.
Il codice generato da ANTLR è difatti un'intera unità di frontend provvista di tokenizer (lexer), parser e anche 
di un visitor, ovvero una classe astratta che può essere estesa per implementare agevolmente meccanismi di 
navigazione dell'AST. \\ 

Assieme al codice autogenerato, è necessario aggiungere al progetto una dipendenza, ovvero il runtime di ANTLR,
che si occupa sostanzialmente di fornire le funzionalità di base sulle quali il codice autogenerato si basa. \\

ANTLR supporta vari linguaggi di programmazione, tra i quali Java, C\#, Python, JavaScript, Go e C++ e non solo.

\subsubsection{Considerazioni generali}
L'utilizzo di frameworks nello sviluppo di compilatori è una pratica ormai consolidata, anche se vale 
la pena chiedersi se tale pratica sia sempre conveniente. \\ 

LLVM è un framework largamente usato 
da numerosi compilatori per linguaggi di successo, diffusi, performanti, sicuri ed efficienti. LLVM ha 
dimostrato di essere un framework affidabile e robusto, e soprattutto la sua larga user-base ne garantisce
la manutenzione e l'aggiornamento costante, il quale a sua volta garantisce la compatibilità con le ultime
versioni di CPU e architetture, che in un periodo come questo è molto importante data la proliferazione di 
nuove archietture ARM / RISC-V. \\

L'egemonia dell'architettura x86 sarà, con ogni
probabilità, messa in discussione nei prossimi decenni, pertanto lo sviluppo di un backend da zero, richiederebbe
uno studio approfondito di molti standard, diversi fra loro, ed in continuo aggiornamento.  \\ 

In definitiva, l'utilizzo di LLVM come backend-framework è una scelta saggia, che permette di concentrarsi
su ciò che realmente conta, ovvero le symbol-tables e l'analisi statica. \\

Per quanto ANTLR invece, la situazione è un po' diversa. ANTLR è un framework molto potente, ma vale la pena 
sottolineare che esso, occupandosi di frontend a tutto tondo, si occupa anche della gestione dei commenti e nell'emissione
degli errori di sintassi. Questo comporta un minor controllo sul tipo di feedback da dare all'utente, il quale potrebbe
ricevere messaggi di errore non chiari o addirittura fuorvianti. \\

Inoltre, utilizzare ANTLR in C++, porta a dei problemi di compatibilità per via dell'utilizzo di alcune features 
deprecate dei vecchi standard C++ all'interno del codice autogenerato. Ciò è facilmente risolvibile mediante 
correzione manuale, ma ciò andrebbe a inficiare la facilità di manutenzione e di aggiornamento, rendendo difficile 
modificare la grammatica del linguaggio (ogni modifica della grammatica richiederebbe una nuova generazione, 
e dunque una nuova correzione manuale). \\

\newpage