\subsubsection{Utilizzo della memoria in LL}
In LL, la memoria è gestita in modo drasticamente diverso rispetto 
a come avviene nei linguaggi di programmazione di uso comune. \\ 

Nei linguaggi di programmazione tipici una variabile è associata ad un indirizzo 
di memoria, ed è possibile assegnare valori a tale variabile riferendosi ad essa con il suo nome. Quando 
servirà leggere il valore di tale variabile, sarà sufficiente, ancora una volta, riferirsi ad essa 
con il suo nome. \\

In LL ciò non avviene, in quanto le variabili non hanno un vero e proprio indirizzo di memoria, 
e sono invece solo dei simboli. Si consideri ad esempio il seguente frammento di codice Basalt, 
ed alla sua traduzione in LL:

\vspace{0.5cm}
\begin{lstlisting}[frame=single]
var number : Int = 42;
console::println(number);
\end{lstlisting}

\begin{lstlisting}[frame=single]
example_block:
  %numberAddress = alloca i64
  store i64 42, i64* %numberAddress
  %numberValue = load i64, i64* %numberAddress
  call void @"console::println<Int>"(i64 %numberValue)
\end{lstlisting}
\vspace{0.5cm}

Si noti come la variabile \texttt{number} sia stata tradotta come la \texttt{numberAddress} e \texttt{numberValue}.
La variabile \texttt{numberAddress} è un puntatore ad una locazione di memoria, e la variabile \texttt{numberValue}
è il valore numerico contenuto in tale area di memoria. \\

Nè \texttt{numberValue} nè \texttt{numberAddress} in 
questo caso hanno dei veri e propri indirizzi di memoria propri. Essi sono solo dei simboli 
a cui sono associati dei valori. \\

La gestione della memoria di LL può sembrare controintuitiva al programmatore medio, ed è proprio per questo 
che anche se LL è un vero e proprio linguaggio, esso non è considerato veramente utilizzabile. Si è infatti abituati
a ragionare in termini di variabili, intese come coppie indirizzo/valore, ed un approccio simile è facilmente 
classificabile come innaturale. \\

È bene ricordarsi però che i calcolatori utilizzano internamente una rappresentazione 
analoga a quella proposta da LL, ed è proprio per questo che LL è facilmente ed efficientemente traducibile 
in linguaggio macchina nativo per molteplici architetture. \\