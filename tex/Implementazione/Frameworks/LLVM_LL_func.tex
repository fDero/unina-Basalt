\subsubsection{Introduzione ad LL (LLVM-IR)}
LLVM, come già detto nel paragrafo precedente, è un progetto composto da un 
insieme di librerie, e da alcuni tool. Due di questi tool sono \texttt{lli} ed 
\texttt{llc}, rispettivamente un interprete IR e un compilatore IR. \\

Questi due strumenti possono essere usati per eseguire codice IR, fornito 
sotto forma di file di testo con estensione \texttt{.ll}. Analizzare la sintassi 
e, più in generale, la semantica di tale formato testuale renderà più facile comprendere 
l'utilizzo della API nei capitoli successivi. \\

Un simbolo globale deve essere preceduto dal carattere speciale "@", ad 
esempio la funzione che fa da entry-point per l'esecuzione del programma 
è definita come \texttt{@main}. \\ 

Per definire una funzione è necessario specificare il tipo di ritorno, il 
nome della funzione, e la lista dei parametri. Il corpo della funzione 
è racchiuso tra parentesi graffe. \\ 

\begin{lstlisting}[frame=single, language=LLVM]
define i32 @main() {
  entry:
    ret i32 0
}
\end{lstlisting}
\vspace{0.5cm}

Una funzione in LL è fatta da uno o più blocchi, ognuno dei quali è composto da 
una lista di istruzioni. Ogni blocco termina con un'istruzione di return o di salto. 
Nella funzione \texttt{@main} definita sopra, \texttt{entry} è una label, e definisce 
l'ingresso nell'omonimo blocco. Il blocco entry è un blocco speciale in quanto esso è 
il primo ad essere eseguito in ogni funzione. \\

L'istruzione \texttt{ret} effettua il return e, pertanto, sancisce la fine del blocco 
corrente. Il valore restituito deve essere dello stesso tipo specificato nella 
firma della funzione e tale tipo va esplicitamente specificato. \\

I nomi dei simboli possono contenere spazi e/o caratteri speciali purchè
racchiusi tra doppi apici. Per effettuare un return da una funzione void 
servirà scrivere \texttt{ret void}. \\

\begin{lstlisting}[frame=single]
define void %"example void function"() {
  entry:
    ret void
}
\end{lstlisting}
\vspace{0.5cm}

L'utilizzo di "\%" in luogo di "@" identifica un simbolo come locale invece che globale, pertanto 
esso non sarà visibile all'esterno del file \texttt{.ll} in cui esso è definito.