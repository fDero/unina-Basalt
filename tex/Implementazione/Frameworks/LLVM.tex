\subsection{LLVM: Low Level Virtual Machine}
Ciò che accomuna Basalt, Rust, Zig e molti altri linguaggi moderni è l'adozione di LLVM come backend-framework.
LLVM è un progetto open-source che si occupa di offrire una API per costruire manualmente rappresentazioni IR per 
le varie istruzioni di un programma, e di fornire un set di tool per ottimizzare e generare codice macchina a partire
da tali rappresentazioni IR. \\ 

In altri termini, LLVM si occupa di implementare le fasi di IR-optimization e codegen discusse in precedenza, 
essendo cross-platform, ovvero supportando varie CPU e varie architetture (x86, ARM, Web-Assembly, etc...). \\

LLVM è nato come libreria in C, ma sono stati successivamente scritti porting per C++ in primis, e a seguire 
per Rust ed Ada. \\

LLVM ha infine reso possibile la creazione di cling, un interprete C/C++ basato sul compilatore clang, il quale 
utilizza LLVM per effettuare JIT-compilation (compilazione "just in time", ovvero al volo) di codice C/C++. Tale 
strumento è tutt'ora di fondamentale importanza per la comunità scientifica, ed è in utilizzo presso il CERN.

\newpage