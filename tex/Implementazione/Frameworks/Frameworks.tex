\subsection{Frontend/backend compiler-frameworks}
Adottando una terminologia diffusasi primariamente nel contesto della programmazione web, 
è possibile introdurre i concetti di frontend e backend in un compilatore. Con il frontend di un 
compilatore si intende l'insieme di componenti software (package, classi, metodi, funzioni) che
si occupano di implementare le fasi di tokenizzazione e di parsing del codice sorgente. Con il termine 
backend, invece, si fa riferimento all'insieme di componenti software che si occupano di implementare
le fasi di ottimizzazione dell'IR e di codegen. \\

Con il passare del tempo, ci si è accorti che estrapolare dalle codebase dei vari compilatori 
i loro frontend e backend, per poi generalizzarli e renderli riutilizzabili, avrebbe portato a
una maggiore efficienza nello sviluppo di nuovi linguaggi. Questo ha portato alla nascita di
frameworks per la creazione di compilatori, che offrono funzionalità più o meno avanzate 
negli ambiti del frontend e del backend, in modo da lasciare liberi i designer di un dato compilatore 
di concentrarsi sull'analisi statica. \\

Questo ha consentito, ad esempio, a linguaggi come Rust di svilupparsi in termini di features semantiche,
quali ad esempio il borrow checker, senza dover preoccuparsi prima di implementare un backend perfettamente
funzionante da zero.

\subfile{LLVM_Intro}
\newpage

\subfile{LLVM_LL_func}
\newpage

\subfile{LLVM_LL_stack}
\newpage

\subfile{LLVM_LL_mem}
\newpage

\subfile{ANTLR_Intro}
\subfile{ANTLR_grammar}
\newpage

\subfile{ANTLR_usage}
\subfile{ConsiderazioniGenerali}
\newpage