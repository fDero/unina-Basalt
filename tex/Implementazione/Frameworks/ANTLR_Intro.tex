\subsubsection{ANTLR: Another Tool for Language Recognition}
ANTLR è un progetto open-source che si occupa di offrire uno strumento per generare codice in vari linguaggi,
a partire da grammatiche e vocabolari in formato g4, che rappresentano la sintassi di un linguaggio di programmazione.
Il codice generato da ANTLR è difatti un'intera unità di frontend provvista di tokenizer (lexer), parser e anche 
di un visitor, ovvero una classe astratta che può essere estesa per implementare agevolmente meccanismi di 
navigazione dell'AST. \\ 

Assieme al codice autogenerato, è necessario aggiungere al progetto una dipendenza, ovvero il runtime di ANTLR,
che si occupa sostanzialmente di fornire le funzionalità di base sulle quali il codice autogenerato si basa. \\