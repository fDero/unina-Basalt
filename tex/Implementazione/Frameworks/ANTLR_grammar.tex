\subsubsection{ANTLR: Vocabolari e grammatiche}
ANTLR lavora con due file di testo in formato \texttt{.g4}, che rappresentano rispettivamente 
il vocabolario e la grammatica del linguaggio. \\

Un vocabolario è un file contentente le regole di tokenizzazione in forma di espressioni regolari. 
Ogni regola è composta da un nome, seguito dai due punti, e da un'espressione regolare oppure un exact-match 
delimitato da apici. \\

Il seguente frammento di vocabolario \texttt{.g4} è un'estratto dal vocabolario di Basalt,
dove si definiscono i token \texttt{ID} e \texttt{TYPENAME}, rispettivamente per gli identificatori
(nomi di variabili, costanti, funzioni) e per i nomi dei tipi. \\

\vspace{0.3cm}
\begin{lstlisting}[frame=single]
ID       : [a-z][a-zA-Z_0-9]* ;
TYPENAME : [A-Z][a-zA-Z_0-9]* ;
\end{lstlisting}
\vspace{0.3cm}

Una grammatica è un file contentente le regole di parsing del linguaggio. Ogni regola è composta da un nome,
seguito dai due punti, e da una sequenza di token e/o regole eventualmente combinate. \\

Il seguente frammento di grammatica \texttt{.g4} è un'estratto dalla grammatica di Basalt, dove si definiscono
le regole per il parsing di variabili e costanti.  \\

\vspace{0.3cm}
\begin{lstlisting}[frame=single]
variableDeclaration
    : VAR ID COLON typeSignature ASSIGN expr SEMICOLON
    | VAR ID COLON typeSignature SEMICOLON
    ;
    
constDeclaration
    : CONST ID COLON typeSignature ASSIGN expr SEMICOLON
    ; 
\end{lstlisting}
\vspace{0.3cm}