\subsubsection{Implementazione di strutture dati in LL}
È possibile definire delle struct in LL, le quali sono simboli come le funzioni e possono essere 
globali o locali. Di seguito è riportato un esempio di definizione di due struct locali, che rappresentano 
rispettivamente uno stack (struttura dati) ed un suo nodo. \\

\vspace{0.5cm}
\begin{lstlisting}[frame=single]
%Stack = type { 
    %Node*, 
    i64 
}

%Node = type { 
    %Node*, 
    i32 
}
\end{lstlisting}
\vspace{0.5cm}

Volendo mostrare solo un esempio di come è possibile usare le struct definite sopra,
si riporta la definizione di una funzione che stampa tutti gli elementi di uno stack. \\

Nel seguente frammento di codice \texttt{.ll}, si può notare che per chiamare una funzione, 
si utilizza l'istruzione \texttt{call}, la quale richiede l'esplicita specificazione del tipo
di ritorno della funzione chiamata. L'istruzione \texttt{br} è una istruzione di salto. \\

\vspace{0.5cm}
\begin{lstlisting}[frame=single]
define void @printStack(%Stack* %stack) {
  entry:
    %cursorPtr = call %Node* @getHead(%Stack* %stack)
    br label %loopCondition

  loopCondition:
    %cursor = load %Node*, %Node** %cursorPtr    
    %condition = icmp ne %Node* %cursor, null
    br i1 %condition, label %loopBody, label %loopExit

  loopBody:
    %currentNumberPtr = call i32* @getNumber(%Node* %cursor)
    %currentNumber = load i32, i32* %currentNumberPtr
    call void @printInt(i32 %currentNumber) 
    %currentNextPtr = call %Node* @getNext(%Node* %cursor)
    %currentNext = load %Node*, %Node** %currentNextPtr
    store %Node* %currentNext, %Node** %cursorPtr
    br label %loopCondition

  loopExit:
    ret void
}
\end{lstlisting}
\vspace{0.5cm}