\subsubsection{Ottimizzazione dell'IR}
Una volta che l'IR è stato generato, esso si assume semanticamente corretto, e si può finalmente perdere traccia 
dell'AST, delle symbol-table e di tutte le altre strutture dati intermedie, comprese tutte le informazioni relative 
alle coordinate dei vari elementi del programma all'interno del sorgente. \\


La fase di ottimizzazione dell'IR consiste nel migliorare il codice IR mediante l'applicazione di trasformazioni
che ne riducano la complessità e ne migliorino le prestazioni ma che ne lasciano invariati gli effetti osservabili. \\

Le ottimizzazioni possono essere di vario tipo, alcune delle più comuni sono:

\begin{itemize}
    \item \textbf{Constant folding}: Calcolo di espressioni costanti in fase di compilazione.
    
    \item \textbf{Common subexpression elimination}: Rimozione di espressioni comuni.
    
    \item \textbf{Loop minimization}: Estrazione di codice dal corpo dei cicli iterativi.
    
    \item \textbf{Inlining}: Sostituzione di chiamate a funzione con il corpo della funzione stessa.
    
    \item \textbf{Register optimization}: Utilizzo più efficiente dei registri del processore.

    \item \textbf{Code Reordering}: Riordinamento di istruzioni non dipendenti fra loro per
    poter raggruppare istruzioni che possono essere eseguite in parallelo in un'unica sezione.
    
    \item \textbf{SIMD}: Utilizzo di istruzioni SIMD (Single Instruction Multiple Data) al 
    posto di sequenze di istruzioni normali ripetute (Possibile solo per alcune architetture). 
\end{itemize}

Una trattazione dettagliata dell'argomento è fuori dallo scopo di questo documento, in quanto è un 
ambito estremamente ampio ed in continua espansione. \\