\subsubsection{Validazione ed analisi statica}
La fase di validazione è una delle fasi più importanti del processo di compilazione. Essa consiste 
nel navigare con uno o più visitor l'AST generato dalla fase di parsing e verificare che esso sia
corretto rispetto a determinate regole semantiche. \\

Basalt, effettua i seguenti controlli di validità sul codice sorgente:
\begin{itemize}
    \item Aciclicità delle dipendenze dirette tra tipi: (Non esistenza di struct o union definite per ricorsione diretta)
    
    \item Non ambiguità dei tipi (Non esistenza di tipi con lo stesso nome e con lo stesso numero di parametri 
    formali di tipo nello stesso package, o in package diversi ma importati in uno stesso file)

    \item Address sanitizing (Verifica che non si stia provando a calcolare l'indirizzo di un entità non allocata)

    \item Typechecking (Verifica che tutti i tipi usati esistano e siano coerenti con il contesto, controllo degli 
    assignments per correttezza di tipo, risoluzione delle chiamate a funzione e controllo sui tipi dei parametri, sui 
    generics e sul tipo di ritorno)

    \item Immutability-checking (ispezione degli assignments e delle chiamate a funzioni ia fini di impedire
    modifiche a entità immutabili quali costanti e/o espressioni di sola lettura)

    \item Exit-path-checking (ispezione dei flussi di esecuzione delle funzioni, ai fini di garantire l'assenza di 
    codice irraggiungibile e la presenza di un return statement in tutti i possibili flussi di esecuzione per funzioni 
    non void)
\end{itemize}