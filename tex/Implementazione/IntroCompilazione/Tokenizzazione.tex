\subsubsection{Tokenizzazione}
Con tokenizzazione, si intende il processo di suddivisione del codice sorgente in \textit{token}, ovvero in unità 
atomiche che rappresentano i componenti del linguaggio. Queste unità atomiche possono essere rappresentate come 
semplici stringhe, oppure come entità più ricche di informazioni, come ad esempio il nome del file sorgente dal quale
sono stati estratti, la posizione all'interno del file (riga, colonna) e così via.\\

Il processo di tokenizzazione è il primo passo del processo di compilazione, ed è possibile implementare un tokenizzatore
in diversi modi. In generale, è utile immaginare un tokenizzatore come un algoritmo iterativo che dato un input testuale
continua a leggere caratteri finchè essi non formano un token valido. Una volta che un token è stato riconosciuto, 
esso viene conservato in una opportuna collezione. \\

Commenti, spazi e alcuni caratteri speciali sono generalmente ignorati dal tokenizzatore, nel senso che essi 
vengono correttamente riconosciuti ma non vengono conservati nella collezione. \\

Un token che non viene riconosciuto porta ad un errore a tempo di compilazione. 