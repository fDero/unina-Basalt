\subsection{Indicizzazione ed analisi statica}
Con indicizzazione si riferisce in generale al processo di creazione delle symbol-tables
e della loro gestione. In questo capitolo si affronterà il tema della logica interna 
del compilatore Basalt che passa per l'indicizzazione di tutte le definizioni di tutti i file 
sorgente e che si conclude nell'analisi statica del programma indicizzato. \\

\subfile{ProgramRepresentation}
\newpage

\subfile{ProjectFileStructure}
\newpage

\subsubsection{Controllo di aciclicità delle dipendenze dirette fra tipi}
\subsubsection{Controllo di non-shadowing dei tipi}
\subsubsection{Tracciamento degli scope e delle definizioni locali}
\subsubsection{Typechecking}
\subsubsection{Algoritmo di Type-inference}
\subsubsection{Scoring degli overload}
\subsubsection{Costruzione della tabella delle funzioni}
\subsubsection{Generics: Sistema di reificazione}
\subsubsection{Gestione ad alto livello della CFA}
\subsubsection{Astrazione rispetto ad overload semplici ed overload CFA}
