\subsubsection{Build automatizzata}
Posto che la build del compilatore nella sua versione basata su ANTLR è manuale, e può essere condotta 
solo usando specificamente il compilatore gcc con lo standard 17, la build del compilatore Basalt
nella sua versione proposta sulla repository principale è automatizzata. \\

La build automatizzata utilizza conan, un package manager per C++, 
per scaricare le dipendenze del progetto (LLVM e libxml2), le quali sono 
poi compilate in locale automaticamente solo al primo utilizzo. \\

Scaricate le dipendenze, il progetto viene compilato con cmake, che è uno strumento 
che consente di effettuare build incrementali di progetti C e C++. \\

\subsubsection{Installer per Windows x86}
Per facilitare l'installazione del compilatore Basalt su sistemi Windows x86, è stato creato un installer
automatico nel formato msi (Microsoft Installer). \\

Tale installer posiziona l'eseguibile (staticamente linkato) del compilatore Basalt nella directory 
\texttt{\%Program Files\%}\textbackslash\texttt{basalt}\textbackslash\texttt{<version>}, 
e aggiunge la directory alla variabile d'ambiente \texttt{PATH}. In fase di disinstallazione, 
che avviene dal pannello di controllo, rimuove l'eseguibile e la directory dalla variabile d'ambiente. \\

Tale installer è stato creato con WiX Toolset, uno strumento del \texttt{.NET} framwork di Microsoft.
Esso mostra in fase di installazione una End-User License Agreement (EULA) in formato rtf (Rich text format). \\

\subsubsection{Package per linux}
Per facilitare l'installazione del compilatore Basalt su sistemi Linux, è stata predisposta 
la pacchettizazione per snapcraft (package manager di canonical). \\

Snapcraft è uno strumento che consente di creare pacchetti snap, che sono pacchetti software
sottoforma di container, che contengono tutte le dipendenze necessarie per l'esecuzione del software. 
Snapcraft è nato nell'ecosistema Ubuntu, ma è possibile utilizzarlo anche in altre distribuzioni Linux 
basate su Debian, e il supporto è attualmente in espansione anche per altre distribuzioni. \\ 

Il pacchetto non è attualmente sulle repository ufficiali di snapcraft, ma è possibile installarlo
scaricandolo manualmente dalla repository github del progetto. Il nome per la futura pubblicazione di 
Basalt sui repository ufficiali di Snapcraft è già stato riservato. \\