\subsection{Sviluppo del compilatore Basalt}
Alla luce di quanto detto nei capitoli precedenti, si è deciso di sviluppare il compilatore 
per il linguaggio di programmazione Basalt in C++. Questa scelta è stata fatta per diversi motivi, 
tra cui:

\begin{itemize}
    \item \textbf{Facilità di utilizzo del compilatore}: É stato dato un particolare peso alla facilità di utilizzo 
    del prodotto finito. Un compilatore scritto in Java ad esempio, avrebbe richiesto l'installazione di una JVM, 
    rendendo il prodotto meno accessibile (Considerazioni analoghe possono essere fatte per C\# o simili). 
    Inoltre, in C++ è possibile creare eseguibili linkati staticamente, semplificando il processo di 
    distribuzione del compilatore
    
    \item \textbf{Performance}: C++ è un linguaggio altamente performante, requisito
    fondamentale per un compilatore. Un compilatore scritto in un linguaggio più lento avrebbe richiesto
    tempi di compilazione più lunghi, rendendo l'esperienza dell'utente finale meno piacevole

    \item \textbf{Supporto di LLVM}: L'adozione di LLVM è stata una scelta compiuta fin dalle primissime fasi di 
    design del compilatore. La pool di linguaggi supportati da LLVM è comunque piuttosto ristretta, rendendo 
    C++ una scelta quasi obbligata. Inoltre, LLVM è scritto in C++, rendendo la scelta ancora più naturale.
\end{itemize}

\subfile{DueRepo.tex}
\subfile{Deploy}