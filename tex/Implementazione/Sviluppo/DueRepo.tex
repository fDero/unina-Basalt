\subsubsection{Doppia repository: Con e senza ANTLR}
Il progetto Basalt è stato suddiviso in due repository distinte. Una di esse, denominata \textit{Basalt}, 
contiene il compilatore vero e proprio, scritto in C++ e basato su LLVM. L'altra, denominata \textit{unina-Basalt}, 
contiene l'adattamento della codebase principale ad di ANTLR4. \\ 

Sono state create due repository per toccare con mano i vantaggi e gli svantaggi dell'utilizzo di ANTLR,
i quali sono stati già discussi in precedenza. É opportuno evidenziare come se ANTLR non causasse 
problemi di compatibiltà che inficiano la riproducibiltà delle build, la repository basata su ANTLR avrebbe
rimpiazzato la repository principale.  \\



\vspace{0.5cm}
\begin{table}[h]
    \centering
        \begin{tabularx}{\textwidth}{|b|b|} \hline
            \cheader{Repository} & \cheader{URL}                                                          \\ \hline
            \texttt{unina-Basalt}: basata su ANTLR4 & \texttt{www.github.com/fDero/unina-Basalt} \\ \hline
            \texttt{Basalt}: principale             & \texttt{www.github.com/fDero/Basalt}       \\ \hline
            
        \end{tabularx}
    \caption{Repository github}
\end{table}
\vspace{0.5cm}

\newpage