\subsubsection{Scoring degli overload}
Durante il processo di \textit{overloading resolution}, viene generato un \texttt{FunctionSpecificityDescriptor}
per ogni funzione candidata. Questo oggetto contiene informazioni utili per individuare la funzione più specifica. \\

Una considerazione da fare è che due funzioni in Basalt possono essere comparate per specificità in senso assoluto, 
e non necessariamente in relazione al contesto in cui sono state chiamate. Questo semplifica di molto la logica di 
attribuzione dello score. Si ricordi:

\subfile{../../Design/Funzioni/SpecificityRules}

Come è possibile constatare dalle regole sopra elencate, la specificità di una funzione è un concetto che può essere
valutato in modo assoluto, senza dover considerare il contesto in cui la funzione è stata chiamata. \\

In particolare, la valutazione delle conversioni di tipo è un'operazione che può essere effettuata in modo assoluto, in quanto,
per un qualunque tipo \texttt{T}, il compilatore è a conoscenza del numero esatto di tipi diversi \texttt{U$_1$}, \texttt{U$_2$}, ..., \texttt{U$_i$}
tali che espressioni di quel tipo siano assegnabili a \texttt{T}. Questo numero è finito e noto a priori previa l'opportuna 
interrogazione della tabella dei tipi. \\

Quindi, in generale, è possibile ordinare tutte le funzioni di un overload set per specificità in modo assoluto. Poi, la/le più specifica/e tra 
queste potrebbero venire scartate per incompatibilità con i tipi concreti degli argomenti, difatti portando ad una corretta overloading resolution. \\

Un \texttt{FunctionSpecificityDescriptor} è quindi una strcut che contiene diversi indici corrispondenti a ciascuna delle regole sopra elencate, 
sul quale viene effettuato un confronto secondo il principio dell'ordinamento lessicografico.