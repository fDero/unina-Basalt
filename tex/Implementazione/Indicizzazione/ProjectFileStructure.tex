
\subsubsection{Costruzione della tabella dei file/package}
La tabella dei file e dei package è una struttura dati che traccia le relazioni fra 
diversi file sorgenti. Essa tiene traccia dei package importati da ogni file e dei file 
contenuti in ogni package. \\

Internamente essa è implementata sotto il nome di \texttt{ProjectFileStructure} 
ed è un wrapper su delle hash-map che permettono di
effettuare ricerche in tempo costante. \\

\vspace{0.5cm}
\begin{lstlisting}[frame=single]
class ProjectFileStructure {

    public:
        ProjectFileStructure() = default;
        ProjectFileStructure(vector<FileRepresentation>&);

        /* public exposed API */

    private:
        /* other internal data-structures */ 
        
        list<string> package_names;
        unordered_map<string, string> package_name_by_file_name;
        unordered_map<string, list<FileRepresentation>> files_by_pkg;
        unordered_map<string, vector<string>> imports_by_file;
};
\end{lstlisting}
\vspace{0.5cm}

Tale classe possiede i metodi necessari ad iterare sui file di un package, a trovare
il package di un file, a trovare i file importati da un file e soprattutto a 
registrarli (e.g. indicizzarli, conservarli all'interno della struttura dati). \\

I restanti metodi della classe sono di supporto e permettono di iterare sui file e 
sui package, o di ottenere informazioni specifiche per un dato file o package. \\

Durante la compilazione di un programma Basalt, viene costruita una ed una sola 
istanza di \texttt{ProjectFileStructure} che sarà passata come parametro di costruttore 
alle restanti tabelle e strutture dati che necessitano le informazioni che essa codifica. \\

L'utilizzo della lista texttt{package\_names} permette di avere una collezione da poter iterare 
anche durante la modifica, ciò consente di poter alterare le mappe sottostanti senza invalidare 
l'iterazione. \\

In C++ infatti, non è possibile modificare una hash-map durante l'iterazione, in quanto 
l'iteratore potrebbe diventare invalido (ciò non avviene con le liste).