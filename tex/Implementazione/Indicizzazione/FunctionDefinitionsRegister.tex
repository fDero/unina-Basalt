\subsubsection{Costruzione della tabella delle funzioni}
La tabella dei tipi è una struttura dati che traccia le definizioni di tutte le funzioni
e rende possibile risalire ad esse partendo da un oggetto di tipo \texttt{FunctionCall} e dai 
tipi concreti degli argomenti passati alla chiamata. \\

\vspace{0.5cm}
\begin{lstlisting}[frame=single]
class FunctionDefinitionsRegister {

    public:
        FunctionDefinitionsRegister(
            ProjectFileStructure&,
            TypeDefinitionsRegister&
        );

        void store_function_definition(FunctionDefinition&);

        FunctionDefinition retrieve_function_definition(
            const FunctionCall& function_call, 
            const vector<TypeSignature>& arg_types
        );

        /* public exposed API */

    protected:
        /* other internal methods */

    private:
        /* other internal data-structures */

        list<FunctionDefinition> function_definitions;
        unordered_map<string, OverloadSet> overload_sets;
        unordered_map<string, FunctionDefinition> fast_retrieve_cache;
        
        TypeDefinitionsRegister& type_definitions_register;
        ProjectFileStructure& project_file_structure;
};
\end{lstlisting}
\vspace{0.5cm}

Così come per le altre tabelle, anche la tabella delle funzioni tiene traccia di tutte le definizioni
di funzioni presenti nel progetto usando una lista, così da potervi iterare anche quando la si sta modificando 
(ad esempio generando nuove definizioni tramite reificazione). \\

Per effettuare il recupero della definizione a funzione a partire da una chiamata, è necessario 
fornire anche i tipi concreti degli argomenti passati alla chiamata per poter effettuare overloading resolution 
basandosi proprio sui suddetti tipi. La classe \texttt{ProgramRepresentation} si occupa di generare tali tipi 
a partire da uno \texttt{ScopeContext}. \\

In fase di registrazione di una definizione di funzione, si aggiunge la stessa ad un \texttt{OverloadSet}, 
il quale è una collezione di definizioni di funzioni che condividono lo stesso nome, lo stesso package, lo stesso numero di argomenti e 
parametri formali di tipo. \\

Dato che si può chiamare una funzione anche senza specificare i parametri formali di tipo, è necessario
inserire una funzione generica anche nell'overload set di quello in cui essa sarebbe stata inserita se non fosse 
stata generica. \\

Se la chiamata a funzione specifica il package in cui la funzione è definita, si effettua una ricerca diretta
nell'overload set corrispondente (e.g. \texttt{console::println} specifica il package \texttt{console}). \\

In fase di recupero di una definizione di funzione per il quale il package non è stato esplicitamente specificato, 
si effettua una ricerca nell'overload set corrispondente prima usando il package relativo al file all'interno del 
quale è stata effettuata la chiamata, e poi cercando tra tutti i package importati da tale file.

\vspace{0.5cm}
\begin{lstlisting}[frame=single]
FunctionDefinitionsRegister.retrieve_function_definition(func, types):
    current_package = project_file_structure
        .get_package_containing_file(func.file)
    if type_signature.package != nil:
        key = get_key(current_package, type_signature)
        return search_in_overload_set(key, func, types)
    else:
        imports = project_file_structure
            .get_imports_by_file(type_signature.file)
        for pkg in List.concat(current_package, imports):
            key = get_key(current_package, type_signature)
            func = search_in_overload_set(key, func, types)
            if func != nil:
                return func
    return nil
\end{lstlisting}
\vspace{0.5cm}

L'algoritmo di recupero di una definizione di funzione, sopra schematizzato tratta gli overload set in 
maniera analoga a quanto visto per le definizioni di tipo e analizza un overload set per volta. Solo che 
in questo caso, sarà necessario anche analizzare ogni funzione all'interno di un dato overload set. \\

L'operazione di scelta del miglior candidato, ove possibile, da un dato overload set è detta \textit{overloading resolution}. \\

Dato che tale operazione è \textit{molto} esosa di risorse, si è deciso di implementare una cache per velocizzare
il recupero delle definizioni di funzione. La cache è implementata come una mappa che associa ad una 
chiave di tipo stringa nella forma calcolata a partire dal nome del file in cui si trova una chiamata, il nome 
della funzione e i tipi concreti dei generics e degli argomenti la definzione stessa. \\

\newpage

Durante la ricerca della definizione di una funzione all'interno di un overload set, si tiene traccia dei candidati 
migliori (più specifici) e, una volta terminata la ricerca, se è stato trovato un solo candidato, esso viene restituito. \\

In caso di più candidati viene generato un errore a tempo di compilazione per chiamata ambigua, mentre, nel caso di 
zero candidati, si prosegue la ricerca nell'overload set successivo. \\

Per ogni definizione di funzione dell'overload set si verifica se essa sia compatibile con i tipi concreti 
degli argomenti. Se e solo se la funzione risulta compatibile allora si genera un \texttt{FunctionSpecificityDescriptor} il quale
viene comparato con il \texttt{FunctionSpecificityDescriptor} del miglior candidato trovato finora. \\

In caso di funzione ugualmente specifica, essa viene salvata tra i potenziali candidati. In caso di funzione più specifica, 
essa diventa il nuovo miglior candidato. In caso di funzione meno specifica, essa viene scartata. \\

\vspace{0.5cm}
\begin{lstlisting}[frame=single]
FunctionDefinitionsRegister.search_in_overload_set(key, f, arg_types):
    best_matches = { }
    best_so_far = FunctionSpecificityDescriptor.worst()
    for func_def : overload_sets[key]: 
        current_specificity = FunctionSpecificityDescriptor(func_def)
        if comparison_result.less_specific(): 
            continue
        comparison_result = current_specificity
            .compare_with(best_so_far);
        if !is_compatible(func_def, f, arg_types): 
            continue
        best_maches_so_far.append(func_def);
        if comparison_result.more_specific(): 
            best_maches_so_far = { func_def };
            best_specificity_so_far = current_specificity;
    return extract_single_or_zero_best_match(best_matches);
\end{lstlisting}
\vspace{0.5cm}

Successivamente sarà dedicato l'opportuno spazio alla trattazione approfondita sia del sistema di scoring 
delle funzioni che all'istanziazione delle definizioni generiche (sia di tipi che di funzioni). \\

È opportuno precisare che lo pseudocodice mostrato è semplificato e non riflette esattamente l'implementazione
reale, che invece si occupa anche di reificare le definizioni generiche ed aggiornare la cache mappando 
la definizione trovata con una opportuna chiave per il recupero veloce. \\

Anche il funzionamento del metodo \texttt{is\_compatible} è particolarmente complesso e verrà trattato
in dettaglio in seguito con il dovuto rigore formale. \\