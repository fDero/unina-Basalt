\subsubsection{Costruzione della tabella dei tipi}
La tabella dei tipi è una struttura dati che traccia le definizioni di tutti i tipi
e rende possibile risalire ad esse partendo da un oggetto di tipo \texttt{TypeSignature}. \\

\vspace{0.5cm}
\begin{lstlisting}[frame=single]
class TypeDefinitionsRegister {

    public:
        TypeDefinitionsRegister(ProjectFileStructure&);

        void store_type_definition(TypeDefinition&);
        TypeDefinition retrieve_type_definition(TypeSignature&);

        /* public exposed API */

    protected:
        /* other internal methods */

    private:
        list<string> type_definitions_ids;
        unordered_map<string, TypeDefinition> type_definitions;
        ProjectFileStructure& project_file_structure;
};
\end{lstlisting}
\vspace{0.5cm}

Internamente, la tabella dei tipi è implementata come una mappa che associa ad ogni
nome di tipo, completamente qualificato, la definizione del tipo corrispondente. Inoltre, viene mantenuta una 
lista dei nomi di tipo per consentire l'iterazione con modifica in modo analogo a quanto visto nel paragrafo precedente. \\

Per registrare una definizione di tipo all'interno della tabella, viene generato una stringa 
univoca nella forma \texttt{<package>::<type\_name>::<N>} (dove \texttt{<N>} è il numero di parametri formali di tipo)
e viene usata come chiave nella mappa per l'inserimento della definizione.

In fase di recupero, la tabella dei tipi permette di ottenere la definizione di un tipo a partire da una \texttt{TypeSignature}.
Ciò avviene mediante il recupero diretto della definizione associata alla chiave generata a partire dal nome qualificato del tipo, 
se tale tipo è stato utilizzando specificandone il package di provenienza (e.g. math::Complex). \\

Se il package di provenienza non è stato specificato, si tenterà prima con il package a cui appartiene il file 
in cui è stata trovata la \texttt{TypeSignature} e, in caso di fallimento, si proverà tutti i package 
importati da tale file. \\

Risalire al package in cui si trova il file che contiene la \texttt{TypeSignature}, così come i package da esso importati, 
è un'operazione resa possibile dalla classe \texttt{ProjectFileStructure} approfondita precedentemente. \\


È possibile schematizzare il processo di recupero della definizione di un tipo in pseudocodice come segue
(si assuma che \texttt{get\_key} restituisca la chiave univoca): \\

\vspace{0.5cm}
\begin{lstlisting}[frame=single]
TypeDefinitionsRegister.retrieve_type_definition(type_signature):
    current_package = project_file_structure
        .get_package_containing_file(type_signature.file)
    if type_signature.package != nil:
        key = get_key(current_package, type_signature)
        return type_definitions[key]
    else:
        imports = project_file_structure
            .get_imports_by_file(type_signature.file)
        for pkg in List.concat(current_package, imports):
            key = get_key(current_package, type_signature)
            if type_definitions[key] != nil:
                return type_definitions[key]
    return nil
\end{lstlisting}
\vspace{0.5cm}

In fase di inserimento, si controlla che non vi siano duplicati. In caso di duplicati, viene 
sollevata un'eccezione la quale corrisponderà ad un errore a tempo di compilazione dal punto 
di vista dell'utente. Se il recupero di una definizione di tipo fallisce viene sollevata 
ugualmente un'eccezione. \\

Si noti come il formato scelto per le chiavi consente di indicizzare correttamente tipi 
apparentemente duplicati ma con un numero diverso di parametri formali, come ad esempio 
\texttt{Pair<T>} e \texttt{Pair<T,U>}. \\

Non è possibile distinguere però \texttt{Pair<T,U>} da \texttt{Pair<T,T>}, in quanto il 
numero di parametri formali di tipo è lo stesso. Questo è un limite del sistema di indicizzazione 
scelto, ma è accettabile in quantoin caso contrario si potrebbe incorrere nella complessità 
accessoria di dover gestire i casi ambigui come \texttt{Pair<T,U>} e \texttt{Pair<U,T>}.

Si tenga poi a mente che l'operazione di generazione della chiave è un'operazione esosa di risorse.
Il costo computazionale maggiore deriva dalla ricerca del package corretto iterando su tutti i package
importati dal file in cui è stata trovata la \texttt{TypeSignature} qualora essa non abbia un 
package specificato e/o non sia trovata all'interno del package corrente. \\

L'utilizzo di una cache per velocizzare il recupero delle definizioni di tipo è stato considerato,
ma non è stato implementato in quanto non si è ritenuto necessario. Tuttavia sarebbe possibile 
implementare un sistema di cache per memorizzare come un dato tipo sprovvisto di package esplicitamente 
specificato sia stato risolto partendo da uno specifico file. \\

Tale ottimizzazione favorirebbe codebase con pochi file molto grandi, mentre non sarebbe granchè d'impatto 
su codebase con molti file piccoli, per i quali, anzi, sarebbe sfavorevole. \\



