\subsubsection{Precedenza degli operatori}
Sia che l'AST venga generato da ANTLR, sia che venga generato manualmente, è necessario
che l'AST rispetti la precedenza degli operatori. In fase di costruzione dell'AST,
gli operatori vengono inseriti come nodi dell'albero seguendo il principio di fondo 
che un operando seguito da un operatore, a sua volta seguito da un'espressione, è 
un'espressione. \\ 

Subito dopo l'inserimento dell'espressione come figlio destro del nodo operatore,
è necessario verificare che la precedenza dell'operatore appena inserito sia maggiore
o uguale alla precedenza dell'operatore che si trova nel nodo padre. In caso contrario,
è necessario ruotare l'albero in modo che l'operatore con la precedenza maggiore si
trovi in cima all'albero. \\

\begin{figure}[H]
    \centering
    \begin{tikzpicture}
        \centering
        \node (before) {
            \resizebox{7cm}{5cm}{   
                \begin{tikzpicture}[
                    node distance=1.5cm,
                    elem/.style={
                        circle, 
                        draw=black, 
                        thick, 
                        text centered, 
                        minimum height=2cm,
                        minimum width=2cm
                    },
                    arrow/.style={
                        ->, 
                        thick
                    }
                ]

                    \node[elem] (times) {*};
                    \node[elem, below left=of times] (x) {x};
                    \node[elem, below right=of times] (plus) {+};
                    \node[elem, below left=of plus] (y) {y};
                    \node[elem, below right=of plus] (z) {z};

                    \draw[arrow] (times) -> (x);
                    \draw[arrow] (times) -> (plus);
                    \draw[arrow] (plus) -> (y);
                    \draw[arrow] (plus) -> (z);
                \end{tikzpicture}
            }
        };  

    
        \node[right of=before, xshift=7cm] (after) { 
            \resizebox{7cm}{5cm}{   
                \begin{tikzpicture}[
                    node distance=1.5cm,
                    elem/.style={
                        circle, 
                        draw=black, 
                        thick, 
                        text centered, 
                        minimum height=2cm,
                        minimum width=2cm
                    },
                    arrow/.style={
                        ->, 
                        thick
                    }
                ]

                    \node[elem] (plus) {+};
                    \node[elem, below left=of times] (times) {*};
                    \node[elem, below left=of times] (x) {x};
                    \node[elem, below right=of times] (y) {y};
                    \node[elem, below right=of plus] (z) {z};

                    \draw[arrow] (times) -> (x);
                    \draw[arrow] (times) -> (y);
                    \draw[arrow] (plus) -> (times);
                    \draw[arrow] (plus) -> (z);
                \end{tikzpicture}
            }
        };  

        \draw[->] (before) -- (after);
    \end{tikzpicture}
    \caption{
        \centering
        Esempio di generico AST formato da operatori 
        binari prima e dopo una rotazione
    }
\end{figure}

Una situazione simile si verifica anche per gli operatori unari, come ad esempio l'operatore 
di negazione logica (!). Di seguito è riportato uno scenario di rotazione riguardante operatori 
logici binari ed unari. \\

\begin{figure}[H]
    \centering
    \begin{tikzpicture}
        \centering
        \node (before) {
            \resizebox{6cm}{5.5cm}{   
                \begin{tikzpicture}[
                    node distance=1.5cm,
                    elem/.style={
                        circle, 
                        draw=black, 
                        thick, 
                        text centered, 
                        minimum height=2cm,
                        minimum width=2cm
                    },
                    arrow/.style={
                        ->, 
                        thick
                    }
                ]

                    \node[elem] (minus) {!};
                    \node[elem, below right=of minus] (plus) {\&\&};
                    \node[elem, below left=of plus] (a) {a};
                    \node[elem, below right=of plus] (b) {b};

                    \draw[arrow] (minus) -> (plus);
                    \draw[arrow] (plus) -> (a);
                    \draw[arrow] (plus) -> (b);
                \end{tikzpicture}
            }
        };  

    
        \node[right of=before, xshift=7cm] (after) { 
            \resizebox{6cm}{5.5cm}{   
                \begin{tikzpicture}[
                    node distance=1.5cm,
                    elem/.style={
                        circle, 
                        draw=black, 
                        thick, 
                        text centered, 
                        minimum height=2cm,
                        minimum width=2cm
                    },
                    arrow/.style={
                        ->, 
                        thick
                    }
                ]

                    \node[elem] (plus) {\&\&};
                    \node[elem, below left=of plus] (minus) {!};
                    \node[elem, below right=of minus] (a) {a};
                    \node[elem, below right=of plus] (b) {b};

                    \draw[arrow] (plus) -> (minus);
                    \draw[arrow] (plus) -> (b);
                    \draw[arrow] (minus) -> (a);
                \end{tikzpicture}
            }
        };  

        \draw[->] (before) -- (after);
    \end{tikzpicture}
    \caption{
        \centering
        Esempio di generico AST formato da operatori 
        unari e binari prima e dopo una rotazione
    }
\end{figure}