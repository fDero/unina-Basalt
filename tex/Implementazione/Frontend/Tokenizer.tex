\subsubsection{Tokenizzazione nella repository principale}
Nella repository principale il tokenizer
è stato realizzato su misura, senza l'uso di alcuna
dipendenza esterna. \\

La classe \texttt{Tokenizer} espone due costruttori, 
uno per la costruzione a partire da un file ed uno 
per la costruzione a partire da una stringa 
, utilizzato durante il testing (in C++ è possibile 
trattare enteambi i casi in modo molto simile grazie
alla classe \texttt{std::istream} che astrae sui 
dettagli implementativi sottostanti).

Il metodo \texttt{tokenize()} della classe tokenizer 
si occupa di effettuare la tokenizzazione dell'intero sorgente. Esso 
Segue il seguente algoritmo (espresso in pseudocodice):

\vspace{0.5cm}
\begin{lstlisting}[frame=single]
Tokenizer.tokenize():
    tokens = { }
    for line in source_code:
        char_index = 0
        while char_index < length(line):
            token, success = extract(line, &char_index)
            if (success): 
                tokens.append(token)
    ensure_multiline_comments_closed()
    return tokens
\end{lstlisting}
\vspace{0.5cm}

Il metodo \texttt{tokenize()} si occupa di estrarre un singolo token 
per volta dal sorgente, esso è concretamente implementato come un dispatch 
sui vari metodi specifici per estrarre uno specifico tipo di token.

\vspace{0.5cm}
\begin{lstlisting}[frame=single]
Tokenizer.extract(line, char_index)
    handle_multiline_comments(line, &char_index)
    handle_simple_comments(line, &char_index)
    first_char = line[char_index]
    tok = null
    switch(first_char):
        case '"' "'" '`': tok = extract_string(line, &char_index)
        case '0' ... '9': tok = extract_number(line, &char_index)
        case 'a' ... 'z': tok = extract_identifier(line, &char_index)
        case 'A' ... 'Z': tok = extract_typename(line, &char_index)
        case '+' ... '=': tok = extract_symbol(line, &char_index)
    return tok
\end{lstlisting}    
\vspace{0.5cm}

I singoli metodi \texttt{extract\_string}, \texttt{extract\_number}, 
\texttt{extract\_identifier}, \texttt{extract\_typename}, 
\texttt{extract\_symbol}, verificano se il testo della riga corrente, 
partendo dalla posizione \texttt{char\_index}, coincida con quanto 
atteso. In caso di successo, essi restituiscono il token corrispondente 
alla porzione di testo che è stata riconosciuta e si occupano di 
avanzare l'inidce fino al primo carattere non riconosciuto. 

\newpage

Il tokenizer mantiene traccia dei commenti multi-riga mediante uno 
stack. Esso viene popolato con i token relativi alle aperture dei 
commenti multi-riga (tali token sono salvati solo temporaneamente ai 
fini di error checking). \\

Ad ogni apertura di un commento multiriga, esso viene tokenizzato 
e salvato nello stack, mentre ad ogni chiusura di un commento multi-riga, 
si effettua un pop dallo stack. \\


\vspace{0.5cm}
\begin{lstlisting}[frame=single]
Tokenizer.handle_multiline_comments(line, char_index):
    maybe_open_comment = line.substring(char_index, 2)
    if (maybe_open_comment == "/*"):
        comments_stack.push(make_token(line, char_index))
        char_index = char_index + 2
    while comments_stack.size() > 0 AND char_index < length(line):
        maybe_comment_seq = line.substring(char_index, 2)
        switch(maybe_comment_seq):
            case "/*": 
                comments_stack.push(make_token(line, char_index))
                char_index = char_index + 2
            case "*/:
                comments_stack.pop()
                char_index = char_index + 2
            default:
                char_index = char_index + 1
\end{lstlisting}    
\vspace{0.5cm}

Se al termine della tokenizzazione lo stack non dovesse essere vuoto, 
ciò significherebbe che almeno un commento multi-riga non è stato chiuso, 
e i token che sono stati salvati nello stack potranno essere usati 
per fornire errori significativi in console. \\

Per quanto riguarda la gestione dei commenti su singola riga, essi 
sono gestiti in modo molto semplice. Qualora si incontri la sequenza
di caratteri corrispondente all'apertura di un commento a singola riga, 
l'indice corrispondente al carattere corrente viene aggiornato 
in modo da far sembrare che la riga sia stata già tokenizzata fino al termine.


\vspace{0.5cm}
\begin{lstlisting}[frame=single]
Tokenizer.handle_simple_comments(line, &char_index):
    maybe_open_comment = line.substring(char_index, 2)
    if (maybe_open_comment == "//"):
        char_index = length(line)
\end{lstlisting}    
\vspace{0.5cm}