\subsubsection{Analisi dei legami post-assegnamento}
Assegnando un espressione detta \textit{sorgente} ad un'altra espressione detta \textit{destinazione}
è opportuno chiedersi se, manipolando la destinazione, non si possa alterare in qualche modo la sorgente. \\

Ad esempio, nel caso in cui sia sorgente che destinazione fossero puntatori, al termine dell'assegnamento, 
manipolando il puntatore destinazione si potrebbe accedere in scrittura all'area di memoria puntata 
dal puntatore sorgente. \\

In generale, ci si deve porre il problema di quando assegnare un'espressione sorgente espone 
tale espressione alla possibilità di essere indirettamente modificata, del tutto o in parte, 
mediante azioni di qualsiasi tipo compiute sulla destinazione. \\

La risposta a tale domanda è fornita dalla classe \texttt{BondInspector} e dall'algoritmo che 
essa implementa sottoforma di metodi mutuamente ricorsivi. Tale classe infatti implementa un 
metodo chiamato \texttt{does\_the\_type\_of\_this\_expr\_imply\_a\_bond} che, visitando 
in profondità (DFS) l'AST relativo ad una data typesignature, restituisce \texttt{true} se 
assegnare espressioni aventi tale tipo può portare a scenari di modifica indiretta. \\

\vspace{0.5cm}
\begin{lstlisting}[frame=single]
BondInspector.does_the_type_of_this_expr_imply_a_bond(type):
    switch type:
        case InlineUnion:   
            return does_this_inline_union_imply_a_bond(type)
        case CustomType:    
            return does_this_custom_type_imply_a_bond(type)
        case ArrayType:     
            return does_the_type_of_this_expr_imply_a_bond(
                type.stored_type
            )
        case PrimitiveType: return false
        case PointerType:   return true
        case SliceType:     return true
        case TemplateType:  return true
    
BondInspector.does_this_union_imply_a_bond(union):
    for alternative in union.alternatives:
        does_the_type_of_this_expr_imply_a_bond(alternative):
            return true
    return false
\end{lstlisting}
\vspace{0.5cm}

\newpage
