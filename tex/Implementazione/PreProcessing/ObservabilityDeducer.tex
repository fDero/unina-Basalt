\subsubsection{Analisi dell'osservabilità post-assegnamento}
Assegnando un espressione detta \textit{sorgente} ad un'altra espressione detta \textit{destinazione}
è opportuno chiedersi se sia realmente possibile osservare le modifiche apportate. \\

Ad esempio, nel caso in cui la destinazione fosse il valore restituito da una chiamata a funzione, 
non sarebbe possibile osservare alcuna modifica apportata alla destinazione, in quanto di essa non 
rimane più alcun riferimento ad assignment completato. \\

In generale, un assignment ad una espressione detta \textit{destinazione} è
considerato osservabile o non-osservabile in base in base a che tipologia di 
espressione sia la destinazione. \\

\vspace{0.5cm}
\begin{itemize}
    \item \textbf{Operatori binari}: la loro modifica non è osservabile
    \item \textbf{Operatori unari (matematici)}: la loro modifica non è osservabile
    \item \textbf{Operatori unari (logici)}: la loro modifica non è osservabile
    \item \textbf{Operatori unari (su puntatori)}: la loro modifica è sempre osservabile
    \item \textbf{Cast}: osservabile se è osservabile ciò che si vuole castare
    \item \textbf{Accessi a membro}: la modifica è osservabile se lo sarebbe una modifica 
    all'espressione a cui si accede oppure se l'espressione a cui si accede è un puntatore.
    \item \textbf{Accessi a cella}: la modifica è osservabile se lo sarebbe una modifica 
    all'espressione a cui si accede oppure se l'espressione a cui si accede è una slice.
    \item \textbf{Chiamate a funzioni}: la loro modifica non è osservabile
    \item \textbf{Identificatori}: la loro modifica è sempre osservabile
    \item \textbf{Literal}: la loro modifica non è osservabile
\end{itemize}
\vspace{0.5cm}

A partire dalle seguenti regole, la classe \texttt{ObservabilityDeducer} implementa un metodo
chiamato \texttt{is\_expression\_update\_observable} che, visitando in profondità (DFS) l'AST relativo
ad una data espressione, restituisce \texttt{true} se una ipotetica modifica a tale 
espressione sarebbe osservabile, \texttt{false} in caso contrario. \\

\newpage

Si considerino ad esempio le seguenti definizioni (saranno fornite solo le firme 
delle funzioni per facilitare la lettura): \\

\vspace{0.5cm}
\begin{lstlisting}[frame=single]
struct S {
    ptr : #S;
    val : Int;
}

func f() -> S { 
    /* ... */ 
}

func g() -> #S { 
    /* ... */ 
}

var s : S = /* ... */;
\end{lstlisting}
\vspace{0.5cm}

Date le definizioni appena fornite, si deducono i seguenti verdetti di 
osservabilità: \\

\vspace{0.5cm}
\begin{lstlisting}[frame=single]
f() = s; 
// Non-osservabile

f().val = 10; 
// Non-osservabile

f().ptr.val = 10; 
// Osservabile

g() = &s; 
// Non-osservabile

g().val = 10; 
// Osservabile

g().ptr.val = 10; 
// Osservabile

#g() = s; 
// Osservabile
\end{lstlisting}
\vspace{0.5cm}

\newpage