\subsubsection{Typechecking degli assegnamenti}
Verificare che espressioni di tipo \texttt{T} siano assegnabili ad espressioni di tipo \texttt{U} è
responsabilità della classe \texttt{AssignmentTypeChecker}. Tale classe espone una API estremamente minimale, ma
al contempo fondamentale, composta di due soli metodi. \\

\vspace{0.5cm}
\begin{lstlisting}[frame=single]
class AssignmentTypeChecker {

    public:
        AssignmentTypeChecker(
            TypeDefinitionsRegister& program_representation, 
            ProjectFileStructure& project_file_structure
        );

        bool validate_assignment(
            const TypeSignature& source, 
            const TypeSignature& dest,
            bool is_strict_mode
        );
        
        GenericSubstitutionRule::Set&
            get_generic_substitution_rules(); 

    private:
        /* other methods */
        
        TypeDefinitionsRegister& type_definitions_register;
        ProjectFileStructure& project_file_structure;
        set<string> strict_type_inference_deductions;       
        GenericSubstitutionRule::Set generic_substitution_rules;
};
\end{lstlisting}
\vspace{0.5cm}

Il metodo \texttt{validate\_assignment} è il metodo principale della classe, che si occupa di verificare
che un'espressione di tipo \texttt{T} sia assegnabile ad un'espressione di tipo \texttt{U}. Esso prende 
tre argomenti, due dei quali sono le \texttt{TypeSignature} prese in esame mentre il terzo è una flag booleana 
che indica se il controllo debba essere effettuato in modalità \textit{strict} (stringente, conservativa) o meno. \\

Tale metodo restituisce un valore booleano che indica se l'assegnamento è valido o meno. In caso l'assegnamento 
non possa essere controllato in quanto i tipi presi in esame sono generici e non ancora reificati, sarà ugualmente 
restituito \texttt{true}. \\

\newpage

Presupposto che \texttt{TypeSignature} è supertipo di tutte le classi che modellano una type-signature nell'AST, e presupposto che 
quindi è necessario scoprire a tempo di esecuzione quale sia il tipo concreto degli oggetti il cui tipo dichiarato è \texttt{TypeSignature},
allora è chiaro come la chiamata al metodo \texttt{validate\_assignment} debba ricorsivamente visitare l'AST delle due \texttt{TypeSignature},
mediante continue chiamate ricorsive a metodi interni della classe, per poi arrivare ad una ultima chiamata ad un metodo specifico per 
il caso concreto che si è verificato. \\

Alla luce di quanto detto è chiaro come la classe \texttt{AssignmentTypeChecker} abbia moltissimi metodi interni (privati), anche se 
ha solo due metodi pubblici. Ognuno di questi metodi interni implementa un grafo di chiamate altamente ricorsivo dove ad ogni chiamata 
si ci avvina allo scoprire i tipi concreti degli oggetti il cui tipo dichiarato è \texttt{TypeSignature}. \\

\begin{lstlisting}[frame=single]
AssignmentTypeChecker.validate_assignment(source, dest, mode):
   switch dest:
      case TemplateType:  return validate_assignment_to_template(...)
      case CustomType:    return validate_assignment_to_custom(...)
      case PointerType:   return validate_assignment_to_pointer(...)
      case ArrayType:     return validate_assignment_to_array(...)
      case SliceType:     return validate_assignment_to_slice(...)
      case InlineUnion:   return validate_assignment_to_union(...)
      case PrimitiveType: return validate_assignment_to_primitive(...)

AssignmentTypeChecker.validate_assignment_to_pointer(source, ptr):
   if source.is_pointer():
       return validate_assignment(
          source.pointed_type(), 
          ptr.pointed_type(), 
          STRICT_MODE
       )
   else:
      return false
\end{lstlisting}
\vspace{0.5cm}

Nel precedente frammento di codice si intuisce l'utilizzo e lo scopo della flag relativa alla 
strict-mode. Tale flag è necessaria per segnalare che non si desidera restituire \texttt{true}
qualora l'assegnamento comporti una conversione implicita di tipo. \\

Ad esempio, è possibile assegnare \texttt{Int} ad \texttt{Int | Float} ma non è 
possibile assegnare \texttt{List<Int>} a \texttt{List<Int|Float>} e nemmeno 
\texttt{\#Int} a \texttt{\#(Int|Float)}. La differenza fra questi scenari è che le chiamate al 
metodo \texttt{validate\_assignment} fatte nei contesti appena illustrati vengono fatte in \textit{strict-mode}. \\

\newpage

Si consideri ad esempio il metodo \texttt{validate\_assignment\_to\_union}, che si occupa di verificare
se un'espressione di tipo \texttt{T} sia assegnabile ad un'espressione di tipo \texttt{U} dove \texttt{U}
è un \texttt{InlineUnion} oppure un \texttt{CustomType} corrispondente ad una \textit{named-union}: \\

\vspace{0.5cm}
\begin{lstlisting}[frame=single]
AssignmentTypeChecker.validate_assignment_to_union(src, dest, mode):
    if mode == STRICT_MODE:
        return source.is_equal_to(dest)
    if src.is_equal_to(dest):
        return true
    for type in dest.alternatives():
        if validate_assignment(src, type, mode):
            return true
    for type in src.alternatives():
        if !validate_assignment(type, dest, mode):
            return false
    return true
\end{lstlisting}
\vspace{0.5cm}

Risulta evidente analizzando lo pseudocodice sopra riportato come il controllo di assegnabilità
restituisca \texttt{true} per assegnamenti di tipo \texttt{Int} a \texttt{Int|Float} in modalità \textit{non-strict}, 
e invece restituisca \texttt{false} in modalità \textit{strict}. \\

In generale, valgono le seguenti affermazioni:

\begin{itemize}
    \item \texttt{\#T} è assegnabile a \texttt{\#U} se e solo se \texttt{T} è assegnabile a \texttt{U} in modalità \textit{strict}
    \item \texttt{\$T} è assegnabile a \texttt{\$U} se e solo se \texttt{T} è assegnabile a \texttt{U} in modalità \textit{strict}
    \item \texttt{A<T>} è assegnabile ad \texttt{A<U>} se e solo se \texttt{T} è assegnabile a \texttt{U} in modalità \textit{strict}
    \item \texttt{[]T} è assegnabile in modalità \textit{non-strict} ad \texttt{[]U} se e solo se \texttt{T} è assegnabile a \texttt{U}
    \item \texttt{[]T} è assegnabile in modalità \textit{strict} ad \texttt{[]U} se e solo se \texttt{T} = \texttt{U}
    \item \texttt{String} è assegnabile in modalità \textit{non-strict} ad \texttt{RawString}
    \item \texttt{A} è assegnabile in modalità \textit{strict} a \texttt{B} se e solo se \texttt{A} e \texttt{B} sono identici
    \item la prima chiamata a \texttt{validate\_assignment} è fatta in modalità \textit{non-strict}
\end{itemize}

Valgono inoltre le regole di assegnabilità già trattate nella sezione dedicata al design del linguaggio al 
capitolo dedicato all'assegnabilità, in particolare, come confermato dal frammento di codice appena discusso, 
è possibile assegnare espressioni di tipo \texttt{SubType} ad espressioni di tipo \texttt{SuperType} in modalità 
\textit{non-strict} se: \\

\begin{itemize}
    \item \texttt{SubType} è identico a \texttt{SuperType}
    \item \texttt{SuperType} è una union e \texttt{SubType} è uno dei tipi che essa descrive
    \item \texttt{SubType} è una union e tutti i tipi da essa descritti sono assegnabili a \texttt{SuperType}    
\end{itemize}