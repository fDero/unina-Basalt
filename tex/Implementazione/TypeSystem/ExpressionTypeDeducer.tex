\subsubsection{Deduzione del tipo di una espressione}
La deduzione del tipo di un'espressione è responsabilità della classe \texttt{ExpressionTypeDeducer}. 
Di essa si crea un'istanza per ogni scope all'interno del quale si desidera analizzare le espressioni. \\

\vspace{0.5cm}
\begin{lstlisting}[frame=single]
class ExpressionTypeDeducer {
    
    public:
        ExpressionTypeDeducer(
            TypeDefinitionsRegister&,
            FunctionDefinitionsRegister&,
            CommonFeatureAdoptionPlanGenerationEngine&,
            ProjectFileStructure&,
            ScopeContext&
        );

        std::optional<TypeSignature> deduce_expression_type(
            Expression& expression
        );

        std::vector<TypeSignature> deduce_arguments_types(
            FunctionCall& function_call
        );

        /* other methods */

    private:
        TypeDefinitionsRegister& type_definitions_register;
        FunctionDefinitionsRegister& function_definitions_register;
        CFAPlanGenerationEngine& cfa_plan_generation_engine;
        ProjectFileStructure& project_file_structure;
        ScopeContext& scope_context;
};
\end{lstlisting}
\vspace{0.5cm}

Il metodo \texttt{deduce\_expression\_type} è il metodo principale della classe, che si occupa di dedurre 
il tipo di un'espressione. Il risultato può essere \texttt{std::nullopt} (null) nel caso in cui la 
deduzione non sia possibile (e.g. dipendente da un tipo generico non ancora reificato). \\

Il metodo \texttt{deduce\_arguments\_types} è un metodo di supporto che si occupa di dedurre i tipi
degli argomenti di una chiamata di funzione, al fine di poter effettuare l'overloading resolution. Nel caso 
in cui numero di tipi dedotti non coincida col numero di argomenti della funzione, ciò vuol dire che la deduzione 
di uno o più tipi non è stata possibile (e.g. dipendente da un tipo generico non ancora reificato). \\

\newpage

La deduzione del tipo di un'espressione avviene mediante visita ricorsiva dell'AST dell'espressione stessa,
attraverso una visita in profondità (DFS). Durante la visita si applicano le seguenti regole:

\vspace{1cm}
\begin{itemize}
    \item \textbf{Literal}: il tipo di un letterale è noto a priori, e viene restituito direttamente.
    \item \textbf{Identifier}: il tipo di un identificatore è conservato all'interno dello \texttt{ScopeContext}.
    \item \textbf{Operatore unario}: \begin{itemize}
        \item Se l'operatore è la negazione logica, il tipo dell'operatore è \texttt{Bool}
        \item Se l'operatore è la dereferenziazione, si deduce che l'espressione è del tipo puntato dall'operando
        \item Se l'operatore è la referenziazione (\texttt{\&}), si deduce che l'espressione è un puntatore al tipo dell'operando
        \item Negli altri casi, l'espressione è del tipo dell'operando
    \end{itemize}
    \item \textbf{Operatore binario}: \begin{itemize}
        \item Se l'operatore è un operatore aritmetico, si deduce il tipo dell'operando
        \item Se l'operatore è un operatore di confronto, si deduce il tipo \texttt{Bool}
        \item Se l'operatore è un operatore logico, si deduce il tipo \texttt{Bool}
        \item Se l'operatore è l'operatore \texttt{is}, si deduce il tipo \texttt{Bool}
        \item Se l'operatore è un cast, si deduce il tipo a cui si sta castando
    \end{itemize}
    \item \textbf{Chiamata di funzione}: Si deduce il tipo di ritorno, previa overloading resolution
    \item \textbf{Accesso a membro}: \begin{itemize}
        \item Se l'operando non è un puntatore, si deduce il tipo del membro richiesto
        \item Se l'operando è un puntatore, si simula la sua dereferenziazione
        \item Se l'operando è array/stringa/slice, si deduce il tipo \texttt{Int} per l'accesso a \texttt{.len}
    \end{itemize}
    \item \textbf{Accesso a cella di array/slice}: Si deduce il tipo del generico elemento
\end{itemize}

\newpage