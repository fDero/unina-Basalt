\subsubsection{Algoritmo di Type-inference}
Con \textit{type-inference} si intende il processo che porta alla deduzione di uno o più 
tipi a partire dal contesto. In Basalt, l'unico scenario in cui si applica la type-inference è
la deduzione dei parametri attuali di tipo di una chiamata a funzione. \\

Come detto in precedenza, durante la fase di \textit{overloading-resolution}, il 
compilatore deve scegliere quale tra le funzioni contenute in un overload set è 
quella più specifica tra quelle \textit{compatibili con la chiamata}. \\

Solo per le chiamate a funzione che non specificano i parametri attuali di tipo, ma che non 
sono compatibili con nessuna funzione non-generica, si tenta di dedurre i parametri attuali
di tipo con la type-inference. \\

Durante l'atto di controllare che una definizione di funzione sia compatibile con una chiamata,
viene controllata l'assegnabilità di tutti i tipi concreti degli argomenti rispetto ai tipi 
dichiarati dei parametri formali. Durante la validazione dell'assegnabilità, se il tipo a cui si desidera
assegnare è generico, come \textit{by-product} del controllo di assegnabilità, vengono dedotti i vincoli di tipo
che legano quel tipo generico con il tipo concreto. \\

\vspace{0.5cm}
\begin{lstlisting}[frame=single]
FunctionDefinitionsRegister.check_compatiblity(fdef, fcall, types):
   checker = new AssignmentTypeChecker(...)
   for actual, expected in arg_types(fdef, fcall):
      if !checker.validate_assignment(actual, expected, !STRICT_MODE):
         return (false, null)
   engine = new GenericsInstantiationEngine(
      checker.get_generic_substitution_rules() // inferred
   )
   instantiated = engine.instantiate(fdef, ...)
   return (true, instantiated)
\end{lstlisting}
\vspace{0.5cm}

Se la compatibilità è verificata, si restituisce non solo \texttt{true}, ma anche la definizione 
di funzione reificata grazie ai parametri attuali di tipo dedotti. Nel caso in cui i parametri attuali di 
tipo siano stati esplicitamente specificati, essi saranno passati al costruttore di \texttt{AssignmentTypeChecker}
il quale li restituirà con il metodo \texttt{get\_generic\_substitution\_rules} e l'algoritmo resterà invariato. \\

L'istanza di \texttt{AssignmentTypeChecker} in uso è specifica per la coppia definizione-chiamata in esame,
essa tiene traccia internamente delle assunzioni che essa stessa effettua durante il type-checking e 
le restituisce al chiamante al termine. Tali assunzioni sono specifiche per l'istanza e quindi non hanno 
effetto sul controllo di compatibilità delle future definizioni. \\ 

\newpage

I vincoli di assegnabilità che legano un parametro formale di tipo ad un tipo concreto sono dedotte 
in sede di controllo di validità dell'assegnabilità del tipo concreto al tipo generico. \\

\vspace{0.5cm}
\begin{lstlisting}[frame=single]
AssignmentTypeChecker.validate_assignment_to_template(src, dst, mode):
    already_locked = false
    if strict_type_inference_deductions.contains(dst.type_name):
        already_locked = true
    if mode == STRICT_MODE:
        strict_type_inference_deductions.insert(dst.type_name);

    for rule in generic_substitution_rules:
        if rule.to_be_replaced != dst.type_name:
            continue
        if validate_assignment(src, rule.replacement, STRICT_MODE):   
            return true;
        if validate_assignment(src, rule.replacement, !STRICT_MODE):   
            return !already_locked
        if validate_assignment(rule.replacement, src, !STRICT_MODE):
            rule.replacement = src
            return !already_locked
        if rule.replacement.is_inline_union():
            rule.replacement.alternatives.append(src);
            return !already_locked 
        rule.replacement = new InlineUnionType(
            rule.replacement, src
        );
        return !already_locked

    generic_substitution_rules.insert(dst.type_name, src)
    return !already_locked
\end{lstlisting}
\vspace{0.5cm}

In particolare, è possibile schematizzare ulteriormente l'algoritmo come segue:

\begin{itemize}
    \item Se esiste già un plausibile candidato per il tipo generico, ed è possibile 
    assegnare il tipo concreto a tale candidato in modalità \textit{strict}, allora il candidato viene 
    lasciato intatto e si restituirà \texttt{true} sia che ci si trovi in modalità 
    \textit{strict} che \textit{non-strict}.

    \item Se non esiste un plausibile candidato per il tipo generico, il tipo che si desidera 
    assegnare diventa esso stesso un candidato. Esso sarà scelto se e solo se l'assunzione non 
    sarà affinata in seguito all'analisi dei rimanenti argomenti. In caso di \textit{strict-mode},
    l'assunzione sarà marcata come non-modificabile.

    \item In tutti gli altri casi si richiede di non essere in strict-mode, e si può mettere 
    in discussione l'assunzione precedente rimpiazzandola con il tipo che si desidera assegnare, 
    qualora tale modifica preservi i vincoli dedotti in precedenza, oppure con la union anonima 
    del tipo concreto che si desidera assegnare con il precedente candidato.
\end{itemize}

Un dubbio che è lecito porsi è il seguente: È giusto che anche qualora in \textit{strict-mode}, vi sia 
un modo modificare l'assunzione precedente basandosi sull'esito di un controllo di assegnabilità che 
non è stato condotto in \textit{strict-mode}? \\

Si consideri la seguente definizione di funzione generica:

\vspace{0.5cm}
\begin{lstlisting}[frame=single]
func contains<T>(target : T, list : List<T>) -> Bool {
    cursor : #Node = list.head;
    while (cursor != null) {
        if (utils::compare_eq(cursor.value, target)) {
            return true;
        }
        cursor = cursor.next;
    }
    return false;
}
\end{lstlisting}
\vspace{0.5cm}

Deve essere possibile chiamare la funzione \texttt{contains} con argomenti \texttt{Int} e \texttt{List<Int|Float>}
affidandosi alla type-inference per dedurre il tipo di \texttt{T}. Al momento in cui si analizza l'assegnamento 
di \texttt{List<Int|Float>} a \texttt{List<T>} in modalità \textit{non-strict}, il quale comporta il controllo 
dell'assegnamento di \texttt{Int|Float} a \texttt{T} in modalità \textit{strict}, il tipo \texttt{T} avrà
già un possibile candiato ovvero \texttt{Int}. \\

Tale assunzione è modificabile, ed è quindi giusto che venga modificata anche in \textit{strict-mode}. Questo 
significa che la \textit{strict-mode} impedisce l'assegnazione di un tipo ad un altro se questo avviene mediante 
conversione implicita di tipo, ma solo se il tipo destinazione non è generico. \\

Per tipi di destinazione generici, la \textit{strict-mode} impedisce solo la modifica futura dell'assunzione 
che si sta attualmente facendo. \\

La modifica dell'assunzione precedente è possibile se essa è modificabile e se il nuovo candidato 
non comporta la perdita di vincoli di tipo dedotti in precedenza. La non-perdita dei vincoli è garantita
dal fatto che è stato già verificato che il nuovo candidato possa ricevere assegnamenti da parte del candidato 
precedente a patto di supportare le conversioni implicite di tipo, ciò significa che dunque è assolutamente corretto 
che tale controllo non avvenga in \textit{strict-mode}. \\

Si ricordi che l'assegnabilità di un tipo ad un altro è una relazione transitiva, e quindi se è possibile assegnare 
ad un nuovo candidato il precedente, allora ciò implica che è possibile assegnarvi tutti i candidati che sono 
stati dedotti fin ora e che sono poi stati scartati. Questa semplice osservazione garantisce la correttezza 
dell'algoritmo. \\