\subsubsection{Conversioni implicite ed operatori di tipo}
In sede di chiamata a funzione, di return da una funzione e di assegnamento è spesso necessario applicare conversioni 
implicite di tipo. Ad esempio, assegnando un \texttt{Int} ad un \texttt{Int|Float}, esso deve venire prima convertito, 
ovvero deve essere creata una union implicitamente a partire dal valore intero, e solo dopo tale union potrà essere assegnata. \\

Assegnando un puntatore ad array ad una slice, assegnando una slice ad una stringa e così via, è necessario applicare
conversioni di tipo implicite, e ciò è responsabilità della classe \texttt{TypeManipulationsLLVMTranslator}. \\

\vspace{0.5cm}
\begin{lstlisting}[frame=single]
class TypeManipulationsLLVMTranslator {

    public:
        TypeManipulationsLLVMTranslator(
            ProgramRepresentation& program_representation, 
            TypesLLVMTranslator& types_llvm_translator
        );

        TranslatedExpression test_concrete_type_of_union_in_llvm(
            llvm::BasicBlock* block,
            TranslatedExpression union_expression,
            const TypeSignature& type_to_check
        );

        TranslatedExpression cast_translated_expression_as_copy(
            llvm::BasicBlock* block,
            TranslatedExpression expression,
            const TypeSignature& expression_original_type,
            const TypeSignature& dest_type
        );

        TranslatedExpression cast_translated_expression_as_ref(
            llvm::BasicBlock* block,
            TranslatedExpression expression,
            const TypeSignature& expression_original_type,
            const TypeSignature& dest_type
        );

    private:
        /* other fields */
        /* other methods */
};
\end{lstlisting}
\vspace{0.5cm}

\newpage

Nei casi di conversione implicita in sede di return, di assegnamento o di un parametro attuale al tipo di un parametro formale 
di una funzione in sede di chiamata la conversione porta ad una copia, mentre l'utilizzo dell'operatore \texttt{as} converte per riferimento. \\

In entrambi i casi, l'approccio scelto è stato quello di generare una \texttt{CastStrategy} basandosi sul tipo attuale dell'espressione da 
convertire e sul tipo di destinazione. Questa strategia è poi utilizzata per effettuare un dispatch e chiamare un metodo opportuno per la gestione 
del caso concreto. \\

\vspace{0.5cm}
\begin{lstlisting}[frame=single]
class TypeManipulationsLLVMTranslator {

    public:
        /* other methods */
        
        CastStrategy compute_cast_strategy(
            TypeSignature,
            TypeSignature
        );

    private:
        enum class CastStrategy {
            noop,
            union_to_union,
            union_to_alternative,
            alternative_to_union,
            array_pointer_to_slice,
            array_pointer_to_string,
            array_pointer_to_raw_string,
            slice_to_string,
            slice_to_raw_string,
            string_to_raw_string,
            array_to_array,
        };

        /* other fields */
        /* other methods */
};
\end{lstlisting}
\vspace{0.5cm}

I due metodi pubblici di cui sopra \texttt{cast\_translated\_expression\_as\_copy} e \texttt{cast\_translated\_expression\_as\_ref} si occupano di
effettuare un dispatch basandosi sulla strategia di cast calcolata, e di chiamare il metodo (privato) opportuno per la gestione del caso concreto
tenendo conto del fatto che serve effettuare una conversione rispettivamente o per copia o per riferimento. \\