\subsubsection{Traduzione degli overload CFA in LLVM}
La traduzione in LLVM-IR degli overload CFA è responsabilità della classe 
\texttt{CallableCodeBlocksLLVMTranslator} così come già detto in precedenza. La 
traduzione inizia dalla creazione di un \textit{entry-block} all'interno della 
funzione LLVM-IR che si sta creando, per poi proseguire ricorsivamente traducendo 
all'interno del blocco corrente (che all'inizio è proprio l'entry-block) l'albero 
decisionale codificato dal \texttt{CFAPlanDescriptor}. \\

\vspace{0.5cm}
\begin{lstlisting}[frame=single]   
Translator.translate_cfa_descriptor_to_llvm(
    CFAPlanDescriptor descriptor, 
    llvm_function
):
    entry_block = llvm_create_block("entry", llvm_function)
    translate_cfa_plan_to_llvm(
        descriptor, 
        descriptor.plan, 
        llvm_function, 
        entry_block
    )
    return llvm_function
\end{lstlisting}
\vspace{0.5cm}

\vspace{0.5cm}
\begin{lstlisting}[frame=single]   
Translator.translate_cfa_plan_to_llvm(
    CFAPlanDescriptor cfa_plan_descriptor, 
    CFAPlan cfa_plan,
    llvm_function,
    current_block
):
    if (cfa_plan.is_direct_adoption()):
        selected_concrete_function = cfa_plan.get_direct_adoption()
        translate_cfa_direct_adoption_to_llvm(
            cfa_plan_descriptor, 
            selected_concrete_function, 
            llvm_function, 
            current_block
        )
    else:
        recursive_plan = cfa_plan.get_recursive_adoption()
        translate_cfa_recursive_adoption_to_llvm(
            cfa_plan_descriptor, 
            recursive_plan, 
            llvm_function, 
            current_block
        )
\end{lstlisting}
\vspace{0.5cm}

\newpage

La traduzione di una \texttt{FunctionDefinition} sotto forma di caso base 
dell'albero decisionale non è in alcun modo difforme da una normale chiamata 
a funzione effettuata all'interno del blocco corrente. \\

La traduzione invece di un \texttt{RecursiveCFAPlan} è più complessa, in quanto
richiede la creazione di un nuovo blocco per ogni figlio del nodo dell'albero 
decisionale che esso codifica. \\

\vspace{0.4cm}
\begin{lstlisting}[frame=single]
Translator.translate_cfa_recursive_adoption_to_llvm(
    CFAPlanDescriptor cfa_plan_descriptor, 
    RecursiveAdoptionPlan recursive_plan,
    llvm_function,
    current_block
):
    alternative_blocks = { current_block };
    for alt_index = 1 ... recursive_plan.alternatives.size():
        prev_block = alternative_blocks.[alt_index - 1]
        alternative_block = llvm_create_block("", llvm_function)
        alternative_blocks.push_back(alternative_block)

    for alt_index = 0 ... recursive_plan.alternatives.size() - 1:
        alternative_block = alternative_blocks[alt_index]
        llvm_argument = llvm_function.args[recursive_plan.arg_index]
        helper = TypeManipulationsLLVMTranslator(...)
        is_operator = helper.test_concrete_type_of_union_in_llvm(
            alternative_block, 
            llvm_argument, 
            recursive_plan.alternatives[alt_index]
        )
        
        success_block = llvm_create_block("", llvm_function)
        alternative_block_builder.create_conditional_jump(
            is_operator.value, 
            success_block, 
            alternative_blocks[alt_index + 1]
        )

        successful_plan = recursive_plan.nested_plans[alt_index];
        translate_cfa_plan_to_llvm(
            cfa_plan_descriptor, 
            successful_plan, 
            llvm_function, 
            success_block
        )
    
    // handle last alternative as successful
    translate_cfa_plan_to_llvm(...)
\end{lstlisting}

\newpage

In totale, il numero di blocchi di una funzione in LLVM-IR che implementa 
un overload autogenerato tramite CFA avrà un numero di blocchi calcolabile come:

\begin{equation*}
    \text{numero di blocchi} = \sum_{i=1}^{U} (2K_i - 1)
\end{equation*}

\begin{itemize}
    \item $U$ è il numero di argomenti il cui tipo è una union
    \item $K_i$ è il numero di sottotipi diretti della $i$-esima union tra gli argomenti
\end{itemize}

Si consideri ad esempio il caso di una ipotetica chiamata alla funzione \texttt{double} 
effettuata con un solo argomento di tipo \texttt{Int|Float}, dunque da risolvere 
mediante CFA date le due definizioni, le quali coprono tutti i casi: 

\vspace{0.5cm}
\begin{lstlisting}[frame=single]
func double(x : Int) -> Int { return x * 2; }
func double(x : Float) -> Float { return x * 2.0; }
\end{lstlisting}
\vspace{0.5cm}

Il numero di blocchi necessari per tradurre l'overload autogenerato in LLVM-IR è 
3, in accordo con la forma generale della formula sopra esposta nel 
caso in cui $U = 1$ e $K_1 = 2$. \\

L'overload sarà tradotto in LLVM-IR come segue, al netto delle opportune semplificazioni: \\

\vspace{0.5cm}
\begin{lstlisting}[frame=single]
define void @cfa_double(%"Int|Float" %0) {
    entry:
        ...
        ;; check if the argument is an Int
        %flag = icmp eq i8* %3, getelementptr inbounds (...)
        br i1 %flag, label %IntCase, label %FloatCase
    
    IntCase:
        ...
        %out = call i64 int_double(i64 %9)
        ...
    FloatCase:
        ...
        %out = call f64 float_double(f64 %14)
        ...
    }
\end{lstlisting}
\vspace{0.5cm}