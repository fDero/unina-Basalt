\subsubsection{Traduzione delle funzioni in LLVM}
La traduzione delle funzioni in codice LLVM è responsabilità della classe 
\texttt{CallableCodeBlocksLLVMTranslator}, la quale si occupa di tradurre 
un generico \texttt{CallableCodeBlock}, sia che esso abbia tipo concreto 
\texttt{FunctionDefinition} sia che esso abbia tipo concreto 
\texttt{CFAPlanDescriptor}. \\

\vspace{0.5cm}
\begin{lstlisting}[frame=single]
class CallableCodeBlocksLLVMTranslator {
    
    public:
        CallableCodeBlocksLLVMTranslator(
            ProgramRepresentation& program_representation, 
            TypeDefinitionsLLVMTranslator& types_translator,
            llvm::LLVMContext& llvm_context,
            llvm::Module& llvm_module
        );

        llvm::Function* translate_callable_code_block_to_llvm(
            const CallableCodeBlock& callable_code_block
        );

        llvm::Function* translate_cfa_descriptor_to_llvm(
            const CFAPlanDescriptor& cfa_plan_descriptor, 
            llvm::Function* llvm_function
        );

        llvm::Function* translate_function_definition_to_llvm(
            const FunctionDefinition::Ref& function_definition, 
            llvm::Function* llvm_function
        );

    protected:
        /* other methods */

    private:
        /* other fields */
};

\end{lstlisting}
\vspace{0.5cm}

Il metodo \texttt{translate\_callable\_code\_block\_to\_llvm} si occupa di
effettuare un dispatch sui due metodi \texttt{translate\_cfa\_descriptor\_to\_llvm}
e \texttt{translate\_function\_definition\_to\_llvm} in base al tipo concreto
dell'oggetto passato come parametro. \\

\newpage

Il metodo \texttt{translate\_function\_definition\_to\_llvm} si occupa di tradurre
una definizione di funzione in codice LLVM. Questo metodo è responsabile di creare
un nuovo \texttt{llvm::Function} all'interno del modulo LLVM, e di tradurre il corpo
della funzione in codice LLVM. \\

Una funzione non extern sarà tradotta con un corpo composto da almeno un blocco,
chiamato \textit{entry}, in accordo con le specifiche di LLVM. In tale blocco saranno 
presenti le allocazioni su stack necessarie per poter avere oggetti di tipo 
\texttt{TranslatedExpression} che rappresentino i parametri formali della funzione. Senza 
soluzioni di continuità tale blocco sarà utilizzato come blocco iniziale in cui 
chiamare il metodo \texttt{translate\_all} della classe 
\texttt{ExpressionsAndStatementsLLVMTranslator}. \\

%\vspace{0.5cm}
\begin{lstlisting}[frame=single]
Translator.translate_local_function_definition_to_llvm(
    func_def, llvm_function
):
    entry_block = llvm_create_block("entry", llvm_function)
    body_translator = ExpressionsAndStatementsLLVMTranslator(
        func_def, entry_block, ...
    )
    body_translator.alloc_args(func_def, entry_block)
    exit_block = body_translator.translate_all(
        func_def, entry_block, ...
    )
    inject_return_statement_if_needed(exit_block, ...)
    return llvm_function
\end{lstlisting}
\vspace{0.5cm}

Se la funzione è stata dichiarata come extern, allora la traduzione 
sarà effettuata sotto forma di dichiarazione di simbolo esterno, primitiva 
dell'IR di LLVM. \\

%\vspace{0.5cm}
\begin{lstlisting}[frame=single]
Translator.translate_extern_function_definition_to_llvm(
    func_def, llvm_function
):  
    llvm_function.delete_body()
    llvm_function.set_linkage(llvm.ExternalLinkage)
    llvm_function.set_name(func_def.extern_name)
    return llvm_function
\end{lstlisting}
\vspace{0.5cm}

Una funzione senza corpo, e con policy di linkage "external", sarà tradotta 
come una dichiarazione di simbolo esterno. Ad esempio, la dichiarazione della funzione 
\texttt{square\_root(Float) -> Float} in Basalt, viene tradotta in LLVM-IR come segue: \\

%\vspace{0.5cm}
\begin{lstlisting}[frame=single]
//BASALT
extern square_root(x: Float) -> Float = "sqrt";
\end{lstlisting}
\begin{lstlisting}[frame=single]
;;LLVM-IR
declare double @sqrt(double)
\end{lstlisting}
%\vspace{0.5cm}

\newpage