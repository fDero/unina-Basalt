\subsubsection{Traduzione dei tipi in LLVM-IR}
La traduzione delle definizioni di tipo e delle type-signatures in LLVM-IR mediante è responsabilità della classe 
\texttt{TypesLLVMTranslator}. Questa classe è in grado di tradurre sia i tipi primitivi che i tipi definiti dall'utente.

\vspace{0.5cm}
\begin{lstlisting}[frame=single]
class TypesLLVMTranslator {

    public:
        TypesLLVMTranslator(
            ProgramRepresentation& program_representation, 
            llvm::LLVMContext& context,
            llvm::Module& llvm_module
        );

        llvm::Type* translate_type_definition(TypeDefinition&);
        llvm::Type* translate_typesignature(TypeSignature&);
        vector<llvm::Type*> translate_all(vector<TypeSignature>&);

        size_t compute_sizeof(const TypeSignature& typesignature);

        llvm::GlobalVariable* fetch_type_info(TypeSignature&);

    protected:
        /* other methods */

    private:
        ProgramRepresentation& program_representation;
        llvm::LLVMContext& context;
        llvm::Module& llvm_module;

        unordered_map<string, llvm::Type*> 
            llvm_type_definitions;
        
        unordered_map<string, llvm::GlobalVariable*> 
            llvm_exact_type_infos;

        unordered_map<string, std::vector<llvm::GlobalVariable*>> 
            llvm_compatible_type_infos;
};
\end{lstlisting}
\vspace{0.5cm}

La traduzione di un tipo in LLVM-IR comporta la creazione di un oggetto di tipo \texttt{llvm::Type} che rappresenta
una type-signature in LLVM-IR. In particolare, di tale oggetto si tende a mantenere solo un puntatore, in quanto l'API di 
LLVM lavora solo ed esclusivamente con puntatori. La \textit{lifetime} di tali oggetti è gestita da LLVM stesso, ed in particolare essi sono 
automaticamente deallocati quando il modulo che li contiene viene deallocato. Come si può vedere, la classe \texttt{TypesLLVMTranslator}
mantiene un riferimento a \texttt{llvm::Module} che corrisponde appunto al modulo appena citato. \\ 

La traduzione delle definizioni di struct avviene in modo molto naturale e diretto, in quanto LLVM-IR supporta le struct nativamente, 
ed è quindi sufficiente utilizzare l'API di LLVM opportunamente.

\vspace{0.5cm}
\begin{lstlisting}[frame=single]
TypesLLVMTranslator.translate_struct_definition(struct_def):
    fully_qualified_name = program_representation
        .get_fully_qualified_typedefinition_name(struct_definition)

    if llvm_type_definitions[fully_qualified_name] != nil:
        return llvm_type_definitions[fully_qualified_name]
    
    llvm_type_def = llvm_create_struct(fully_qualified_name)
    llvm_type_definitions.insert(fully_qualified_name, llvm_type_def)
    for field : struct_definition.fields:
        llvm_type_def.fields.append(
            translate_typesignature(field.field_type)
        )
    
    if llvm_type_def.fields.is_empty():
        fields_types.append(llvm::Type::getInt1Ty(context))
    
    return llvm_type_def
\end{lstlisting}
\vspace{0.5cm}

La traduzione avviene solo la prima volta, tutte le altre volte ci si limita a recuperare 
il tipo già tradotto dalla mappa \texttt{llvm\_type\_definitions} usando il nome completamente qualificato come 
chiave di recupero. \\

Inserendo immediatamente il \texttt{llvm::Type*} nella mappa \texttt{llvm\_type\_definitions}, 
si fa sì che per definizioni ricorsive (indirette), si possa evitare di entrare in un loop infinito. \\

Per far si che sia possibile allocare oggetti il cui tipo è una struct senza field si è deciso di inserire un 
campo fittizio di un byte dentro ogni struct che non ha field. A seconda dell'architettura, tale campo potrebbe
potrebbe portare ad una dimensione maggiore per motivi di alignment. \\

In generale, l'ordine di definizione di un field influenza l'alignment, così come avviene in C e C++. Sarebbe stato 
possible implementare un sistema di alignment personalizzato, ma ciò avrebbe compromesso l'interoperabilità con C. \\

\newpage

Per la traduzione delle union invece, è stato necessario utilizzare un approccio differente, in quanto LLVM-IR non supporta
le union nativamente. Per ovviare a questo problema, si è deciso di utilizzare una struct contenente due campi, corrispondneti
ad header e payload. L'header è un puntatore che punta alla \textit{typeinfo} dell'oggetto contenuto nel payload. Il payload
è un array di byte di dimensione pari alla massima dimensione tra i campi della union. Entrambi questi campi sono definiti in base
all'output dei metodi \texttt{fetch\_type\_info} e \texttt{compute\_sizeof}. \\

\vspace{0.5cm}
\begin{lstlisting}[frame=single]
TypesLLVMTranslator.translate_struct_definition(union_def):
    fully_qualified_name = program_representation
        .get_fully_qualified_typedefinition_name(union_def)

    if llvm_type_definitions[fully_qualified_name] != nil:
        return llvm_type_definitions[fully_qualified_name]
    
    llvm_type_def = llvm_create_struct(fully_qualified_name)
    llvm_type_definitions.insert(fully_qualified_name, llvm_type_def)
    for alternative : union_def.alternatives:
        union_payload_sizeof = max(
            compute_sizeof(alternative),
            union_payload_sizeof
        )

    llvm_type_def.fields = [
        llvm_create_pointer_to_type_info(),
        llvm_create_array_of_bytes(union_payload_sizeof)
    )
    
    return llvm_type_def
\end{lstlisting}
\vspace{0.5cm}

Per array e puntatori scalari la traduzione è banale in quanto l'API di LLVM prevede che dato un \texttt{llvm::Type*} si possa 
facilmente ottnere il \texttt{llvm::Type*} corrispondente ad un array di dimensione nota o un puntatore a tale tipo. Quando si incontra 
una \texttt{TypeSignature} si traduce in un \texttt{llvm::Type*} mediante traduzione diretta se è un tipo primitivo, oppure si recupera 
la definizione di tipo corrispondente e si restituisce la traduzione di tale definizione (il recupero restituisce una definizione già reificata). \\

Per i puntatori vettoriali si è deciso di utilizzare una traduzione basata su struct, in quanto LLVM-IR non supporta i vettori
come tipi primitivi. In particolare, si è deciso di utilizzare una struct contenente un campo di tipo \texttt{uint64\_t} che rappresenta
la lunghezza del vettore, e un campo di tipo puntatore al tipo base del vettore. \\

\newpage