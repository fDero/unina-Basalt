\subsubsection{Traduzione degli statements in LLVM-IR}
La traduzione degli statements in LLVM-IR è abbastanza diretta per quanto riguarda istruzioni di assegnamento,
allocazioni, return e simili, ma è invece decisamente più complessa per quanto riguarda le istruzioni di controllo
del flusso di esecuzione quali branch condizionali e cicli iterativi. \\

In generale, in LLVM, attraverso l'API in C++, si può inserire uno statement all'interno di un blocco di codice, 
a sua volta all'interno di una funzione, e si deve concludere ogni blocco con un istruzione di salto. \\

La traduzione di uno o più statement in LLVM-IR, così come la traduzione di una o più espressioni, è responsabilità
della classe \texttt{ExpressionsAndStatementsLLVMTranslator}. Questa classe è stata progettata per essere istanziata 
una volta per ogni blocco di codice delimitato da parentesi graffe, ed è equipaggiata con tutti i metodi necessari a 
tradurre l'intero contenuto di tale blocco nonchè dei blocchi in esso innestati (essa è capace di generare autonomamente
cloni di se stessa specializzati per i blocchi innestati che le si richiede di tradurre). \\

\vspace{0.5cm}
\begin{lstlisting}[frame=single]
class ExpressionsAndStatementsLLVMTranslator {
    
    public:
        ExpressionsAndStatementsLLVMTranslator(
            /* other parameters */
            llvm::BasicBlock* loop_entry_block = nullptr,
            llvm::BasicBlock* loop_exit_block = nullptr
        );

        llvm::BasicBlock* translate_statement(llvm::Block*, ...);
        llvm::BasicBlock* translate_conditional(llvm::Block*, ...);
        llvm::BasicBlock* translate_while_loop(llvm::Block*, ...);
        llvm::BasicBlock* translate_until_loop(llvm::Block*, ...);
        
        TranslatedExpression translate_expression_to_llvm(...);
        TranslatedExpression translate_type_operator(...);

        /* other methods */

    protected:        
        /* other methods */

    private:
        /* other fields */
        llvm::BasicBlock* loop_entry_block;
        llvm::BasicBlock* loop_exit_block;
};
\end{lstlisting}
\vspace{0.5cm}

La traduzione di un branch condizionale (if-statement) schematizzabile con come segue:

\vspace{0.5cm}
\begin{lstlisting}[frame=single]
Translator.translate_conditional(current_block, conditional):
    if_cond_block = create_block_after(current_block)
    if_then_block = create_block_after(if_cond_block)
    if_else_block = create_block_after(if_then_block)
    if_exit_block = create_block_after(if_else_block)

    current_block.create_jump(if_cond_block)

    condition = translate_expression_to_llvm(
        if_cond_block, 
        conditional.condition
    )
    if_cond_block.create_conditional_jump(
        condition.value, 
        if_then_block, 
        if_else_block
    )

    auto then_translator = create_translator_for_nested_conditional()
    if_then_block = then_translator.translate_all(
        if_then_block,
        conditional.then_branch
    )
    if_then_block.create_jump(if_exit_block)

    auto else_translator = create_translator_for_nested_conditional()
    if_else_block = else_translator.translate_all(
        if_else_block,
        conditional.else_branch
    )
    if_else_block.create_jump(if_exit_block)

    return if_exit_block
\end{lstlisting}
\vspace{0.5cm}

Ogni chiamata ai metodi che traducono in LLVM-IR uno statement restituisce il blocco di codice corrente al termine della traduzione
di tale statement. Questo permette di concatenare le traduzioni di più statement in modo relativamente semplice. Al termine della traduzione 
di questo if-statement, il nuovo codice sarà scritto nel blocco chiamato “exit” che corrisponde al blocco a cui si effettua salto non condizionato 
al termine di entrambi i branch (sia “then” che “else”). La creazione di un blocco apposito per la condizione non era necessaria ma 
rende il codice IR generato più leggibile quando emesso in formato testuale, e ciò è utile a fini di debugging. \\

\newpage

La traduzione di un ciclo while è schematizzabile come segue:

\vspace{0.5cm}
\begin{lstlisting}[frame=single]
Translator.translate_while_loop(llvm_block, while_loop):
    while_cond_block = create_block_after(llvm_block)
    while_body_block = create_block_after(while_cond_block)
    while_exit_block = create_block_after(while_body_block)

    llvm_block.create_jump(while_cond_block);

    condition = translate_expression_to_llvm(
        while_cond_block, 
        while_loop.condition
    )
    while_cond_block.create_conditional_jump(
        condition.vale, 
        while_body_block, 
        while_exit_block
    )

    body_translator = create_translator_for_nested_loop(
        while_cond_block, 
        while_exit_block
    )
    while_body_block = body_translator.translate_all(
        while_body_block,
        while_loop.body
    )
    while_body_block.create_jump(while_cond_block)

    return while_exit_block
\end{lstlisting}
\vspace{0.5cm}

La chiamata a \texttt{create\_translator\_for\_nested\_loop} è responsabile della creazione di un nuovo oggetto
\texttt{ExpressionsAndStatementsLLVMTranslator} che sarà utilizzato per tradurre il corpo del ciclo. Tale oggetto
è configurato in modo da avere come blocco di entrata il blocco della condizione del ciclo e come blocco di uscita. \\

Ciò significa che in caso di istruzioni \texttt{break} o \texttt{continue} all'interno del ciclo, tali blocchi saranno 
utilizzati per effettuare rispettivamente o il salto al termine del ciclo o al termine dell'iterazione corrente. \\

\newpage

La traduzione di un ciclo until (simile al do-while) è schematizzabile come segue: