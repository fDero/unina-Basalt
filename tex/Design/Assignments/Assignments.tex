\subsection{Assignments}
Come in ogni linguaggio fortemente tipato, esistono regole ben precise che determinano
quando è possibile assegnare un valore ad una variabile. In generale, è possibile assegnare 
espressioni non solo a variabili ma anche a vere e proprie espressioni. \\

Ci sono due tipi di vincoli
che possono essere posti su un assegnamento: i vincoli di tipo ed i vincoli di immutabilità. \\

I vincoli di tipo sono il motivo stesso dell'esistenza di un linguaggio fortemente tipato,
essi permettono di evitare errori di esecuzione dovuti ad un uso improprio delle variabili
mediante un'analisi basata sul loro tipo dichiarato a tempo di compilazione. \\

I vincoli di immutabilità, invece, hanno a che fare con l'utilizzo delle costanti e/o
delle espressioni immutabili (e.g. stringhe, numeri, ecc.). \\

In una assegnazione, in inglese "assignment", sono coinvolte due espressioni: l'espressione a sinistra
dell'uguale (il target) e l'espressione a destra dell'uguale (il valore da assegnare). Ogni dichiarazione
di costante o di variabile qualora inizializzata è implicitamente considerata un'assegnazione
all'oggetto che si sta dichiarando.

\subfile{AssignmentSemplici}
\subfile{AssignmentTraUnion}
\subfile{AssignmentTraPuntatori}
\subfile{AssignmentTraTipiGenerici}
\subfile{AssignmentTraArray}
\subfile{AssignmentVersoTargetImmutabili}
\subfile{AssignmentDiValoriImmutabili}
\subfile{AssignmentVersoSolaLettura}