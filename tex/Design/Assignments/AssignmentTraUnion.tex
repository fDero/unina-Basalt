\subsubsection{Assignment tra union}
É utile immaginare una union come l'insieme dei tipi nominati direttamente
o indirettamente al suo interno. É possibile assegnare ad una union valori il cui tipo è
in tale insieme o valori il cui tipo è una union corrispondente ad un suo sottoinsieme. \\

Più formalmente, diciamo che è possibile assegnare ad espressioni di tipo union:
\begin{itemize}
    \item valori del suo stesso tipo
    \item valori il cui tipo compare esplicitamente nel suo elenco dei tipi
    \item valori il cui tipo è assegnabile ad un tipo elencato nel suo elenco dei tipi
    \item valori il cui tipo è un’altra union, definita a partire da tipi a loro volta assegnabili
\end{itemize} 

\vspace{0.4cm}

In altri termini, per le union e per le union soltanto si applica una politica di structural-compatibility,
in luogo della name-equivalence.
\vspace{0.4cm}