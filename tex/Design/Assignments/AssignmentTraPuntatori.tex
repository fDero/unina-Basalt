\subsubsection{Assignment tra puntatori}
Quando il target di un'assignment è un puntatore scalare, è possibile assegnarvi solo 
espressioni che siano anch'essi puntatori scalari. Inoltre, è richiesto che i tipi degli oggetti puntati
da entrambi coincidano strutturalmente, ovvero che essi siano identici 
oppure che essi siano union mutuamente assegnabili fra loro. \\  

Ciò significa che anche se fosse possibile assegnare
espressioni di tipo \texttt{V} a target di tipo \texttt{T}, qualora il viceversa non fosse vero, 
allora non sarebbe possibile assegnare espressioni di tipo \texttt{\#V} a target di tipo \texttt{\#T}. \\

Per quanto riguarda i puntatori vettoriali invece, vale quanto detto per i puntatori scalari 
ma con una dovuta precisazione, ovvero che è possibile assegnare ad un puntatore scalare 
\texttt{\$T} un valore di tipo \texttt{[N]T}. Ciò è ovviamente vero in quanto in caso contrario 
verrebbe a mancare il senso stesso dei puntatori vettoriali, ovvero essere uno strumento per la 
gestione deglia array la cui dimensione è ignora a tempo di compilazione. \\

Per capire il motivo di tali restrizioni, è utile considerare il seguente esempio: si considerino
due puntatori scalari \texttt{ptr : \#Int} e \texttt{ptr2 : \#(Int|Float)}, e si immagini cosa 
accadrebbe se fosse possibile assegnare \texttt{ptr} a \texttt{ptr2}. In tal caso, ci si ritroverebbe
con un puntatore che punta ad un'area di memoria della dimensione sbagliata, il che potrebbe portare
a comportamenti imprevedibili e a crash del programma, specialmente se si considera che tramite  un 
riferimento a \texttt{ptr2} si potrebbe tentare di scrivere un valore di tipo \texttt{Float} in un'area
di memoria che può contenere solo valori di tipo \texttt{Int}. \\