\subsubsection{Assignment di espressioni immutabili}
Assegnare espressioni immutabili a target immutabili è sempre permesso 
(ciò per costruzione può solo avvenire in sede di dichiarazione di una costante), 
assegnarli a target mutabili invece, è permesso solo se tale assegnamento non implica un legame
del valore immutabile con un target mutabile. \\

Un legame di un valore immutabile con un target mutabile si ha ad esempio provando ad assegnare
a tale target l'indirizzo di memoria del valore immutabile in formato di puntatore scalare 
o vettoriale. \\

Tale legame consentirebbe infatti di modificare il valore immutabile deferenziando il puntatore 
mutabile così ottenuto, il che è chiaramente in contrasto con la natura stessa del concetto di 
immutabilità. \\