\subsection{Variabili e costanti}
Variabili e costanti sono dei contenitori logici capaci di contenere dei valori decisi a tempo di esecuzione. Ci si potrà riferire 
al valore contenuto in un dato istante di tempo da una variabile o da una costante utilizzando il suo nome. Tramite un apposito 
costrutto detto assegnazione, è possibile riassegnare il valore di una variabile, ciò non è però possibile per una costante, la quale 
una volta dichiarata non potrà mai cambiare valore. L’utilizzo delle costanti in Basalt è infatti concesso solo nei contesti dove il loro 
valore non viene modificato (accesso in sola lettura). 

\subsubsection{Dichiarazione di variabili}
La dichiarazione di una variabile può avvenire con inizializzazione o senza, laddove un valore di inizializzazione sia mancante il valore 
di tale variabile sarà casuale. Ci si aspetta che in tale scenario un valore venga poi assegnato in un secondo momento. Qualunque sia la 
tipologia di dichiarazione scelta, essa deve essere introdotta dalla keyword var, seguida dal nome della variabile, dai due punti e dal tipo 
di tale variabile.

\vspace{0.5cm}

\begin{lstlisting}[frame=single]
var x : Int = 6;
var y : Int;
\end{lstlisting}


\vspace{0.5cm}

\subsubsection{Dichiarazione di costanti}
In maniera del tutto analoga a quanto detto per le variabili, la dichiarazione delle costanti deve essere introdotta dalla keyword const, 
seguita dal nome della costante, dai due punti e dal tipo, solo che a differenza di una variabile, una costante deve necessariamente essere 
inizializzata in sede di dichiarazione.

\vspace{0.5cm}

\begin{lstlisting}[frame=single]
const pi : Float = 3.14;    
\end{lstlisting}

\vspace{0.5cm}

Le costanti possono eventualmente essere rimpiazzate dal loro valore in tutte le loro occorrenze qualora tale valore sia noto a tempo di 
compilazione, e qualora l’indirizzo di memoria della costante non sia utilizzato all’interno del codice sorgente in alcun modo.

É comunque possibile assegnare ad una costante (in sede di dichiarazione) un valore noto a tempo di esecuzione, in tal caso 
l’ottimizzazione appena descritta sarà impossibile, anche se potrebbe comunuqe essere rimpiazzato il suo utilizzo con l’espressione 
in sè qualora tale utilizzo sia unico, e qualora ciò non comporti un cambiamento osservabile del programma ipotizzando un’esecuzione single thread.
