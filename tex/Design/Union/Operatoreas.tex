\subsubsection{Operatore \texttt{as}}
Per poter accedere al valore internamente contenuto da una variabile il cui tipo è una union è possibile usare 
l’operatore \texttt{as}, operatore binario il cui operando sinistro è un'espressione il cui tipo è una union, mentre l’operando destro 
è un tipo che si desidera estrarre dalla union. \\

L’operatore as, è utilizzabile solo su un tipo che sarebbe teoricamente assegnabile all’espressione sulla quale esso viene usato, 
pena un errore a tempo di compilazione. \\

Se lo si usa su espressioni che contengono un tipo diverso, esso non fallisce a tempo di esecuzione, ma si limita a fornire valori 
indefiniti corrispondenti all’interpretazione dei byte del contenuto reale della union come se essi fossero invece del tipo richiesto. \\

L’uso dell’operatore as è consigliato solo all’interno di dei branch condizionali, o dopo degli assert, la cui condizione assicura che la union
contenga effettivamente il tipo che il programmatore si aspetti a tempo di esecuzione. \\   

L’operatore as fornisce un vero e proprio riferimento utilizzabile non solo in lettura ma anche in scrittura, 
è analogo al reinterpret-cast di C++ ma, se usato in condizioni in cui l’operatore is con gli stessi operandi 
avesse valore true, allora il suo buon funzionamento è sempre garantito.