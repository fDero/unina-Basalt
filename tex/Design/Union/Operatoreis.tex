\subsubsection{Operatore \texttt{is}} 
Per conoscere il tipo effettivo rappresentato in un certo momento dell’esecuzione del programma da un oggetto il cui tipo è una union, si può utilizzare 
l’operatore is, il quale si comporta in modo analogo ad instanceof in java o all’omonimo operatore is in C\#, ovvero restituisce true se e solo se il tipo 
concreto dell’oggetto fornito come operando sinistro è assegnabile al tipo fornito come operando destro. \\

\vspace{0.5cm}
\begin{lstlisting}[frame=single]
var num : Int | Float = 6;

if (num is Int) {
    console::println("num is an integer");
}
else { 
    console::println("num is a float");
}
\end{lstlisting}
\vspace{0.5cm}