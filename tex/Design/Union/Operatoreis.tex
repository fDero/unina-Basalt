\subsubsection{Operatore \texttt{is}} 
Per conoscere il tipo effettivo rappresentato in un certo momento dell’esecuzione del programma da un oggetto il cui tipo è una union, si può utilizzare 
l’operatore is, il quale si comporta in modo analogo ad instanceof in java o all’omonimo operatore is in C\#, ovvero restituisce true se e solo se il tipo 
concreto dell’oggetto fornito come operando sinistro è assegnabile al tipo fornito come operando destro. \\

\vspace{0.5cm}
\begin{lstlisting}[frame=single]
var num : Int | Float = 6;

if (num is Int) {
    console::println("num is an integer");
}
else { 
    console::println("num is a float");
}
\end{lstlisting}
\vspace{0.5cm}


Così come le struct, anche le union devono avere una dimensione in byte nota a tempo di compilazione, e dunque analogamente a 
quanto visto per le struct, Basalt reputa non corrette tutte le definizioni di union aventi dipendenze cicliche dirette. \\ 

Una union occupa in memoria dimensione pari al massimo tra le dimensioni dei tipi non-union a partire dai quali è stata definita, 
con un overhead aggiuntivo di otto bytes utilizzati per contenere l’indirizzo in memoria di un oggetto di tipo Type il quale rappresenta 
il tipo effettivo rappresentato da un certo oggetto di tipo union in un dato momento dell’esecuzione del programma. Il tipo Type è un 
tipo speciale in Basalt, esso è la spina dorsale su cui è stato costruito il meccanismo di reflection, 
e ad esso è stata dedicata una sezione apposita di questo documento. \\

Alla luce di quanto detto, ogni oggetto di tipo PrimitiveType dunque occuperà in memoria sedici bytes, otto dedicati a rappresentare 
il valore da esso incapsulato, che può essere di tipo Int, Float, Char, Bool o Byte, ed otto per rappresentare 
l’indirizzo di un oggetto tipo Type, che descrive il tipo di tale valore incapsulato. \\

Si noti come in altri linguaggi, come Java, Kotlin e C\#, ogni oggetto possiede un "object-header”, ovvero un’area del proprio memory 
layout atta a contenere varie informazioni, tra le quali il tipo effettivo dell’oggetto. La differenza principale tra 
tali linguaggi e Basalt è che quest’ultimo limita l’overhead alle sole union, che infatti sono il solo costrutto che consentono 
il disallineamento tra tipo dichiarato e tipo effettivo.