\subsubsection{Assignment tra puntatori a union}
Per le variabili di tipo puntatore a union, sia scalari che vettoriali, le regole di assignment sono più stringenti. Ad una 
variabile di tipo puntatore a union, è possibile assegnare un puntatore ad un’altra union, 
solo se tali union sono in relazione di structural-equivalence tra loro. \\

Due union sono in relazione di structural-equivalence tra loro, se l’una è in relazione di structural-compatibility 
con l’altra, secondo la nozione di structural compatibility data nel pragrafo immediatamente precedente. \\

La relazione di structural-compatibility, nonostante la sua formulazione apparentemente debole, garantisce 
che il memory layout interno sia il medesimo, e pertanto anche strumenti quali i puntatori, che sono per 
definizione dipendenti dalla conoscenza del memroy layout dei valori 
puntati possono godere di una flessibilità maggiorata. \\ 