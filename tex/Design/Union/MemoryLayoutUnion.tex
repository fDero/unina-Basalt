\subsubsection{Memory-layout di una union}
Come già detto nel paragrafo precedente, è stato detto che la dimensione in Byte occupata da una union, è calcolata in funzione della dimensione 
del tipo con la dimensione più grande tra quelli a partire dalla quale essa è stata definita. \\

Una union è internamente rappresentata come due blocchi di byte adiacenti in memoria, il primo, di 8 byte, è detto header, ed è usato per 
contenere metadati necessari al corretto funzionamento dell’operatore is, mentre il secondo è detto payload, e contiene la rappresentazione 
in byte del valore rappresentato a tempo di esecuzione dalla union. \\
 
\begin{figure}[htbp]
    \centering
    \begin{tikzpicture}[
        label/.style={font=\footnotesize, minimum height=1cm, minimum width=2cm, anchor=west},
        box/.style={draw, rectangle, minimum height=1cm, minimum width=3.5cm, anchor=west}
    ]
    \node[box] (addr) at (0, -1.5) {header (8 byte)};
    \node[box] (length) [right=0cm of addr] {payload (dimensione variabile)};
    \end{tikzpicture}
    \caption{Memory layout dei tipi String e RawString}
    \label{fig:string_mem_layout}
\end{figure}

Definiamo dimensione netta di un tipo la dimensione del suo payload, se esso è una union, o 
la sua dimensione complessiva in byte se esso è un tipo di altra natura. \\

La dimensione in byte del payload di una union, ovvero la sua dimensione netta, è pari alla 
dimensione netta del tipo con dimensione netta maggiore tra quelli a partire dai quali la union è stata definita. \\

Per union definite a partire da altre union dunque, gli overhead dati dagli header non sono cumulativi. Gli 8 
byte dedicati all’header sono usati per conservare l’indirizzo in memoria a cui sono conservate le type informations 
relative al tipo di volta in volta contenuto all’interno della union. In sede di assignment, che è l’unica occasione 
in cui il tipo contenuto possa cambiare, tale puntatore viene eventualmente aggiornato. \\

L’assignment ad una union quindi è in realtà una coppia di due operazioni, la prima è la scrittura 
dei byte all’interno del payload (nel caso in cui il tipo del valore assegnato sia una union, saranno 
copiati solo i byte del payload), la seconda è la scrittura dei byte relativi all’header con l’indirizzo, 
staticamente noto, delle type-informations del tipo che si è andati ad assegnare 
(nel caso in cui tipo del valore assegnato sia una union, saranno copiati i byte del suo 
header all’interno dell’header della union destinazione). \\

Qualcosa di funzionalmente analogo a quanto descritto fin ora, sono le std::variant introdotte 
nella libreria standard C++ a partire dallo standard C++17. Esse non sono parte del core language, 
e sono invece definite usando la metaprogrammazione C++. \\