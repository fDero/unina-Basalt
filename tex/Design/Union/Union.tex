\subsection{Union}
Le union sono un costrutto che consente al programmatore di definire un tipo di dato la cui rappresentazione 
interna può variare nell’ambito di un numero finito di opzioni mutuamente esclusive e note a priori. \\ 

Le union in Basalt non sono implementate come in C, e sono invece più simili ad i "sum-types” presenti in molti 
linguaggi funzionali come Haskell, Idris o ML. \\

In Basalt, la definizione di una union avviene utilizzando la parola chiave "union" seguita dal nome della union, 
che deve iniziare per maiuscola come ogni altro tipo nel linguaggio, dal simbolo uguale, 
e da una serie di tipi separati da "|” (pipe). \\

\vspace{0.5cm}
\begin{lstlisting}[frame=single]
union Number = Int | Float
\end{lstlisting}
\vspace{0.5cm}

Non è necessario definire una union dandole un nome, è infatti possibile utilizzare union anonime, 
ovvero union definite su una singola riga direttamente al momento dell’utilizzo. \\

La sintassi per fare ciò, prevede semplicemente di utilizzare  una serie di tipi separati da "|”
in tutti i contesti in cui il type-system richiede l’utilizzo di un tipo. In automatico 
tale entità verrà interpetata come union-anonima. \\

\vspace{0.5cm}
\begin{lstlisting}[frame=single]
var named_union_example : Number = 3.14;     
var inline_union_example : Int | Float = 7;
\end{lstlisting}
\vspace{0.5cm}

\subfile{UnionGeneriche} \newpage

\subfile{AssignmentTraUnion} 
\subfile{AssignmentTraPuntatoriUnion}
\subfile{UnionComeParametriAttualiDiTipo} 
\newpage


\subfile{Operatoreis} \newpage
\subfile{Operatoreas} \newpage
\subfile{UnionRicorsive} \newpage
\subfile{MemoryLayoutUnion} \newpage
