\subsubsection{Espressioni elementari}
Una espressione elementare è una espressione che ricade in una delle seguenti categorie:
\begin{itemize}
    \item Applicazione di un operatore binario (come ad esempio l'operatore di somma \texttt{+} o l'operatore di confronto \texttt{==})
    \item Applicazione di un operatore unario non di dereferenziazione (come ad esempio l'operatore di negazione logica \texttt{!})
\end{itemize}
Espressioni di questo tipo sono sostanzialmente paragonabili ai valori letterali, nel senso che svolgono una funzione
analogo all'interno del codice, e pertanto condividono con queste ultime la proprietà di essere immutabili.