\subsubsection{Espressioni di sola lettura}
Una espressioni di sola lettura, è una espressione che restituisce un valore temporaneo preso per copia. Questo 
significa che anche qualora essa fosse mutabile, la modifica non sarebbe osservabile in alcun modo. Un esempio di 
espressione di sola lettura è la chiamata ad una funzione che restituisce un semplice numero intero. La modifica ad una variabile 
che conserva il risultato di tale chiamata è lecita, ma una modifica al valore restituito dalla chiamata stessa non avrebbe senso 
in quanto non sarebbe osservabile.