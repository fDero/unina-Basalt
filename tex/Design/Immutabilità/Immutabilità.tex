\subsection{Immutabilità}
Con il termine immutabilità si intende la proprietà di un oggetto di non poter essere modificato una volta creato. In Basalt, le variabili
sono mutabili di default, ovvero è possibile modificarne il valore in qualsiasi momento. Tuttavia, esistono molteplici costrutti in basalt 
che permettono di ottenere l'immutabilità. Ciò che è immutabile non può essere modificato ne tramite assegnazione, ne tramite assegnazione 
ai suoi field, se è una struct, alle sue celle, se è una slice, un array o una stringa, ne all'area di memoria da esso, se è un puntatore. 

\subfile{ValoriLetterali}
\subfile{EspressioniSemplici}
\subfile{SolaLettura}
\subfile{Costanti}