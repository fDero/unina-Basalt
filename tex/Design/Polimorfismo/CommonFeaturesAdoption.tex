\subsubsection{Common features adoption (CFA)}
Con \textit{Common Features Adoption}, abbreviato come CFA, ci si riferisce all’abilità del compilatore di generare un'overload di 
una funzione, basandosi sugli overload già esistenti, direttamente in sede di chiamata. In particolare, gli overload auto-generati 
implicitamente tramite CFA sono costruiti in modo tale da effettuare un dispatch a tempo di esecuzione sui tipi concreti degli argomenti
di una chiamata a funzione, in modo da poter chiamare il corretto overload già esistente. \\

Si considerino ad esempio le seguenti definizioni di funzioni:

\vspace{0.5cm}
\begin{lstlisting}[frame=single]
func half(x : Int) -> Int {
    return x / 2;
}

func half(x : Float) -> Float {
    return x * 0.5;
}
\end{lstlisting}
\vspace{0.5cm}

Se si effettuasse una chiamata alla funzione \texttt{half}, con un argomento di tipo 
\texttt{Int|Float}, il sistema non troverebbe un overload specifico. Esso però sarebbe in grado 
di generare un overload ad-hoc dietro le quinte simile a quanto riportato di seguito:

\vspace{0.5cm}
\begin{lstlisting}[frame=single]
func half(x : Int | Float) -> Int | Float {
    if (x is Int) {
        return half(x as Int);
    } else {
        return half(x as Float);
    }
}
\end{lstlisting}
\vspace{0.5cm}

Tutto ciò che il programmatore dovrà fare sarà effettuare una chiamata a funzione. Qualora non esistesse 
un overload specifico per i tipi degli argomenti passati, il compilatore proverà a generare un overload 
definito per casi analizzando uno per uno tutti gli argomenti passati in chiamata da sinistra a destra.

Si considerino infatti i seguenti overload per la funzione \texttt{add}:

\vspace{0.5cm}
\begin{lstlisting}[frame=single]
func add(x : Int, y : Int) -> Int {
    return x + y;
}

func add(x : Float, y : Int) -> Float {
    return x + utils::convert_to<Float>(y);
}

func add(x : Int, y : Float) -> Float {
    return utils::convert_to<Float>(x) + y;
}

func add(x : Float, y : Float) -> Float {
    return x + y;
}
\end{lstlisting}
\vspace{0.5cm}

In questo contesto, sarebbe possibile effettuare le seguenti chiamate a funzione:

\vspace{0.5cm}
\begin{table}[h]
    \centering
        \begin{tabularx}{\textwidth}{|b|h|} \hline
            \cheader{CHIAMATA A FUNZIONE}             & \cheader{DEFINIZIONE CORRISPONDENTE}  \\ \hline
            \texttt{add(Int, Int)}                    & overload definito dall'utente         \\ \hline
            \texttt{add(Int, Float)}                  & overload definito dall'utente         \\ \hline
            \texttt{add(Float, Int)}                  & overload definito dall'utente         \\ \hline
            \texttt{add(Float, Float)}                & overload definito dall'utente         \\ \hline
            \texttt{add(Int|Float, Int)}              & overload generato tramite CFA         \\ \hline
            \texttt{add(Int|Float, Float)}            & overload generato tramite CFA         \\ \hline
            \texttt{add(Int, Int|Float)}              & overload generato tramite CFA         \\ \hline
            \texttt{add(Float, Int|Float)}            & overload generato tramite CFA         \\ \hline
            \texttt{add(Int|Float, Int|Float)}        & overload generato tramite CFA         \\ \hline
        \end{tabularx}
    \caption{Comparazione specificità di overload per la funzione \texttt{add}}
\end{table}
\vspace{0.5cm}

\newpage