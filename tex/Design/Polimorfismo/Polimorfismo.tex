\subsection{Pseduo-polimorfismo}
Con il termine polimorfismo in informatica, ci si riferisce alla capacità di un linguaggio 
di poter astrarre dal tipo concreto di un oggetto, permettendo di scrivere codice che 
possa essere applicato a tipi diversi, i quali condividono la stessa API. \\

Questo concetto viene spesso legato a doppio filo con quello di ereditarietà nei contesti
di programmazione ad oggetti, in quanto l'API condivisa a tutti i tipi concreti su cui 
si desidera operare viene codificata nella forma della loro classe base (parent-class). \\

In Basalt, così come in Go, Rust e altri linguaggi moderni, è possibile ottenere gli stessi 
effetti del polimorfismo object-oriented senza dover ricorrere all'ereditarietà. In particolare
Basalt utilizza un approccio unico nel panorama dei linguaggi di programmazione di basso livello,
che sfrutta una feature chiamata \textit{Common Features Adoption (CFA)} per implementare così una forma di 
pseudo-polimorfismo. \\

\subfile{CommonFeaturesAdoption}
\subfile{ImplicazioniDellaCFA}
\subfile{Considerazioni}