\subsubsection{Considerazioni e compromessi riguardanti la CFA}
Risulta evidente come la CFA possa risultare peggiore dell'ereditarietà in termini 
di performance, in quanto ogni singola chiamata a funzione potrebbe comportare un controllo 
in tempo lineare sul numero di tipi concreti che è possibile assegnare ai vari tipi base degli argomenti. \\

Ciò nonostante, la CFA consente di scrivere codice estremamente flessibile e modulare, senza 
costringere il sistema a dover tener traccia di v-tables e/o object-headers di sorta, i quali
sono invece necessari per implementare il polimorfismo tramite ereditarietà come ad esempio accade 
in Java, Kotlin, C\#, i quali sono costi di overhead che impattano l'intera codebase, e non solo le sezioni
che necessitano di usare il polimorfismo. \\

Basalt è ottimizzato per le situazioni dove non è necessario polimorfismo dinamico, in quanto ci si aspetta
che solo una minima parte della codebase necessiti di tale feature, e pertanto, si ritiene che sia più giusto
pagare un costo anche considerevole in termini di performance in situazioni mediamente rare pur di ottenere
codice meglio performante in tutti gli altri scenari. \\

Si tenga poi a mente che per scenari dove il numero totale di tipi concreti da considerare 
per la generazione di overload CFA è molto contenuto (non più di 5 tipi), la CFA potrebbe 
risultare competitiva in termini di performance dato che il numero totale di istruzioni macchina
corrispondenti a risolvere un riferimento a metodo in una v-table non è troppo dissimile dal numero 
di istruzioni necessarie per risolvere tale overload CFA. \\