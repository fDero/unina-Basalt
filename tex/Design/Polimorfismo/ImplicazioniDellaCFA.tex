\subsubsection{Implicazioni della CFA}
Considerato quanto detto finora, è possibile trarre alcune conclusioni riguardo i benefici
e le limitazioni che derivano dall'utilizzo della CFA in Basalt. \\

È necessario sottolineare come la CFA consenta di esporre al programmatore un'API basata 
sulle funzionalità comuni a tutti i tipi concreti che è possibile assegnare a un tipo 
base. Nel caso di Basalt, tale tipo base è una union. Questo significa che il programmatore
può scrivere codice che opera su un tipo base, senza dover conoscere il tipo concreto,
usando le funzionalità comuni a tutti i tipi concreti proprio come se tale tipo base fosse un'interfaccia
di un linguaggio ad oggetti (e.g. Java, Kotlin, C\#). \\

Ad esempio, sarà possibile chiamare la funzione \texttt{size} su una union dei tipi 
\texttt{List<T>}, \texttt{Tree<T>}, \texttt{HashSet<T>} come se tale union fosse un'interfaccia
che esponesse tale funzionalità. Affinché ciò avvenga, sarà necessario che esista un
overload definito dall'utente della funzione \texttt{size} per tutti i tipi citati.