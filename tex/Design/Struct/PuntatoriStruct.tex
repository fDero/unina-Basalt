\subsubsection{Puntatori a struct}
In C e in C++, per accedere ai campi di un oggetto dato un puntatore a tale oggetto, occorre o dereferenziare il 
puntatore, per poi utilizzare l’operatore “.” sull’oggetto così ottenuto, oppure usare l'operatore 
apposito “->” funzionalmente analogo. \\ 

In Basalt, così come in Go, è possibile usare l’operatore “.” direttamente sul puntatore per accedere ai campi 
dell’oggetto puntato. Tale sintassi non porta ambiguità dato che non vi sono altri 
significati per l’operatore “.” applicato ad un puntatore. \\

Consideriamo infatti il seguente frammento di codice, dove vengono definite diverse variabili, e alcune delle 
quali sono puntatori. In questo esempio, sarà fatto riferimento alla 
definizione per la struct Person data nella pagina precedente. \\

\vspace{0.5cm}
\begin{lstlisting}[frame=single]
var person : Person;
var person_ptr : #Person = &person;
var person_ptr_ptr : ##Person = &person_ptr;
\end{lstlisting}
\vspace{0.5cm}
 
Allora si potrà accedere al campo “name” della variabile “person” posponendo 
“.name” a una qualunque di queste tre variabili. \\ 

Go è stato il primo linguaggio ad introdurre un meccanismo del genere, seppur in una forma 
più limitata dove è possibile accedere al campo dell’oggetto puntato solo da un 
puntatore che vi punti direttamente, e non consentendolo invece nei casi dove vi 
sono puntatori a puntatori, o più in generale, due o più livelli di indirezione. \\ 

Tale sovraccarico della semantica dell’operatore “.”, consente di ridurre al minimo le 
modifiche da effettuare ad un blocco di codice funzionante qualora si voglia decidere 
di cambiare il tipo di una delle variabili che esso utilizza rendendola un puntatore 
invece che un oggetto locale. Dunque la scelta di estendere l’utilizzo di tale 
operatore in tal modo è stata fatta per facilitare refactoring del codice. \\ 

Ciò è particolarmente vero nei casi in cui una funzione utilizza già un puntatore per accedere 
in lettura e scrittura ai campi di un oggetto e ci si accorge in un secondo momento che tale funzione 
ha bisogno di eventualmente riassegnare un nuovo valore all’oggetto stesso. In tal caso, l’unica modifica 
da apportare alla funzione sarà cambiare il tipo dell’argomento in questione e rendendolo un puntatore 
a puntatore, mantenendo il resto del codice intatto. Chiunque abbia programmato C abbastanza a 
lungo si potrà facilmente rendere conto che tale scenario è molto comune e pertanto facilitare 
la risoluzione di un problema del genere è qualcosa di cui il programmatore 
medio può beneficiare in modo concreto e tangibile. \\