\subsection{Struct}
Le struct, abbreviazione di "structures" in inglese, rappresentano un fondamentale costrutto di molti 
linguaggi di programmazione, incluso Basalt. Una struct è difatti un tipo definito dall’utente, preposto alla 
modellazione di entità complesse, concettualmente rappresentabili come un aggregato di dati distinti. \\

In altri linguaggi, costrutti analoghi sono chiamati "records” o "product-types”. \\

In Basalt, la definizione di una struct avviene utilizzando la parola chiave \texttt{struct} seguita dal nome della struct, 
che deve iniziare per maiuscola come ogni altro tipo nel linguaggio, e da una serie di campi all'interno di parentesi graffe. \\

Ogni campo deve essere nella forma <nome> : <tipo>, come mostrato di seguito:

\vspace{0.5cm}
\begin{lstlisting}[frame=single]
struct Person {
    name : String;
    surname : String;
    occupation : String;
}
\end{lstlisting}
\vspace{0.5cm}


Una volta definita una struct, è possibile usare il suo nome nei contesti dove il linguaggio Basalt richiede un tipo, come ad esempio nella dichiarazione di una variabile, 
nei parametri delle funzioni o come tipo di un campo di un’altra struct. 

Su ogni variabile il cui tipo è una struct, è possibile applicare l’operatore binario ".” così come nella maggior parte dei linguaggi di 
programmazione, tale operatore consente di accedere ai campi di una specifica istanza di una struct. 

Assumendo dunque di avere accesso alla definizione di Person dal precedente frammento di codice, sarà dunque 
lecito dichiarare variabili di tipo Person e accedere in lettura e scrittura ai loro campi utilizzando l’operatore ".” 
avendo come operatore sinistro un oggetto di tipo Person e come operatore destro il nome di uno specifico campo.

\vspace{0.5cm}
\begin{lstlisting}[frame=single]
var john : Person;

john.name = "John";
john.surname = "Doe";
john.occupation = "Programmer";
\end{lstlisting}
\vspace{0.5cm}

\newpage
\subfile{PuntatoriStruct} \newpage
\subfile{StructRicorsive} \newpage