\subsubsection{Struct ricorsive}
Le variabili, al momento della creazione sia su stack che su heap, devono avere una dimensione 
in bytes nota a tempo di compilazione. Tale dimensione, per le variabili il cui tipo è una struct, 
è ottenuta calcolando la somma delle dimensioni dei field, i cui tipi possono potenzialmente 
essere anch’essi struct. \\ 

Date queste premesse, è chiaro che definizioni ricorsive come la definizione seguente, 
sono errate e portano ad un errore a tempo di compilazione. \\

\vspace{0.5cm}
\begin{lstlisting}[frame=single]
struct Recursive {

    // Ricorsione diretta -> Errore
    recursive : Recursive;
}
\end{lstlisting}
\vspace{0.5cm}

 
Il compilatore Basalt, non potrebbe calcolare la dimensione in byte di un ipotetico oggetto il cui tipo è 
Recursive e di conseguenza, causa un errore di compilazione. \\ 

Basalt riesce ad identificare questo errore esplorando il grafo orientato che le definizioni di struct implicitamente 
descrivono ed implementa un controllo di aciclicità su di esso. \\

Basalt in tale controllo si limita ad esplorare gli archi relativi a tipi semplici e array, 
e invece non esplora archi relativi a puntatori scalari e vettoriali. Questo perchè un puntatore, 
vettoriale o scalare, ha sempre dimensione nota. Ne consegue che questa definizione alternativa 
della struct Recursive è invece corretta e perfettamente valida. \\

\vspace{0.5cm}
\begin{lstlisting}[frame=single]
struct Recursive {

    // Ricorsione indiretta -> Corretto
    recursive_ptr : #Recursive;
    recursive_slice : $Recursive;
}
\end{lstlisting}
\vspace{0.5cm}