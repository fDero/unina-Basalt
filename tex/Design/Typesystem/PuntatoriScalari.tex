\subsubsection{Puntatori scalari}
In Basalt, i puntatori scalari, più semplicemente detti puntatori, sono dei riferimenti ad un oggetto allocato in memoria, 
avente un certo tipo noto a tempo di compilazione. \\

Dato un qualunque tipo \texttt{T}, allora con \texttt{\#T}, indichiamo il tipo dei puntatori a oggetti di tipo \texttt{T}. In Go, C e C++ il simbolo preposto a 
questo scopo è l’asterisco ("\texttt{*}”), mentre in Jai il simbolo preposto a questo scopo è il carot ("\texttt{\^}”). Il motivo per cui 
Basalt si discosta dagli altri linguaggi per quanto riguarda il simbolo usato per indicare un puntatore è che Basalt vuole cercare di non usare lo stesso 
simbolo in contesti troppo diversi fra loro. In particolare, dato che l’asterisco e il carot sono simboli già in uso in qualità di operatori binari, 
è sembrato più saggio scegliere un altro simbolo da dedicare allo scopo di indicare i puntatori. \\  

In maniera conforme a quanto visto in C, C++, Go e molti altri linguaggi, dato un qualunque oggetto di tipo Type, l’operatore unario prefisso \texttt{\&} 
consente di estrarre l’indirizzo di memoria di tale valore. Tale indirizzo avrà tipo \texttt{\#T} e sarà per tanto assegnabile ad un puntatore 
a Type come mostrato nel seguente esempio. \\

\vspace{0.5cm}
\begin{lstlisting}[frame=single]
var number : Int = 6;
var ptr : #Int = &number;
\end{lstlisting}
\vspace{0.5cm}

Ad un puntatore, è possibile assegnare un valore fittizio detto null per rappresentare il fatto che in quel momento il puntatore non sta puntando a 
un’area di memoria valida. \\

I puntatori possono riferirsi sia ad aree di memoria su stack sia su heap, ma per allocare memoria su 
heap sarà necessario chiamare manualmente funzioni di allocazione. Una volta allocata memoria, essa dovrà essere 
deallocata manualmente in quanto Basalt non possiede un garbage collector a differenza di Go e Java, e invece consente 
all’utente di gestire la memoria manualmente così come C, C++, Zig, Odin e Jai. \\

Nel package memory è possibile trovare una funzione malloc e una funzione free, preposte all’allocazione e alla deallocazione di memoria dinamica su heap, 
di seguito è riportato un esempio d’uso. Si tenga a mente che la sintassi con le parentesi angolari sarà analizzata con maggior 
dettaglio in seguito nella sezione dedicata ai generics. \\


\vspace{0.5cm}
\begin{lstlisting}[frame=single]
var ptr : #Int = memory::malloc<Int>(6);
memory::free<Int>(ptr);\vspace{0.5cm}
\end{lstlisting}
\vspace{0.5cm}