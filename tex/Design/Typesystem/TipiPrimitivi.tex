\subsubsection{Tipi primitivi semplici}
Con "tipi primitivi semplici", in Basalt, si intendono i seguenti tipi di dato:

\vspace{0.5cm}
\begin{table}[h]
    \centering
        \begin{tabularx}{\textwidth}{|b|b|} \hline
            \cheader{IDENTIFICATIVO} & \cheader{DESCRIZIONE}                                                                                                  \\ \hline
            \texttt{Int}             & tipo di dato preposto alla rappresentazione dei numeri interi, rappresentato a 64 bit                                  \\ \hline
            \texttt{Float}           & tipo di dato preposto alla rappresentazione dei numeri decimali frazionari, internamente analogo ad un double in C/C++ \\ \hline
            \texttt{Bool}            & tipo di dato preposto alla rappresentazione di valori logici (booleani) di vero/falso                                  \\ \hline
            \texttt{Char}            & tipo di dato preposto alla rappresentazione di un singolo carattere ascii 8 bit                                        \\ \hline
        \end{tabularx}
    \caption{Tipi primitivi}
\end{table}
\vspace{0.5cm}

In Basalt, variabili il cui tipo è un tipo primitivo semplice, vengono allocate su stack. Tutte le volte che si lavora con una variabile così dichiarata, 
si deve dunque assumere che essa si trovi o sullo stack della funzione corrente (compresi gli argomenti delle funzioni).