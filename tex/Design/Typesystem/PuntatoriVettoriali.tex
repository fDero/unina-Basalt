\subsubsection{Puntatori vettoriali}
Contrapponendosi ai puntatori scalari vi sono poi i puntatori vettoriali. Un puntatore vettoriale è un puntatore ad una sequenza 
di oggetti contigui in memoria il cui tipo è noto a tempo di compilazione, ma la cui lunghezza è nota a tempo di esecuzione. Per 
semplicità è possibile chiamarli ”slice” così come si fa in molti altri linguaggi. \\

I puntatori vettoriali sono internamente implementati come una coppia di un puntatore ed una dimensione.  Dato un tipo T allora il tipo \$T ne denoterà 
il puntatore vettoriale. \\

Un puntatore vettoriale in una macchina a 64-bit occupa internamente 16 byte, di cui 8 sono dedicati a conservare un 
indirizzo di memoria ed altri 8 sono dedicati a conservare la lunghezza, ovvero il numero di celle contigue allocate a partire da tale indirizzo. \\

A differenza dei puntatori scalari, un puntatore vettoriale non può essere null, però può avere dimensione zero, che è infatti il comportamento 
standard per un puntatore vettoriale non ancora inizializzato. Questo consente di poter scrivere codice che lavora con puntatori vettoriali senza 
doversi assicurare ogni volta che il puntatore sia non nullo, ma semplicemente controllando di accedere sempre ad esso con indici 
strettamente minori della sua dimensione come è del resto naturale fare anche per gli array. \\

La sintassi per accedere all’i-esimo elemento di un puntatore vettoriale è del tutto uguale a quanto già 
visto per gli array, ovvero si pone alla destra del puntatore vettoriale, da cui si desidera leggere, un espressione di tipo intero, il 
cui valore numerico sarà interpretato come indice, racchiusa fra parentesi quadre. \\

Il seguente frammento di codice illustra come si può istanziare un blocco di memoria dinamica su heap e come lo 
si può gestire mediante un puntatore vettoriale a tale blocco. In particolare il seguente codice stampa il contenuto di ogni cella del 
blocco. \\

\vspace{0.5cm}
\begin{lstlisting}[frame=single]
var i : Int = 0;
var slice : $Int = malloc<$Int>([5]Int{0, 1, 2, 3, 4});

while (i < slice.size){

    println(slice[i]);
    i = i + 1;
}
memory::free<$Int>(slice);
\end{lstlisting}
\vspace{0.5cm}


É possibile assegnare ad una variabile di tipo “puntatore vettoriale a T”, un’espressione di tipo “puntatore scalare ad array di oggetti di tipo T” 
di qualsiasi dimensione. Ciò consente l'utilizzo del puntatore vettoriale come supertipo di tutti gli array. Tale assegnazione 
comporta un effetto simile a quello osservato quando si assegna l'indirizzo di un oggetto già istanziato a un puntatore utilizzando 
l'operatore di indirizzo \texttt{\&}. In tal modo, entrambi i riferimenti puntano alla stessa area di memoria. \\

\vspace{0.5cm}
\begin{lstlisting}[frame=single]
var array : [10]Int = [10]Int{0,1,2,3,4,5,6,7,8,9};
var slice : $Int = &array;
\end{lstlisting}
\vspace{0.5cm}

Dato che un puntatore vettoriale, così come un puntatore scalare, non conserva informazioni sufficienti a determinare se 
l’oggetto puntato si trovi su stack o su heap, e dato che Basalt si prefigge come obbiettivo quello di non 
effettuare allocazioni nascoste e invece di essere sempre trasparente riguardo alla gestione della memoria, ne consegue che non 
è possibile inserire nuovi elementi in un puntatore vettoriale o ridimensionarlo in qualsiasi altro modo. \\

Un’ipotetica implementazione di un array dinamico propriamente detto con possibilità di 
inserire e rimuovere elementi da esso potrebbe essere quella mostrata nel seguente frammento di codice. 
Si tenga a mente che tale frammento usa struct e generics, entrambi argomenti che saranno trattati in 
dettaglio nelle loro sezioni apposite. \\


\vspace{0.5cm}
\begin{lstlisting}[frame=single]
package slicedemo;

struct Slice<T> {
    storage : $T;
    size : Int;
}

func append<T>(slice : Slice<T>, value : T){
    if (slice.size + 1 > slice.storage.length){
        var old : $T = slice.storage; 
        var new_length = 2 * slice.storage.length;
        slice.storage = memory::malloc<$T>(new_length);
        memory::copy<T>(old, slice.storage);
        memory::free<$T>(old);
    }
    slice.storage[slice.size] = value;
    slice.size += 1;
}
\end{lstlisting}
\vspace{0.5cm}