\subsubsection{Array}
In Basalt, gli array sono dei blocchi di memoria contigua, capaci di contenere un numero noto a tempo di compilazione di oggetti dello stesso tipo. \\

Dato un tipo Type ed una lunghezza N allora il tipo [N]Type denoterà il tipo di un array contenente esattamente  N oggetti di tipo Type. Basalt 
conserva la lunghezza come parte del tipo, ciò implica che è possibile definire una funzione che prenda come parametro di input un array di cui 
sia specificata la lunghezza, a differenza del C dove invece si è obbligati a passare la lunghezza tramite l’utilizzo di un parametro ausiliario.  \\

Basalt supporta array-literals sottoforma di tipo esplicito dell’array, seguito da una lista di valori separati 
da virgole e racchiusi tra parentesi graffe. Tale sintassi può essere usata per inizializzare un array in 
sede di dichiarazione come illustrato di seguito. \\

\vspace{0.5cm}
\begin{lstlisting}[frame=single]
var array : [10]Int = [10]Int{0,1,2,3,4,5,6,7,8,9}
\end{lstlisting}
\vspace{0.5cm}

Così come in quasi tutti i linguaggi imperativi ad oggi usati, dato un array, si può accedere in lettura (e in scrittura qualora non sia costante) 
al suo ennesimo elemento usando la canonica sintassi storicamente introdotta dal C, che prevede 
di posporre all’espressione costituente l’array e racchiusa tra parentesi quadre, un'espressione il cui valore sia intero e che 
corrisponda alla posizione dell’elemento all’interno dell’array, assumendo un’indicizzazione che parte da zero. \\

In generale, un array occupa in memoria un numero di byte pari al prodotto della dimensione 
in byte di un singolo oggetto in esso conservato, moltiplicato per la lunghezza, ed è dunque privo di qualunque overhead 
dato che la dimensione è nota a tempo di compilazione e pertanto non viene conservata in memoria. \\

Un assignment tra array è possibile solo se hanno la stessa dimensione e se i tipi 
degli oggetti in essi conservati sono tali da consentire un ipotetico assegnamento cella a cella. Qualora tali 
requisiti siano soddisfatti allora l’assignment performerà una copia di tutti gli elementi dell’array sorgente nell’array destinazione.