\subsubsection{Stringhe}
La gestione delle stringhe nei linguaggi di basso livello è da sempre una sfida, in C, C++, Zig e Odin le stringhe non sono altro 
che puntatori ad aree di memoria contigue dove sono conservati dei caratteri. Tale è anche l’approccio di Basalt, dove le stringhe, 
indicate con String, sono implementate come puntatori vettoriali a carattere. \\

Per facilitare l’ineroperabilità con C, esiste anche il tipo RawString che è invece implementato come un puntatore scalare a carattere, 
e che punta al primo carattere della sequenza che compone la stringa. In C infatti, una stringa altro non è che un puntatore al primo 
carattere che ne fa parte. Non avendo una dimensione, le stringhe in C devono essere marcate al termine da un carattere speciale ‘\textbackslash0’ 
che ne segnala la terminazione. Al fine di poter convertire agevolmente una String in una RawString, la quale può essere usata per 
interfacciarsi con C, allora in Basalt è comunque presente il carattere speciale ‘\textbackslash0’ al termine di ogni sequenza di caratteri 
conservata in ogni oggetto di tipo String anche se superfluo. Si analizzi dunque il seguente codice. \\

\vspace{0.5cm}
\begin{lstlisting}[frame=single]
var str : String = "hello world!";
var cstr : RawString = str;
\end{lstlisting}
\vspace{0.5cm}

Nel frammento di codice appena mostrato, si assegna il valore di una variabile di tipo String ad una variabile di tipo RawString. É 
possibile descrivere graficamente lo stato della memoria al termine dell’esecuzione di questo frammento di codice con un 
Memory-Layout-Diagram nel seguente modo: \\


\begin{figure}[htbp]
    \centering
    \begin{tikzpicture}[
        label/.style={font=\footnotesize, minimum height=1cm, minimum width=2cm, anchor=west},
        box/.style={draw, rectangle, minimum height=1cm, minimum width=2cm, anchor=west},
        byte/.style={draw, rectangle, minimum height=1cm, minimum width=0.75cm, anchor=west},
        pointer/.style={draw, rectangle, minimum height=1cm, minimum width=3.5cm, anchor=west, fill=gray!20}
    ]

    \node[label] (str) at (0, -0.75) {var str : String};
    \node[label] (str) at (0, -7.25) {var cstr : RawString};
    \node[pointer] (addr) at (0, -1.5) {indirizzo di memoria del contenuto (8 byte)};
    \node[pointer] (length) [right=0cm of addr] {length (8 byte)};
    \node[box] (cstr) at (0, -6.5) {indirizzo di memoria del contenuto (8 byte)};
    \foreach \i/\char in {0/h, 1/e, 2/l, 3/l, 4/o, 5/ , 6/w, 7/o, 8/r, 9/l, 10/d, 11/!, 12/\textbackslash0} {
        \node[byte] (char\i) at (\i*0.75, -4) {\texttt{\char}};
    }

    \draw[->] (addr.south) -- ++(0, -0.75) -| (char0.north);
    \draw[->] (cstr.north) -- ++(0, +0.75) -| (char0.south);

    \end{tikzpicture}
    \caption{Memory layout dei tipi String e RawString}
    \label{fig:string_mem_layout}
\end{figure}
\newpage


Qualunque string-literal, quindi anche "Hello, World!" nell’esempio di prima, viene implicitamente spostata 
nello scope globale così che dall’interno di una funzione si possa restituire una string-literal senza temere 
che alla fine della chiamata lo stack della funzione venga ripulito e che la stringa appena restituita venga 
sovrascritta o invalidata. \\

Questo meccanismo di gestione delle string-literals viene chiamato string-pooling, e l’area di memoria 
nello scope globale dedicata a contenere tutte le string-literal dell’intero programma viene detta string-pool. Questo 
meccansimo consente poi di non dover replicare le string-literal, infatti qualora una stessa string-literal apparisse 
più volte nel programma in scope diversi, sarebbe comunque utilizzato l’indirizzo della stessa unica string-literal 
nella pool per inizializzare variabili o per effetturare accessi in lettura. \\

Così come in Go e in Java, le stringhe sono immutabili, questo è un prerequisito essenziale al funzionamento dello string-pooling, dato 
che stringhe in scope diversi si riferiscono in realtà alle stesse sequenze di caratteri nella pool. \\

Per modificare una stringa, occorrerà dunque allocare una nuova area di memoria (su stack 
utilizzando un array di caratteri da castare successivamente a puntatore vettoriale o su heap utilizzando una 
funzione di allocazione e gestendo un puntatore vettoriale direttamente) per ospitare i caratteri della stringa, 
si procederà alla modifica e si assegnerà il puntatore vettoriale dell’area di memoria contenete la nuova sequenza 
di caratteri modificata alla stringa che si desiderava modificare. In Go accade qualcosa di sostanzialmente analogo. Segue un 
conciso esempio concreto di quanto appena detto. \\

\vspace{0.5cm}
\begin{lstlisting}[frame=single]
var str : String = "some text";
var tmp : $Char = memory::malloc<Char>(str);

var i : Int = 0;
while (i < tmp.length){
    tmp[i] = uppercase_character(tmp[i]);
    i += 1;
}

str = tmp;
console::println(str);
memory::free<String>(str);
\end{lstlisting}
\vspace{0.5cm}


Si noti come in questo caso, dato che la stringa ora conserva un riferimento ad un’area di memoria dinamica, allocata su heap 
manualmente, si rende dunque necessario effettuare una deallocazione manuale al termine dell’utilizzo con la funzione free. \\