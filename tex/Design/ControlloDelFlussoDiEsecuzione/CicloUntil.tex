\subsubsection{Ciclo until}
Il ciclo until è un costrutto utilizzato per ripetere una certa porzione di codice finchè una certa condizione booleana rimane falsa. Il corpo del 
ciclo viene eseguito prima di aver controllato la condizione booleana. Si consideri ad esempio il seguente frammento di codice dove è presentato un ciclo 
while a scopo esemplificativo:

\vspace{0.5cm}

\begin{lstlisting}[frame=single]
var i : Int = 0;
until (i > 10) {
    console.println(x);
    i = i + 1;
}
\end{lstlisting}

\vspace{0.5cm}

L’esecuzione di tale ciclo comporta la stampa in console dei numeri da 0 a 10. Così come per il ciclo while, si può dire che un ciclo until è 
composto da condizione e corpo, e che la sua esecuzione avviene secondo il seguente diagramma di flusso (flow-chart).

\vspace{0.5cm}

\begin{figure}[h]
    \centering
    \begin{tikzpicture}[ 
        auto,
        decision/.style={diamond, draw=black, thick, text width=2cm, text centered, inner sep=1pt},
        block/.style={rectangle, draw=black, thick, text width=3cm, text centered, minimum height=2em},
        line/.style={draw, thick, -latex'},
        node distance=2cm
    ]

    % Place nodes
    \node [block] (init) {ingresso};
    \node [block, below=of init] (exec) {esecuzione del\\corpo del ciclo};
    \node [decision, below=of exec] (cond) {la\\condizione\\è vera?};
    \node [block, below=of cond] (end) {uscita};

    % Draw paths
    \path [line] (init) -- (exec);
    \path [line] (exec) -- (cond);
    \path [line] (cond.east) -- ++(1,0) |- node[near start, right] {Si} (end.east);
    \path [line] (cond.west) -- ++(-1,0) |- node[near start, left] {No} (exec);

    \end{tikzpicture}
    \caption{Diagramma del ciclo until}
    \label{fig:flowchart2}
\end{figure}