\subsubsection{Break e continue}
La keyword \texttt{break} consente di provocare l’interruzione anticipata da un ciclo. Essa è pensata per essere utilizzata assieme ad un branch condizionale 
che monitori una qualche condizione eccezionale che richiede l’interruzione immediata del ciclo.

Ad esempio, si analizzi il seguente frammento di codice che illustra un ciclo while:

\vspace{0.5cm}
\begin{lstlisting}[frame=single]
while (true) {
    var x = math.random<Int>(-5,5);
    if (x % 3 == 0) {
        break;
    }
    console.println(x);
}
\end{lstlisting}
\vspace{0.5cm}

Tale ciclo presenta una condizione da controllare prima della stampa in console, ovvero la non divisibilità per 3 del valore contenuto nella variabile x. 
Qualora tale condizione si verificasse si uscirebbe immediatamente dal ciclo, altrimenti si procederebbe con la stampa in console del valore di x. \\

La keyword \texttt{continue}, similmente alla keyword break, consente di alterare il flusso di esecuzione di un ciclo. Anzichè provocarne 
l’interruzione anticipata, essa consente di saltare l’esecuzione del codice rimanente all’interno del corpo e passare direttamente alla 
successiva iterazione. Ad esempio, si consideri il seguente frammento di codice:

\vspace{0.5cm}
\begin{lstlisting}[frame=single]
var x : Int = 0;
while (x < 10) {
    if (x % 3 == 0) {
        continue;
    }
    console.println(x);
    x = x + 1;
}
\end{lstlisting}
\vspace{0.5cm}

L’esecuzione di questo ciclo avrà come effetto la stampa in console dei numeri da 0 a 9 che non sono divisibili per 3, 
in quanto ad ogni iterazione dove x avrà valore divisibile per 3,  la stampa in console sarà saltata e si 
proseguirà all’iterazione seguente.