\subsubsection{Ciclo while}
Il ciclo while è un costrutto utilizzato per ripetere una certa porzione di codice finché una certa condizione booleana rimane vera. Il corpo del ciclo viene eseguito solo 
dopo aver controllato la condizione booleana. Si consideri ad esempio il seguente frammento di codice dove è presentato un ciclo while a scopo esemplificativo:

\vspace{0.5cm}

\begin{lstlisting}[frame=single]
var i : Int = 0;
while (i < 10) {
    console.println(x);
    i = i + 1;
}
\end{lstlisting}

\vspace{0.5cm}

L’esecuzione di tale ciclo comporta la stampa in console dei numeri da 0 a 9. Più in generale, si può dire che un ciclo while è composto da condizione e 
corpo, e che la sua esecuzione avviene secondo il seguente diagramma di flusso (flow-chart). \\

\vspace{0.5cm}
\begin{figure}[h]
    \centering
    \begin{tikzpicture}[ 
        auto,
        decision/.style={diamond, draw=black, thick, text width=2cm, text centered, inner sep=1pt},
        block/.style={rectangle, draw=black, thick, text width=3cm, text centered, minimum height=2em},
        line/.style={draw, thick, -latex'},
        node distance=2cm
    ]

    % Place nodes
    \node [block] (init) {ingresso};
    \node [decision, below=of init] (cond) {la\\condizione\\è vera?};
    \node [block, below left=2cm and 1cm of cond] (exec) {esecuzione del\\corpo del ciclo};
    \node [block, below right=2cm and 1cm of cond] (end) {uscita};

    % Draw paths
    \path [line] (init) -- (cond);
    \path [line] (cond.west)  -| node[left, yshift=-1.5cm] {Si} (exec.north);
    \path [line] (cond.east) -- ++(1.8,0) -- node[right] {No} (end.north);
    \path [line] (exec) -- ++(0,-2) -| ++(3.5,0) -- (cond.south);

    \end{tikzpicture}
    \caption{Diagramma di flusso del ciclo while}
    \label{fig:flowchart_while_loop}
\end{figure}