\subsubsection{Branch condizionali}
Un branch condizionale, anche chiamato "if-statement”, è un costrutto che consente di eseguire una porzione di codice solo se una certa conzione booleana è vera. Ad esmpio si consideri 
il seguente frammento di codice che illustra l’utilizzo di un if-statement.


\vspace{0.5cm}

\begin{lstlisting}[frame=single]
var x : Int = math::random<Int>(0,10);
if (x % 2 == 0) {
    console::println("x is even");
}
\end{lstlisting}


\vspace{0.5cm}

In questo codice, l’istruzione \texttt{console::println("x is even")} sarà eseguita solo nel caso in cui il valore numerico intero attualmente contenuto nella variabile x sarà pari. 

É possibile aggiungere un blocco di codice da eseguire nel caso in cui la condizione sia falsa utilizzando la keyword else. Ad esempio è possibile stampare del testo che 
informi l’utente del fatto che la variabile x contiene un valore dispari.

\vspace{0.5cm}

\begin{lstlisting}[frame=single]
var x : Int = math::random<Int>(0,10);
if (x % 2 == 0) {
    console::println("x is even");
}
else {
    console::println("x is odd");
}    
\end{lstlisting}


\vspace{0.5cm}

In Basalt l’indentazione non è rilevante, per cui, se lo si preferisce, è accettato (anche se sconsigliato) disporre la keyword else sulla stessa riga della 
chiusura della parentesi graffa relativa al blocco di codice da eseguire nel caso in cui la condizione sia vera.