\subsection{Funzioni}
Una funzione è un blocco di codice riutilizzabile molteplici volte la cui esecuzione può venire 
influenzata dal valore di eventuali parametri, qual’ora presenti, detti argomenti. Ogni argomento 
ha un nome con il quale è possibile riferirsi ad esso e da un tipo. \\

Una funzione può “restituire” un valore al chiamante, ed il tipo di tale valore di ritorno 
è noto a tempo di compilazione qualora presente. È anche possibile che una funzione 
non restituisca nulla al chiamante, in tal caso si parla di procedura o di routine. \\

La definizione di una funzione in Basalt avviene utilizzando la parola chiave func, seguita 
dal nome della funzione, da un eventuale elenco non vuoto di parametri 
formali di tipo separati da virgole e racchiusi tra parentesi angolari, 
da un elenco eventualmente anche vuoto di argomenti, separati da virgole e 
racchiusi in parentesi tonde, da un eventuale tipo di ritorno preceduto 
dal simbolo “->” e infine da un blocco di codice delimitato da parentesi graffe 
chiamato corpo della funzione. \\ 

Di seguito è riportato un esempio di definizione di una funzione “max” che accetta 
due argomenti di tipo Int e restituisce un Int corrispondente al valore maggiore tra loro.

\vspace{0.5cm}
\begin{lstlisting}[frame=single]
func max(first : Int, second : Int) -> Int {
    if (first > second) {
        return first; 
    }
    
    else {
        return second;
    }
}
\end{lstlisting}
\vspace{0.5cm}

È possibile invocare questa funzione passando una qualunque coppia di valori interi, 
e più in generale è possibile invocare una funzione con una lista di valori i cui tipi 
siano compatibili per assegnazione ai tipi degli argomenti di tale funzione. La sintassi 
della chiamata a funzione in Basalt è equivalente a quanto si può vedere in linguaggi 
come C, C++ e Go e consta del nome della funzione, seguito da una eventuale lista di 
parametri attuali di tipo racchiusa in parentesi angolari e separati da virgole e 
da una lista di valori, da assegnare agli argomeni, racchiusi tra parentesi tonde 
e separati da virgole.

\vspace{0.5cm}
\begin{lstlisting}[frame=single]
var n : Int = max(5,6);
\end{lstlisting}
\vspace{0.5cm}
\newpage

\subfile{Overloading}
\subfile{Extern}