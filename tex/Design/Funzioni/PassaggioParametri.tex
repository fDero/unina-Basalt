\subsubsection{Passaggio parametri}
Il passaggio di parametri in Basalt è sempre per valore. Questo significa che, quando si passa un parametro ad una funzione, 
viene creata una copia del valore del parametro passato. Questo comportamento è in contrasto con il passaggio per riferimento, in cui la funzione 
riceve un riferimento alla variabile passata, permettendo alla funzione di modificare direttamente il valore della variabile passata. \\

Per poter accedere in scrittura ad una qualsiasi area di memoria esterna alla funzione, è necessario passare un puntatore (scalare o vettoriale)
a tale area di memoria, così simulando un passaggio parametri per riferimento. Tale modo di operare è prassi consolidata in molti altri linguaggi 
di programmazione quali Go, C, Zig non solo.\\

\vspace{0.5cm}
\begin{lstlisting}[frame=single]
func f(x : MyStruct, y : #MyStruct, z : ##MyStruct) {
    x.field = 10;
    y.field = 20;   // modifica osservabile dall'esterno
    z.field = 30;   // modifica osservabile dall'esterno

    var myStruct : MyStruct;

    x = myStruct;
    y = &myStruct;
    #z = &myStruct; // modifica osservabile dall'esterno
    z = &y;
}
\end{lstlisting}
\vspace{0.5cm}

Così come accade in C, modificare un puntatore all'interno di una funzione non comporta alcuna modifica all'indirizzo di memoria puntato. 
Ciò che si sta modificando è solo la copia locale dell'indirizzo. Modificare field di un puntatore invece, comporta una modifica effettiva
all'indirizzo di memoria puntato, così come modificarne il valore dereferenziato. Analogamente a quanto appena visto per i puntatori scalari 
avviene per array e puntatori vettoriali, i quali vengono copiati. \\

\vspace{0.5cm}
\begin{lstlisting}[frame=single]
func f(x : [10]MyStruct, y : #[10]MyStruct, z : $MyStruct) {
    x[0] = 10;
    (#y)[0] = 20;   // modifica osservabile dall'esterno
    z[0] = 30;      // modifica osservabile dall'esterno

    var array : [10]MyStruct;

    x = array;
    y = &array;
    #y = array;     // modifica osservabile dall'esterno
    z = &array;
}
\end{lstlisting}
\vspace{0.5cm}

\newpage
