\subsubsection{Overloading}
Con overloading delle funzioni, si intende la possibilità di fornire all’interno 
di un programma, eventualmente anche all’interno dello stesso package, multiple 
definizioni di funzioni aventi lo stesso nome, a patto gli argomenti differiscano 
per quantità o per il tipo di almeno uno di essi. Due definzioni di funzioni 
aventi lo stesso nome si dicono una overload dell’altra, l’insieme di tutti 
gli overload di una certa funzione viene chiamato overload-set. \\

Analizziamo per esempio questi due overload per la funzione \texttt{max}, esse sono 
validi overload in quanto pur avendo lo stesso numero di argomenti, tali 
argomenti hanno tipo distinto, e dunque in sede di chiamata il compilatore 
analizzando i tipi degli argomenti concreti della chiamata sarà in grado, 
partendo da essi, di selezionare l’overload adatto. \\

\vspace{0.5cm}
\begin{lstlisting}[frame=single]
func max(first : Int, second : Int) -> Int { 
    if (first > second) {

        return first; 
    }
    else {
        return second;
    }
}

func max(first : Float, second : Float) -> Float {
    if (first > second) {

        return first; 
    }
    else {
        return second;
    }
}

func max(first : Number, second : Number) -> Number {
    if (first > second) {

        return first; 
    }
    else {
        return second;
    }
}
\end{lstlisting}
\vspace{0.5cm}

\newpage

nel caso poi in cui esistano due overload entrambi validi, sarà scelto l’overload rienuto 
più specifico, applicando queste politiche di selezione e filtraggio tra gli overload trovati.

\begin{itemize}
    \item{
        un overload non generico viene sempre preferito rispetto ad un overload generico, il 
        numero di parametri di tipo non è rilevante.
    }
    \item{
        un overload viene preferito ad un altro se i suoi parametri di tipo compaiono 
        meno volte nei tipi dei suoi argomenti
    }
    \item{
        un overload viene preferito ad un altro se i suoi argomenti hanno tipi più complicati, 
        ovvero hanno più parametri di tipo, oppure tali parametri di tipo sono a loro volta, 
        ricorsivamente, più complicati. 
    }
    \item{
        un overload viene preferito ad un altro se tra i tipi dei suoi argomenti 
        compaiono meno volte tipi definiti come union e/o union anonime.
    }
    \item{
        un overload viene preferito ad un altro se il numero totale di casi comperti dalle 
        union che compaiono tra i suoi argomenti è minore.
    }
    \item{
        un overload viene preferito ad un altro se tra i tipi dei suoi argomenti e/o tra i 
        parametri di tipo di questi ultimi vi sono meno conversioni di tipo
    }
\end{itemize}

É possibile analizzare tutti gli aspetti appena elencati durante una fase apposita 
di preprocessing. Le funzioni quindi vengono valutate per la loro specificità 
prima che la ricerca dell’overload più appropriato abbia inizio. Tale ricerca, 
nota come overload-resolution è sostanzialmente un’operazione di scarto degli 
overload non compatibili con i tipi della chiamata, procedendo in ordine di specificità, 
il cui ordine è stato già stabilito a priori. \\

Qualora, alla fine della fase di scarto 
degli overload incompatibili, siano state trovate due definizioni ugualmente specifiche 
allora la chiamata viene definita ambigua e ciò porta ad un errore a tempo di compilazione. \\

in particolare, applicando la regola numero 3, si è in grado di affermare 
che tra i seguenti due overload di seguito riportati, vi è uno più specifico 
dell’altro ed esso è, come è lecito aspettarsi, l’overload che 
accetta un argomento di tipo \texttt{List<List<T>\>>}

\vspace{0.5cm}
\begin{table}[h]
    \centering
        \begin{tabularx}{\textwidth}{|b|h|} \hline
            \cheader{IDENTIFICATIVO}                  & \cheader{SPECIFICITÀ}  \\ \hline
            \texttt{func f<T>(x : List<List<T>\>>)}   & \#1 \\ \hline
            \texttt{func f<T>(x : List<T>)}           & \#2 \\ \hline
            \texttt{func f<T>(x : T)}                 & \#3 \\ \hline
        \end{tabularx}
    \caption{Comparazione specificità di overload per la funzione \texttt{f}}
\end{table}
\vspace{0.5cm}

\newpage

Di seguito è stata riportata una tabella dove sono state riportate alcuni overload 
della funzione \texttt{max}, raggruppate per specificità ed elencate dalla più 
alla meno specifica. 

\vspace{0.5cm}
\begin{table}[h]
    \centering
        \begin{tabularx}{\textwidth}{|X|c|} \hline
            \cheader{IDENTIFICATIVO}                                      & \header{SPECIFICITÀ}  \\ \hline
            \texttt{func max(first : Int, second : Int) -> Int}           & \#1 \\ \hline
            \texttt{func max(first : Float, second : Float) -> Float}     & \#1 \\ \hline
            \texttt{func max(first : Number, second : Int) -> Number}     & \#2 \\ \hline
            \texttt{func max(first : Int, second : Number) -> Number}     & \#2 \\ \hline
            \texttt{func max(first : Number, second : Float) -> Number}   & \#2 \\ \hline
            \texttt{func max(first : Float, second : Number) -> Number}   & \#2 \\ \hline
            \texttt{func max(first : Number, second : Number) -> Number}  & \#3 \\ \hline
            \texttt{func max<T>(first : Number, second : T) -> Number}    & \#4 \\ \hline
            \texttt{func max<T>(first : T, second : Number) -> Number}    & \#4 \\ \hline
            \texttt{func max<T>(first : T, second : T) -> T}              & \#5 \\ \hline
        \end{tabularx}
    \caption{Comparazione specificità di overload per la funzione \texttt{max}}
\end{table}
\vspace{0.5cm}

Durante la fase di overload-resolution per una ipotetica chiamata alla funzione \texttt{max}
con argomenti \texttt{Int} e \texttt{Float}, allora vi sarebbero più overload compatibili, 
ma nell’ambito dei soli overload compatibili, solo uno di essi è nettamente più specifico 
di tutti gli altri, ovvero l'overload corrisponente alla seguente firma:

\vspace{0.5cm}
\begin{lstlisting}[frame=single]
func max(first : Number, second : Number) -> Number
\end{lstlisting}
\vspace{0.5cm}

\newpage