\begin{itemize}
    \item{
        un overload non generico viene sempre preferito rispetto ad un overload generico, il 
        numero di parametri di tipo non è rilevante
    }
    \item{
        un overload viene preferito ad un altro se i suoi parametri di tipo compaiono 
        meno volte nei tipi dei suoi argomenti
    }
    \item{
        un overload viene preferito ad un altro se i suoi argomenti hanno tipi più complicati, 
        ovvero hanno più parametri di tipo, oppure tali parametri di tipo sono a loro volta, 
        ricorsivamente, più complicati
    }
    \item{
        un overload viene preferito ad un altro se tra i tipi dei suoi argomenti 
        compaiono meno volte tipi definiti come union (anonime e non)
    }
    \item{
        un overload viene preferito ad un altro se il numero totale di casi coperti dalle 
        union che compaiono tra i suoi argomenti è minore
    }
    \item{
        un overload viene preferito ad un altro se tra i tipi dei suoi argomenti e/o tra i 
        parametri di tipo di questi ultimi vi sono meno conversioni di tipo
    }
\end{itemize}