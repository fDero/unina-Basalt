\subsubsection{Funzioni extern}
Una funzione esterna, in generale, è una funzione che non è definita all'interno della compilation unit in cui viene 
invocata. Nel caso concreto del linguaggio Basalt, essa è dunque una funzione che si desidera invocare all'interno
di un programma Basalt, ma che è definita in un altro linguaggio ad esempio in C o in C++. \\

Il linker, in fase di linking, si occuperà di risolvere il riferimento alla funzione esterna a patto che 
esso abbia a disposizione l'object file contenente la definizione della funzione. Tale object file può 
essere emesso dai compilatori C e C++. \\

Per poter invocare una funzione esterna, è necessario che essa sia dichiarata all'interno del programma Basalt
utilizzando la keyword \textit{extern} al posto della keyword \textit{func}, e rimpiazzando il corpo della funzione
con un simbolo di \texttt{=} seguito da una string literal detta nome-macchina, rappresentante il nome con cui tale 
funzione viene nominata nell'object file che si desidera linkare, seguita da un punto e virgola. \\

Una funzione extern non può essere generica, e non vi possono essere più funzioni esterne associate 
allo stesso nome macchina. \\

\vspace{0.4cm}
\begin{lstlisting}[frame=single]
// C stdlib: double sqrt(double);
extern square_root(value : Float) -> Float = "sqrt";
\end{lstlisting}
\vspace{0.4cm}

Tale design consente di poter effettuare overloading di funzioni esterne, in quanto il nome da utilizzare per
l'overloading è quello con cui essa è stata dichiarata, che è potenzialmente differente dal nome che essa ha
nell'object file, e che è stato invece specificato in coda alla dichiarazione. Di seguito un esempio di quanto detto: \\

\vspace{0.4cm}
\begin{lstlisting}[frame=single]
// C stdlib: int32_t putchar(int32_t);
extern print(char : Char) = "putchar";

func print(str : RawString) {
    var index : Int = 0;
    while (str[index] != '\0') {
        print(str[index]);
        index = index + 1;
    }
}

func print(str : String) {
    var index : Int = 0;
    while (index < len(str)) {
        print(str[index]);
        index = index + 1;
    }
}
\end{lstlisting}
