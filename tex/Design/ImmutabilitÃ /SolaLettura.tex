\subsubsection{Espressioni di sola lettura}
Un'espressioni di sola lettura, è una espressione che restituisce un valore temporaneo preso per copia. Questo 
significa che anche qualora essa fosse mutabile, la modifica non sarebbe osservabile in alcun modo. Un esempio di 
espressione di sola lettura è la chiamata ad una funzione che restituisce un puntatore. Il valore restituito dalla 
funzione è un valore temporaneo, preso per copia, esso si dice essere in sola lettura e non può essere modificato.
Tuttavia non è corretto parlare di immutabilità in quanto ciò che si trova all'area di memoria indirizzata dal 
puntatore potrebbe essere modificato.

\vspace{0.5cm}
\begin{lstlisting}[frame=single]
func get_ptr() -> #Int { return memory::alloc<Int>(); }
var number : Int = 7;

// Errore di compilazione -> modifica di espressione di sola lettura
get_ptr() = &number;

// Ok -> modifica del valore puntato (non immutabile)
#get_ptr() = number;

\end{lstlisting}
\vspace{0.5cm}
