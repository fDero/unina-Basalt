\subsubsection{Costanti}
Una costante è nient'altro che una variabile immutabile. A differenza di linguaggi come Java e Kotlin, la garanzia di immutabilità per le costanti
si estende non solo a loro stesse ma anche ai loro membri se sono struct, agli oggetti da loro puntati se sono puntatori e così via. In linea generale
è possibile affermare che una costante non può essere modificata in nessuna sua parte. La dichiarazione di una costante necessita di una inizializzazione, 
e avviene in accordo alla seguente sintassi: \texttt{const <nome> : <tipo> = <valore>;}

\vspace{0.5cm}
\begin{lstlisting}[frame=single]
const pi : Float = 3.14;
\end{lstlisting}
\vspace{0.5cm}

Ci si potrebbe chiedere quale sia l'utilità di un puntatore costante, dato che il puntatore stesso non può essere modificato così come il valore a cui punta.
In effetti, l'utilità di un puntatore costante è limitata, specialmente se paragonata con quanto accade in C e C++, però la semantica della keyword \texttt{const}
è molto differente. Un puntatore costante in Basalt può essere usato ad esempio come un alias, per tale scenario è riportato il seguente esempio:

\vspace{0.5cm}
\begin{lstlisting}[frame=single]
struct Job {
    var name : String;
    var salary : Int;
}

struct Person {
    var name : String;
    var job : Job;
}

var person : Person = get_person(); 
    // Si assuma una funzione del genere esista

const job : #Job = &person.job;
\end{lstlisting}
\vspace{0.5cm}

In questo caso il puntatore \texttt{job} è costante, e non è possibile modificare il puntatore direttamente, così come non è possibile modificare l'oggetto puntato 
accedendo ad esso tramite dereferenziazione del puntatore. É però ancora possibile modificare l'oggetto puntato accedendo ad esso mediante la variabile \texttt{person}. \\

In linguaggi come Java e Kotlin, le costanti (o \texttt{final} in Java) offrono garanzie di immutabilità solo per il riferimento, 
ma non per l'oggetto puntato. Questo significa che se si dichiara una costante di tipo \texttt{List}, è possibile modificare la lista
aggiungendo o rimuovendo elementi, ma non è possibile assegnare un nuovo oggetto alla variabile costante. In Basalt, invece, una costante
di tipo \texttt{List} non può essere modificata in nessun modo, ne aggiungendo o rimuovendo elementi. \\

L'approccio di Basalt è pertanto più simile a quello di Go, dove le costanti sono immutabili in tutto e per tutto, e non solo per il riferimento. \\

In C++ il cocetto più simile a quello di costante in Basalt è quello di \texttt{constexpr}, che però nel caso di C++ è ristretto alle sole espressioni
valutabili a tempo di compilazione. \\

Quando possibile, il compilatore cercherà di effettuare delle ottimizzazioni sulle costanti, ad esempio sostituendo le loro occorrenze
all'interno del codice con il loro valore, quando noto a tempo di compilazione.