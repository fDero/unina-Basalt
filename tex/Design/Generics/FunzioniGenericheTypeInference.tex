\subsubsection{Algoritmo di type-inferece}
Per funzioni generiche, è possibile non specificare espressamente dei parametri 
attuali di tipo e lasciare che sia Basalt a dedurli dal contesto. 
L’operazione di deduzione dei parametri attuali di tipo a partire dal 
contesto è detta type-inference. \\

Supponiamo ad esempio di voler chiamare la funzione max, ma 
senza specificare un parametro attuale. Per usare la type-inference, 
è possibile omettere interamente la sezione dei parametri attuali di 
tipo ed utilizzare max in maniera sintatticamente analoga a come 
lo si farebbe per una funzione non generica. \\

\vspace{0.5cm}
\begin{lstlisting}[frame=single]
var x : Int = max(3, 5);
var y : Float = max(3.14, 5.17);
\end{lstlisting}
\vspace{0.5cm}

Consideriamo poi le due seguenti chiamate, dove in entrambi i casi, 
i valori degli argomenti sono di tipo diverso, ma il tipo dichiarato 
degli argomenti è lo stesso parametro formale di tipo. \\

\vspace{0.5cm}
\begin{lstlisting}[frame=single]
var int_value : Int = 3;
var float_value : Float = 3.14;
var number_value : Number = 5;

var z1 : Number = max(int_value, number_value);
var z2 : Number = max(int_value, float_value);
\end{lstlisting}
\vspace{0.5cm}
 
Per la chiamata \texttt{max(int\_value, number\_value)} la type-inference si trova a 
dover trovare un parametro attuale di tipo tale che ad esso possano 
venire assegnati sia valori di tipo \texttt{Int}, che valori di tipo Number. Tra 
questi tipi candidati, Number rispetta entrambi i vincoli e quindi viene scelto. \\

Per la chiamata \texttt{max(int\_value, float\_value)} la type-inference si trova a 
dover trovare un parametro attuale di tipo tale che ad esso 
possano venire assegnati sia valori di tipo \texttt{Int}, che valori 
di tipo \texttt{Float}. Nessuno dei due candidati rispetta tutti i vincoli, 
allora il parametro attuale di tipo sarà un tipo fittizio creato 
appositamente come union senza nome definita come la union dei candidati, 
in questo caso, \texttt{Int} e \texttt{Float}.