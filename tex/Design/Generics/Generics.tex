\subsection{Generics}
Con generics, ci si riferisce a parametri formali di tipo applicabili a definzioni 
di tipi, funzioni ed alias all'intero del linguaggio di programmazione Basalt. \\

Tali definizioni diventano così parametriche, vengono dunque sottoposte a un type-checking ridotto
e vengono utilizzate come dei template per generare definizioni concrete (non-parametriche) al momento
del loro utilizzo istanziandole con i valori concreti di tali parametri di tipo. \\

(Tale approccio all'implementazione dei generics è detto "reificazione"
ed è usato da linguaggi come ad esempio \texttt{C++} e \texttt{C\#}, al contrario, linguaggi come
\texttt{Java} e \texttt{Kotlin} usano un approccio detto "erasure").


\subfile{StructGeneriche}
\subfile{UnionGeneriche}
\subfile{FunzioniGeneriche}
\subfile{FunzioniGenericheTypeInference}