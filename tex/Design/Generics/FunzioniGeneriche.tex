\subsubsection{Funzioni generiche}
Come detto in precedenza, le funzioni in Basalt possono essere generiche. 
La definizione di una funzione generica prevede la presenza di una lista 
non vuota di parametri formali di tipo, separati da virgole e racchiusi tra 
parentesi angolari, che precede la lista di argomenti della funzione. \\

È possibile definire una funzione generica che restituisca il massimo tra due 
valori il cui tipo è specificato al momento della chiamata.

\vspace{0.5cm}
\begin{lstlisting}[frame=single]
func max<T>(first : T, second : T) -> T {
    if (first > second) {
        return first; 
    }
    else {
        return second;
    }
}
\end{lstlisting}
\vspace{0.5cm}


Tale definizione sarà istanziata all’occorrenza e sarà possibile 
istanziare tale funzione solo per tipi confrontabili con l’operatore ‘>’. 
Istanziare tale funzione con tipi non confrontabili genererà un errore a tempo 
di compilazione. \\

In sede di chiamata a funzione, è possibile specificare dei parametri attuali di 
tipo per la funzione stessa dopo il nome e prima dell’elenco degli argomenti, 
elencandoli separati da virgole e racchiusi tra parentesi angolari. \\

Ad esempio per la funzione add è possibile usare sia \texttt{Int}, che \texttt{Float}.

\vspace{0.5cm}
\begin{lstlisting}[frame=single]
var x : Int = max<Int>(3, 5);
var y : Float = max<Float>(3.14, 5.17);
\end{lstlisting}
\vspace{0.5cm}

Così come è lecito aspettarsi, è possibile utilizzare union, eventualmente anche anonime,
come parametri attuali di tipo per funzioni generiche. \\

\vspace{0.5cm}
\begin{lstlisting}[frame=single]
func f<T>(x : T, y : T) { 
    /* no-op */ 
}
\end{lstlisting}
\begin{lstlisting}[frame=single]
f<Int|String>(3, "Hello");
\end{lstlisting}
\vspace{0.5cm}