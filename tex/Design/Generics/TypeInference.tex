\subsubsection{Type-inference}
Per funzioni generiche è possibile non specificare espressamente dei parametri 
attuali di tipo e lasciare che sia Basalt a dedurli dal contesto. 
L’operazione di deduzione dei parametri attuali di tipo a partire dal 
contesto è detta type-inference. \\

In sede di chiamata a funzione, se tale funzione è generica e non vengono specificati
i tipi dei parametri attuali, Basalt analizzerà uno ad uno gli argomenti forniti e cercherà 
di risolvere i vincoli di assegnabilità. \\

Per ogni argomento, Basalt analizzerà il tipo dichiarato in sede di definizione della funzione e 
lo comparerà con il tipo dell'argomento. Se il tipo dell'argomento è generico, e tale generico 
viene usato una sola volta, allora Basalt assumerà che il parametro attuale di tipo corrispondente sia 
proprio il tipo dichiarato dell'argomento. \\

Qualora il tipo dell'argomento sia un generico ed è già stato usato in precedenza, Basalt metterà in discussione
la sua iniziale assunzione, qualora sia possibile, deducendo eventualmente un tipo più generale capace di ricevere assegnamenti di 
espressioni sia del tipo corrispondente alla sua precedente assunzione, sia di tipo corrispondente al tipo concreto 
dell'argomento. \\

In certi casi, è impossibile mettere in discussione l'assuzione fatta dalla type-inference, ad esempio, si consideri 
la seguente funzione generica:

\vspace{0.5cm}
\begin{lstlisting}[frame=single]
struct Wrapper<T> {
    var value : T;
}

func f<T>(x : Wrapper<T>, y : T) { 
    /* no-op */ 
}
\end{lstlisting}
\vspace{0.5cm}

Allora, la chiamata a funzione \texttt{f} con argomenti \texttt{Wrapper<Int>} e \texttt{Float} non sarà 
risolta su questo overload in quanto la type-inference prima assumerà che \texttt{T} sia \texttt{Int}, successivamente 
però, tenterà di aggiornare la sua assunzione precedente e dedurrà che \texttt{T} debba essere \texttt{Int|Float},
ma ciò renderà impossibile l'assegnamento di \texttt{Wrapper<Int>} a \texttt{Wrapper<T>}. \\

Si tenga presente che in Basalt è sempre possibile trovare un tipo \texttt{T} tale da ricevere assegnamenti da parte 
di espressioni di due tipi concreti noti \texttt{X1} ed \texttt{X2}, e tale tipo è la union anonima \texttt{X1 | X2}. Ad 
esempio sarà possibile chiamare la seguente implementazione della funzione \texttt{g} con argomenti arbitrari e 
la type-inference risolverà il tipo \texttt{T} come la union dei tipi concreti degli argomenti della chiamata. \\

\vspace{0.5cm}
\begin{lstlisting}[frame=single]
func g<T>(x : T, y : T) {  /* no-op */ }
\end{lstlisting}
\vspace{0.5cm}

\newpage

Il seguente programma Basalt illustra l'utilizzo della type-inference per 
funzioni generiche:

\vspace{0.5cm}
\begin{lstlisting}[frame=single]
package type_inference_demo;

struct Wrapper<T> {
    value : T;
}

func f<T>(w : Wrapper<T>, t : T) { 
    /* no-op */ 
} 

func g<T>(x : T, y : T) { 
    /* no-op */ 
}

func h<T>(x : T, y : Wrapper<T>) { 
    /* no-op */ 
} 

func demo() {
    var w1 : Wrapper<Float>;
    var w2 : Wrapper<Float|String>;
        
    f(w1, 3.2);     
    // T = Float
    
    f(w2, "Hello"); 
    // T = Float|String

    // f(w1, "Hello");  
    // T = Float|String 
    // ERROR: Wrapper<Float|String> = Wrapper<Float> not allowed

    g(3.2, 3.2);     
    // T = Float
    
    g(3.2, "Hello"); 
    // T = Float|String

    h(3.2, w2);      
    // T = Float|String
}
\end{lstlisting}
\vspace{0.5cm}