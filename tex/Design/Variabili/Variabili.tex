\subsection{Variabili}
Le variabili sono dei contenitori logici capaci di contenere dei valori decisi a tempo di esecuzione. Ci si potrà riferire 
al valore contenuto in un dato istante di tempo da una variabile o da una costante utilizzando il suo nome. Tramite un apposito 
costrutto detto assegnazione, è possibile riassegnare il valore di una variabile, ciò non è però possibile per una costante, la quale 
una volta dichiarata non potrà mai cambiare valore. 

\subsubsection{Dichiarazione di variabili}
La dichiarazione di una variabile può avvenire con inizializzazione o senza, laddove un valore di inizializzazione sia mancante il valore 
di tale variabile sarà casuale. Ci si aspetta che in tale scenario un valore venga poi assegnato in un secondo momento. Qualunque sia la 
tipologia di dichiarazione scelta, essa deve essere introdotta dalla keyword var, seguida dal nome della variabile, dai due punti e dal tipo 
di tale variabile.

\vspace{0.5cm}

\begin{lstlisting}[frame=single]
var x : Int = 6;
var y : Int;
\end{lstlisting}

\subsubsection{Scoping di una variabile}
Una variabile in basalt esiste nello scope della funzione, del ciclo o più in generale del blocco di codice in cui è stata dichiarata. Ciò 
significa che una variabile dichiarata all’interno di un blocco di codice non sarà accessibile al di fuori di esso.

\subsubsection{Shadowing}
Con showowing, si intende la possibilità di, all'interno di un blocco di codice innestato, dichiarare una variabile con un nome già usato
in un blocco esterno. In tal caso, la variabile dichiarata all'interno del blocco innestato avrà precedenza su quella dichiarata nel blocco e 
sarà impossibile riferirsi alla variabile dichiarata nel blocco esterno con il suo nome.

\subsubsection{Lifetime di una variabile}
Al termine dell'esecuzione del blocco di codice corrente, la variabile verrà deallocata e il suo valore non sarà più accessibile. Questo comportamento
è ben distinto da quanto accade in Java, Kotlin ed altri linguaggi garbage collected, dove invece la variabile viene deallocata solo quando il garbage
collector la riconosce come non più utilizzata.

\vspace{0.5cm}