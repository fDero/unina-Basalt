\subsection{Variabili}
Le variabili sono dei contenitori logici capaci di contenere dei valori decisi a tempo di esecuzione. Ci si potrà riferire 
al valore contenuto in un dato istante di tempo da una variabile o da una costante utilizzando il suo nome. Tramite un apposito 
costrutto detto assegnazione, è possibile riassegnare il valore di una variabile.

\subsubsection{Dichiarazione di variabili}
La dichiarazione di una variabile può avvenire con inizializzazione o senza, laddove un valore di inizializzazione sia mancante il valore 
di tale variabile sarà casuale. Ci si aspetta che in tale scenario un valore venga poi assegnato in un secondo momento. Qualunque sia la 
tipologia di dichiarazione scelta, essa deve essere introdotta dalla keyword var, seguita dal nome della variabile, dai due punti e dal tipo 
di tale variabile.

\vspace{0.5cm}

\begin{lstlisting}[frame=single]
var x : Int = 6;
var y : Int;
\end{lstlisting}

\subsubsection{Scoping di una variabile}
Una variabile in Basalt esiste nello scope della funzione, del ciclo o più in generale del blocco di codice in cui è stata dichiarata. Ciò 
significa che una variabile dichiarata all’interno di un blocco di codice non sarà accessibile al di fuori di esso.

\subsubsection{Shadowing}
Con shadowing, si intende la possibilità di, all'interno di un blocco di codice innestato, dichiarare una variabile con un nome già usato
in un blocco esterno, oscurandola. Numerosi linguaggi supportano lo shadowing delle variabili, ma Basalt non è tra questi. Si è ritenuto 
che tale funzionalità potesse portare a confusione e a codice di difficile comprensione, pertanto tentare di oscurare una variabile già definita
in un blocco esterno causerà un errore a tempo di compilazione.

\subsubsection{Deallocazione delle variabili}
Al termine dell'esecuzione del blocco di codice corrente, l'area di memoria occupata dalle variabili dichiarate in esso verrà automaticamente
deallocata. Nel caso in cui tali variabili abbiano per valore un indirizzo di memoria dinamica allocato in precedenza, la deallocazione di tale blocco 
\textbf{non} è automatica, e spetterà dunque al programmatore deallocare tale blocco di memoria manualmente.

\vspace{0.5cm}