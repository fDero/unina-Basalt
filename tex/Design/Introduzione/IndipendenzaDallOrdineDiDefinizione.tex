\subsubsection{Indipendenza dall’ordine di definizione}
Così come in Java, Rust e Go, e a differenza di C e C++, Basalt prevede che ogni definizione possa essere spostata in qualunque punto di un file sorgente o addirittura migrata in 
un altro file sorgente dello stesso package senza compromettere la correttezza del programma. In altri termini, in Basalt ogni definizione è accessibile non solo dalle definzioni che 
la succedono ma anche da quelle che la precedono. \\

L'indipendenza dall'ordine di definizione in un linguaggio di programmazione semplifica notevolmente il refactoring e l'utilizzo del codice. Il programmatore può riorganizzare e ottimizzare 
il codice senza dover preoccuparsi di errori di compilazione dovuti a riferimenti non ancora definiti. Questo favorisce una maggiore modularità e facilita il mantenimento del codice, poiché 
le modifiche possono essere apportate in modo più flessibile e incrementale. Inoltre, consente di migliorare la leggibilità del codice, organizzandolo in modo logico piuttosto che cronologico. \\

In un contesto di sviluppo collaborativo, questa caratteristica è particolarmente vantaggiosa, poiché diversi sviluppatori possono lavorare su parti diverse del codice senza doversi coordinare 
strettamente sull'ordine delle definizioni. Ciò riduce i conflitti di merge e accelera il processo di sviluppo. Anche l'aggiunta di nuove funzionalità o la correzione di bug risulta più semplice, 
in quanto le nuove definizioni possono essere inserite esattamente dove hanno più senso logico, senza dover riscrivere o spostare altre parti del codice esistente. \\

Ciò significa che il seguente codice è valido. Il compilatore è capace di risolvere correttamente il riferimento alla 
funzione \texttt{sum} anche se essa è definita dopo il suo primo utilizzo (ovvero la chiamata avvenuta nella funzione \texttt{main}). \\

\begin{lstlisting}[frame=single]
package main;

func main() {
    var result : Int = sum(3, 5);
    console::print("The sum of 3 and 5 is: ");
    console::println(result);
}

func sum(a: Int, b: Int) -> Int {
    return a + b;
}
\end{lstlisting}