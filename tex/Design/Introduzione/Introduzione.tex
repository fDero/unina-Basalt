\subsection{Introduzione}
Basalt nasce con l'intenzione di offrire un'alternativa moderna a linguaggi di programmazione 
consolidati come C e C++. Nonostante questi ultimi siano ancora ben lontani dall'essere considerati 
obsoleti, è innegabile che comincino a mostrare i segni del tempo se
paragonati con linguaggi moderni quali Go, Rust, Zig, Odin o Carbon. \\

L’obiettivo di Basalt è quello di essere semplice e minimale, faile da imparare, rimanendo 
al tempo stesso un linguaggio di basso livello e pertanto con gestione manuale della memoria. 
Basalt pone l'ergonomia al centro di ogni scelta di design, cercando di ridurre al minimo il tempo
speso dal programmatore a correggere errori banali o scrivere codice banale e ripetitivo. \\

Basalt, così come i sopra citati Go, Rust, Zig, Odin o Carbon, i quali saranno usati come 
termini di paragone per tutto il resto del capitolo, non adotta il paradigma ad oggetti. La 
nuova tendenza nei linguaggi di programmazione sembra essere di abbandonare il paradigma ad 
oggetti puro, offrendone una versione fortemente rivisitata o addirittura eliminandolo del tutto. Nel 
caso di Basalt, il ruolo operativamente ricoperto dalle interfacce è stato preso dalle union (sum-types),
le quali possono essere usate per offrire funzionalità simil-polimorfiche mediante sostanziali integrazioni
con le altre features del linguaggio. \\

Basalt, a differenza del C, è provvisto di un sistema di tipi più avanzato, che permette di
passare array statici o dinamici alle funzioni senza perdere informazioni riguardanti la dimensione di 
questi ultimi. \\

Basalt offre oltretutto pieno supporto alla programmazione generica, implementata mediante reificazione
a tempo di compilazione (così come C++), e non mediante erasure, come ad esempio Java o Kotlin. Il supporto alla 
programmazione generica è stata una tra le prime features ad essere state implementate, ed ha guidato
molte delle scelte di design del linguaggio. \\ 



\newpage
\subfile{ConfrontoConAltriLinguaggi} \newpage
\subfile{StrutturaDiUnProgrammaBasalt} \newpage
\subfile{IndipendenzaDallOrdineDiDefinizione} \newpage