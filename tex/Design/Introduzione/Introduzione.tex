\subsection{Introduzione}
Basalt nasce con l'intenzione di offrire una alternativa moderna a linguaggi di programmazione consolidati come C e C++. Nonostante 
questi linguaggi siano ancora ben lontani dall'essere considerati obsoleti, è innegabile che comincino a mostrare i segni del tempo. \\

L'obiettivo di Basalt è quello di superare alcune delle rigidità e mancanze del linguaggio C, proponendosi come un linguaggio più flessibile, 
espressivo ed ergonomico, mirando ad equipaggiare i programmatori con strumenti di programmazione all'avanguardia, tipici dei linguaggi di più alto livello. Basalt 
offre infatti strumenti avanzati quali riflessione ed introspezione a tempo di compilazione, supporto alla programmazione generica e un framework di unit-testing 
integrato direttamente nel linguaggio. \\

Basalt, così come Go, Rust, Zig, Odin e molti altri linguaggi di basso livello di nuova concezione, non è un linguaggio orientato agli oggetti. La mancata adozione del
paradigma della programmazione orientata agli oggetti in tutti questi linguaggi, e in particolare in Basalt, non comporta una riduzione dell’espressività e della modularità
del codice. Anzi, ha consentito scelte innovative di language design che hanno portato a soluzioni più moderne ed eleganti. Queste scelte hanno introdotto costrutti che
risolvono gli stessi problemi della programmazione orientata agli oggetti, ma in modo più efficiente, mantenendo sia l’astrazione che il controllo a basso livello,
migliorando così la flessibilità e la performance complessiva del codice. \\

Basalt si distingue come l'unico linguaggio che offre un supporto alla riflessione paragonabile in termini di completezza e funzionalità a linguaggi come Java, 
pur mantenendo un ruolo funzionalmente analogo a quello del C o del C++. Al contempo, Basalt adotta un'estetica minimale e orientata alla semplicità, in linea con 
l'approccio stilistico di Go. Pertanto, questi quattro linguaggi costituiranno i principali riferimenti per il confronto nel prosieguo della discussione. \\

\newpage
\subfile{ConfrontoConAltriLinguaggi} \newpage\
\subfile{StrutturaDiUnProgrammaBasalt} \newpage
\subfile{IndipendenzaDallOrdineDiDefinizione} \newpage