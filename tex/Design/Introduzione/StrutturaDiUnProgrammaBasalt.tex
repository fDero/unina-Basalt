\subsubsection{Struttura di un programma Basalt}
Come precedentemente menzionato, Basalt si discosta dall'utilizzo di header files tipico di C e C++, optando invece per un sistema di gestione dei pacchetti simile a quello adottato da Java. In particolare, 
il sistema dei pacchetti di basalt prevede che all’intero di un file apparenente ad un dato pacchetto, sia visibile il contenuto pubblico e non di tutti gli altri file dello stesso pacchetto, e il contenuto 
pubblico dei package importati. \\

Ogni file sorgente contenente codice Basalt deve possedere una intestazione composta dalla dichiarazione del package corrente, ovvero il package a cui il file appartiene, e 
da una lista di package importati dal file, necessari al suo funzionamento. \\
 
Così come C, C++, Zig, Rust, Go, Jai, Odin e molti altri, il flusso di esecuzione parte da una chiamata fittizia ad una funzione speciale detta entry-point del programma. Così 
come da convenzione, tale funzione prende il nome di "main”. Tale funzione deve necessariamente essere in un package di nome "main”. \\


\begin{lstlisting}[frame=single]
package main;
import console;

func main() {
    println("Hello, World!");
}
\end{lstlisting}

\vspace{0.5cm}

In maniera analoga a quanto è possibile vedere in Java, in Basalt importare un package non 
è una precondizione necessaria per l'utilizzo delle funzioni di tale package. É infatti possibile
utilizzare la funzione \texttt{println} anche senza importare il package \texttt{console}, semplicemente riferendosi
a tale funzione con il suo nome completo: \\

\begin{lstlisting}[frame=single]
package main;

func main() {
    console::println("Hello, World!");
}
\end{lstlisting}
    